% -*- Mode:TeX -*-

%% IMPORTANT: The official thesis specifications are available at:
%%            http://libraries.mit.edu/archives/thesis-specs/
%%
%%            Please verify your thesis' formatting and copyright
%%            assignment before submission.  If you notice any
%%            discrepancies between these templates and the 
%%            MIT Libraries' specs, please let us know
%%            by e-mailing thesis@mit.edu

%% The documentclass options along with the pagestyle can be used to generate
%% a technical report, a draft copy, or a regular thesis.  You may need to
%% re-specify the pagestyle after you \include  cover.tex.  For more
%% information, see the first few lines of mitthesis.cls. 

%\documentclass[12pt,vi,twoside]{mitthesis}
%%
%%  If you want your thesis copyright to you instead of MIT, use the
%%  ``vi'' option, as above.
%%
%\documentclass[12pt,twoside,leftblank]{mitthesis}
%%
%% If you want blank pages before new chapters to be labelled ``This
%% Page Intentionally Left Blank'', use the ``leftblank'' option, as
%% above. 

\documentclass[12pt]{mitthesis}
\usepackage{lgrind}
\usepackage[version=3]{mhchem} % Formula subscripts using \ce{} 
\usepackage{textcomp}
\usepackage{siunitx}
\usepackage{marginnote}
\newcommand{\mymarginnote}[1]{%
  \marginpar{\colorbox{yellow}{\parbox{\marginparwidth}{%
  \setstretch{0.5}\textcolor{red}{\scriptsize{#1}}}}}}
%\renewcommand*{\marginfont}{\footnotesize}
%\usepackage{amsmath}
\pagestyle{plain}

%% This bit allows you to either specify only the files which you wish to
%% process, or `all' to process all files which you \include.
%% Krishna Sethuraman (1990).
\usepackage[backend=biber,doi=false,isbn=false,url=false,maxnames=10,firstinits,citestyle=numeric-comp,bibstyle=nature,sorting=none]{biblatex}
\bibliography{thesis.bib}

% Figure inclusion

\usepackage{graphicx}
\graphicspath{{../figures/}}

% Blank page definition

\newcommand{\blankpage}{
    \newpage
    \thispagestyle{empty}
    \mbox{}
    \newpage
}

%Allowance for TOC reporting and alphanumeric labeling of subsubsections.
\setcounter{secnumdepth}{4}
\setcounter{tocdepth}{4}
\renewcommand\thesubsubsection{\thesubsection\alph{subsubsection}}

% SI Units declaration
\usepackage{siunitx}
\sisetup{
    inter-unit-product=\ensuremath{{}\cdot{}},
    range-units=single,
    separate-uncertainty = true,
    multi-part-units=single
}
\DeclareSIUnit[number-unit-product = \;]
\gforce{{$\times$}\emph{g}}
\DeclareSIUnit[]
\wtperwt{\% w/w}
\DeclareSIUnit[]
\wtper{\% w/v}
\DeclareSIUnit[]
\volper{\% v/v}
\DeclareSIUnit[number-unit-product = \;]
\rpm{rpm}
\DeclareSIUnit[number-unit-product = \;]
\EU{EU}
\DeclareSIUnit[number-unit-product = \;]
\Da{Da}
\DeclareSIUnit[number-unit-product = \;]
\ppm{ppm}
\DeclareSIUnit[number-unit-product = \;]
\AU{AU}
\DeclareSIUnit[number-unit-product = \;]
\psi{psi}
\DeclareSIUnit[number-unit-product = \;]
\degtext{degrees}
\DeclareSIUnit[]
\mz{\emph{m/z}}

\usepackage{chemmacros}
%\usepackage{chemformula}
\usepackage{hyperref}
\usepackage{listings}
\usepackage{subcaption}
\usepackage{fixltx2e}

\usepackage{pdflscape}
\usepackage{pdfpages}

\usepackage{caption}
\captionsetup{font={stretch=1.5}}

\usepackage{fancyhdr}

\usepackage{titlesec}
\usepackage{multirow}
% using multiple rows 

% using A unit
\newcommand{\angstrom}{\mbox{\normalfont\AA}} 

\titleformat{\section}[block]
{\LARGE\bfseries\raggedright}{\thesection}{1em}{}

\begin{document}
 
% Some departments (e.g. 5) require an additional signature page.  See
% signature.tex for more information and uncomment the following line if
% applicable.

%\fancypagestyle{plain} {
    %\fancyhf{}
    %\renewcommand{\footrulewidth}{0pt}
    %\renewcommand{\headrulewidth}{0pt}
%}
\fancyhf{}
\fancyhead[R]{\thepage}
\renewcommand{\footrulewidth}{0pt}
\renewcommand{\headrulewidth}{0pt}
\pagestyle{fancy}
\doublespacing

% -*-latex-*-
% 
% For questions, comments, concerns or complaints:
% thesis@mit.edu
% 
%
% $Log: cover.tex,v $
% Revision 1.8  2008/05/13 15:02:15  jdreed
% Degree month is June, not May.  Added note about prevdegrees.
% Arthur Smith's title updated
%
% Revision 1.7  2001/02/08 18:53:16  boojum
% changed some \newpages to \cleardoublepages
%
% Revision 1.6  1999/10/21 14:49:31  boojum
% changed comment referring to documentstyle
%
% Revision 1.5  1999/10/21 14:39:04  boojum
% *** empty log message ***
%
% Revision 1.4  1997/04/18  17:54:10  othomas
% added page numbers on abstract and cover, and made 1 abstract
% page the default rather than 2.  (anne hunter tells me this
% is the new institute standard.)
%
% Revision 1.4  1997/04/18  17:54:10  othomas
% added page numbers on abstract and cover, and made 1 abstract
% page the default rather than 2.  (anne hunter tells me this
% is the new institute standard.)
%
% Revision 1.3  93/05/17  17:06:29  starflt
% Added acknowledgements section (suggested by tompalka)
% 
% Revision 1.2  92/04/22  13:13:13  epeisach
% Fixes for 1991 course 6 requirements
% Phrase "and to grant others the right to do so" has been added to 
% permission clause
% Second copy of abstract is not counted as separate pages so numbering works
% out
% 
% Revision 1.1  92/04/22  13:08:20  epeisach

% NOTE:
% These templates make an effort to conform to the MIT Thesis specifications,
% however the specifications can change.  We recommend that you verify the
% layout of your title page with your thesis advisor and/or the MIT 
% Libraries before printing your final copy.
\pagenumbering{roman}
\title{Detoxification of Organophosphates Using Stabilized Phosphotriesterase}

\author{Ching-Yao Yang}
% If you wish to list your previous degrees on the cover page, use the 
% previous degrees command:
%       \prevdegrees{A.A., Harvard University (1985)}
% You can use the \\ command to list multiple previous degrees
%       \prevdegrees{B.S., University of California (1978) \\
%                    S.M., Massachusetts Institute of Technology (1981)}
\department{Chemical and Biomolecular Engineering}

% If the thesis is for two degrees simultaneously, list them both
% separated by \and like this:
% \degree{Doctor of Philosophy \and Master of Science}
\degree{Doctor of Philosophy}
\degreetitle{Materials Chemistry}
\idnumber{N14332422}

% As of the 2007-08 academic year, valid degree months are September, 
% February, or June.  The default is June.
\degreemonth{January}
\degreeyear{2016}
\thesisdate{January 18th, 2016}

%% By default, the thesis will be copyrighted to MIT.  If you need to copyright
%% the thesis to yourself, just specify the `vi' documentclass option.  If for
%% some reason you want to exactly specify the copyright notice text, you can
%% use the \copyrightnoticetext command.  
%\copyrightnoticetext{\copyright IBM, 1990.  Do not open till Xmas.}

% If there is more than one supervisor, use the \supervisor command
% once for each.
\advisor{Prof. Jin Kim Montclare, Ph.D.}

% This is the department committee chairman, not the thesis committee
% chairman.  You should replace this with your Department's Committee
% Chairman.
\chairman{David Pine}{Chairman, Department Head}

% Make the titlepage based on the above information.  If you need
% something special and can't use the standard form, you can specify
% the exact text of the titlepage yourself.  Put it in a titlepage
% environment and leave blank lines where you want vertical space.
% The spaces will be adjusted to fill the entire page.  The dotted
% lines for the signatures are made with the \signature command.
\maketitle

\cleardoublepage

% -*- Mode:TeX -*-
%
% Some departments (e.g. Chemistry) require an additional cover page
% with signatures of the thesis committee.  Please check with your
% thesis advisor or other appropriate person to determine if such a 
% page is required for your thesis.  
%
% If you choose not to use the "titlepage" environment, a \newpage
% commands, and several \vspace{\fill} commands may be necessary to
% achieve the required spacing.  The \signature command is defined in
% the "mitthesis" class
%
% The following sample appears courtesy of Ben Kaduk <kaduk@mit.edu> and
% was used in his June 2012 doctoral thesis in Chemistry. 

\begin{titlepage}
\raggedright
Approved by the Guidance Committee:

\signatureentry{Professor Jin Ryoun Kim}{Associate Professor of Chemical
Engineering}{New York University}

\signatureentry{Professor Jin Kim Montclare}{Associate Professor of
Chemistry}{New York University}

\signatureentry{Professor Vikas Nanda}{Associate Professor of
Biochemistry}{Rutgers University}

\signatureentry{Professor Evgeny Vulfson}{Industry Professor of
Biotechnology}{New York University}

\end{titlepage}



\cleardoublepage

% --------------------------
\singlespacing
\noindent
Microfilm or copies of this dissertation may be obtained from:\\
\\
UMI Dissertation Publishing\\
ProQuest USA\\
789 E. Eisenhower Parkway\\
P.O. Box 1346\\
Ann Arbor, MI 48106-1346
% --------------------------

\cleardoublepage

\doublespacing
\section*{Vita}
Ching-Yao Yang was born in Taiwan on May 2\textsuperscript{nd}.  


\cleardoublepage

\section*{Acknowledgments}
This body of work encompasses seven years of effort that would not have been
possible without the professional and emotional support of my friends, family,
and newfound colleagues. I declare my immense thanks to them for the
following reasons.

First, to my family:\\
The continuous pride and understanding that my family has expressed in regards
to the pursuit of my doctorate has been the driving force behind my stalwart
efforts these past few years. They have continued to champion my marathon
despite my long-awaited presence at the finish line. And during the times of despair
and frustration that commonly plague scientific research, I have always been
able to benefit from the respite that I received in the sanctuaries that were
their homes. To all of them, I convey my eternal gratitude and love.

To my colleague, friend, and mentor - Jin Montclare:\\
From day one, you have continued to foster and promote my advancement as a
scientist and intellectual. Assuredly, I am a better person for it.  We have
celebrated academic successes as well as suffered scientific and professional
tribulations. But perhaps most importantly, regardless of the context, you have
undoubtedly and consistently demonstrated your utmost trust in both my abilities
and my character - a privilege for which I can only hope I have abundantly
reciprocated in kind. Thank you very much for the momentous opportunity to have
worked with you.

To my colleagues:\\
There are many individuals to whom I owe thanks for their professional support
during the course of my doctoral studies. They include fellow students (high
school, undergraduate, and graduate alike), accomplished scientists, aspiring
professors, and university administrators and staff. A vast majority of them
have provided me with an infinitely expansive sounding board for my ideas and
have in turn stimulated the advancement of my research. Others have quelled
quagmires before they escalated toward affecting the completion of my degree.
And others have simply promoted my advancement as a research scientist through
the extension of rare professional opportunities. 

To my close friends:\\
I submit my sincerest apologies, more so and in addition to my gratitude. A
graduate career of seven years has in truth robbed me of many occasions to spend
with my close personal friends. They have nevertheless continued to support
my pursuit and take pride in my endeavors.

And to my Jennifer:\\
Without an iota of doubt, my professional accomplishments would not
have been possible without the love and support that you have devoted to me. You
have kept me company during late night experiments. You have consoled me during
times of struggle and sadness. You have cheered my successes and announced them
atop mountains. You are my rock and my cushion. And in my seven-year pursuit of
knowledge of the unknown, I have come to only one surety - that I love you.

To all of you again - thank you.


% The abstractpage environment sets up everything on the page except
% the text itself.  The title and other header material are put at the
% top of the page, and the supervisors are listed at the bottom.  A
% new page is begun both before and after.  Of course, an abstract may
% be more than one page itself.  If you need more control over the
% format of the page, you can use the abstract environment, which puts
% the word "Abstract" at the beginning and single spaces its text.

%% You can either \input (*not* \include) your abstract file, or you can put
%% the text of the abstract directly between the \begin{abstractpage} and
%% \end{abstractpage} commands.

% First copy: start a new page, and save the page number.
\cleardoublepage
% Uncomment the next line if you do NOT want a page number on your
% abstract and acknowledgments pages.
% \pagestyle{empty}
%\setcounter{savepage}{\thepage}
\begin{abstractpage}
% $Log: abstract.tex,v $
% Revision 1.1  93/05/14  14:56:25  starflt
% Initial revision
% 
% Revision 1.1  90/05/04  10:41:01  lwvanels
% Initial revision
% 
%
%% The text of your abstract and nothing else (other than comments) goes here.
%% It will be single-spaced and the rest of the text that is supposed to go on
%% the abstract page will be generated by the abstractpage environment.  This
%% file should be \input (not \include 'd) from cover.tex.
With the advancement of technologies to probe and manipulate biophysical matter,
the scientific community continues to ever better engineer biological systems with the
complexity and elegance in design that is necessary to address 
biomedical challenges. The growing maturity of the field of
protein engineering is a testament to this proclamation.
%Within this field,
%modification of naturally occurring proteins as a manner of rests the 
Herein, two fundamental ideas are explored. In Chapter I, an evaluation is
presented on the effects of the incorporation of a non-canonical, fluorinated amino
acid into a protein-based block copolymer. The ramifications of these results,
and similar others in the field, on the promise for predictable tuning of the
physicochemical behavior and properties of protein-based materials are
emphasized.
In Chapter II, an alternative application of an endogenous protein is
examined, harnessing its inherent form and function. Hypotheses postulate
the ability of a coiled-coil protein, of particularly high oligomeric order, to
facilitate the delivery of small molecule therapeutics for the treatment of
osteoarthritis, whilst addressing dominant hurdles pertaining to drug
localization.
This complete body of work rests on the themes of control and repurposed
application of biophysical matter, contributing to the formalization of
engineered systems within protein science.

\end{abstractpage}

% Additional copy: start a new page, and reset the page number.  This way,
% the second copy of the abstract is not counted as separate pages.
% Uncomment the next 6 lines if you need two copies of the abstract
% page.
% \setcounter{page}{\thesavepage}
 %\begin{abstractpage}
 %% $Log: abstract.tex,v $
% Revision 1.1  93/05/14  14:56:25  starflt
% Initial revision
% 
% Revision 1.1  90/05/04  10:41:01  lwvanels
% Initial revision
% 
%
%% The text of your abstract and nothing else (other than comments) goes here.
%% It will be single-spaced and the rest of the text that is supposed to go on
%% the abstract page will be generated by the abstractpage environment.  This
%% file should be \input (not \include 'd) from cover.tex.
With the advancement of technologies to probe and manipulate biophysical matter,
the scientific community continues to ever better engineer biological systems with the
complexity and elegance in design that is necessary to address 
biomedical challenges. The growing maturity of the field of
protein engineering is a testament to this proclamation.
%Within this field,
%modification of naturally occurring proteins as a manner of rests the 
Herein, two fundamental ideas are explored. In Chapter I, an evaluation is
presented on the effects of the incorporation of a non-canonical, fluorinated amino
acid into a protein-based block copolymer. The ramifications of these results,
and similar others in the field, on the promise for predictable tuning of the
physicochemical behavior and properties of protein-based materials are
emphasized.
In Chapter II, an alternative application of an endogenous protein is
examined, harnessing its inherent form and function. Hypotheses postulate
the ability of a coiled-coil protein, of particularly high oligomeric order, to
facilitate the delivery of small molecule therapeutics for the treatment of
osteoarthritis, whilst addressing dominant hurdles pertaining to drug
localization.
This complete body of work rests on the themes of control and repurposed
application of biophysical matter, contributing to the formalization of
engineered systems within protein science.

 %\end{abstractpage}

%%%%%%%%%%%%%%%%%%%%%%%%%%%%%%%%%%%%%%%%%%%%%%%%%%%%%%%%%%%%%%%%%%%%%%
% -*-latex-*-

%\singlespacing
%\section*{Preface}
Your final comments regarding my thesis were very much appreciated.  Attention
was given to each point addressed in the committee letter, and the utmost
consideration was given to your critiques, reflected in this document.
For your convenience, I have paraphrased and itemized the issues
brought forth in the committee's letter in response to my defense presentation in the
following pages.  The manner in which each issue was addressed in this document
is also delineated. 

Please notify me at your earliest possible convenience if these changes are
satisfactory. Pending your confirmation, I will require your signature on final
bound copies of this document prior to Friday, September 19, 2014. If you can
let me know an available time on either Thursday, September 18, or Friday,
September 19, during which you can sign the bound copies, I would greatly
appreciate it.

Thank you very much again, and you have my most sincere gratitude for your
participation.

\hspace{0pt}\\
Sincerely,
\hspace{0pt}\\
\hspace{0pt}\\
Carlo Yuvienco

\begin{landscape}
% --------------------------
\renewcommand{\arraystretch}{1.5}
%START_TABLE
\begin{table}[h!]
    \centering
    \begin{tabular}{ p{0.64\textwidth} p{0.64\textwidth} }
    %\begin{tabular}{ p{0.5\textwidth} p{0.5\textwidth} }
    \hline
    \multicolumn{2}{c}{Chapter I Editorial Review} \\
    \hline
    \multicolumn{1}{c}{Critique} &
    \multicolumn{1}{c}{Manner of Address} \\
    \hline
    
    Calculate the thermodynamic values based on UV/Vis melt curve data.
    &
    Thermodynamic values are now calculated based on UV/Vis data (Section
    \ref{sec:thermo_method}) and reported in Sections \ref{sec:thermo_analysis},
    and \ref{sec:thermo_discussion}.
    \\

    Explain the observation of the \emph{p}FF-ECE exhibiting a loss in
    elasticity at higher concentrations with brightfield microscopy data.
    &
    Anomalous microrheological data pertaining to \emph{p}FF-ECE is addressed
    with further discussion (based on defense presentation points) and
    brightfield micrographs of sample groups (Section \ref{sec:ECE_explanation}).
    \\

    Provide a path forward as to how solid state experiments can lead to imaging
    using \textsuperscript{19}F MRI.
    &
    Additional discussion is provided on future potential for
    \textsuperscript{19}F MRI based on solid-state data presented in the defense
    presentation and in Section \ref{sec:ss_nmr}.
    \\

    \hline
\end{tabular}
\end{table}
%END_TABLE
% --------------------------
\renewcommand{\arraystretch}{1.5}
%START_TABLE
\begin{table}[h!]
    \centering
    \begin{tabular}{ p{0.64\textwidth} p{0.64\textwidth} }
    %\begin{tabular}{ p{0.5\textwidth} p{0.5\textwidth} }
    \hline
    \multicolumn{2}{c}{Chapter II Editorial Review} \\
    \hline
    \multicolumn{1}{c}{Critique} &
    \multicolumn{1}{c}{Manner of Address} \\
    \hline
    
    Using Beer-Lambert's law, you should be able to calculate the concentration
    of the sample and assess what is really going on with zonal elution
    experiments; it may be that perhaps the protein is sticking to the column or
    surface.
    &
    Beer-Lambert calculations of protein concentration based on chromatogram
    integrations was attempted, but considered incompatible with the dataset
    based on the limited collection of absorption data at \SI{280}{\nm} and the
    lack of Trp, Tyr, and Cys residues in the sequence. Discussion is however
    added to Section \ref{sec:ze_premature_assembly} that addresses the possible
    explanation of protein adherence to the LC system for the evolution of
    time-course chromatograms. 
    \\

    The estimate that the \SI{3200}{\Da} represents 8 molecules of BMS493 from
    the calibration curve may not be valid as the error is within
    \SI{10}{\percent}.
    &
    Conclusions of the exact molar ratios of binding based on SEC data have been
    re-evaluated to reflect the discussion points regarding measurement error
    during the defense presentation (see Section \ref{sec:discuss_binding}).
    \\

    Endotoxin contamination might be the cause for the catabolic response and
    that is one possibility under investigation; is there a positive and
    negative control that you could devise and test?
    &
    Future work discussion is added to the document (see Section
    \ref{sec:future_work_endotoxin}) that addresses this point.
    \\

    Are there experiments that you might be able to design to assess the ability
    for COMPcc alone to sequester ATRA leading to OA treatment?
    &
    Future work discussion is added to the document (see Section
    \ref{sec:future_work_ATRA_sequestration}) that addresses this point.
    \\

    Indicate alternative approaches to the Hummer-Dreyer method as presently it
    has not been validated as a suitable approach for determining binding
    affinities for BMS493.
    &
    Sedimentation equilibrium is proposed and discussed as a viable alternative
    approach to evaluate binding affinities of oligomeric COMPcc to BMS493 (see
    Section \ref{sec:discuss_binding}).
    \\

    \hline
\end{tabular}
\end{table}
%END_TABLE
% --------------------------
\end{landscape}
\renewcommand{\arraystretch}{1}

\doublespacing
  % -*- Mode:TeX -*-
%% This file simply contains the commands that actually generate the table of
%% contents and lists of figures and tables.  You can omit any or all of
%% these files by simply taking out the appropriate command.  For more
%% information on these files, see appendix C.3.3 of the LaTeX manual. 
\tableofcontents
\newpage

\listoffigures
\newpage

\listoftables


\pagenumbering{arabic}
\chapter{Improved Stability and Half-Life of Fluorinated Phosphotriesterase
Using Rosetta} 
\label{chap:rosetta}

\begin{refsection}

\section{Introduction}

\subsection{Rosetta and Protein Engineering}
\label{sec:rosetta}

% Introduce Rosetta and protein engineering. Also focus on protein engineering.
Computational tools are widely used for protein engineering [ref]. Rosetta
suite was first developed in University of Washington [ref]. Baker et al.
adapted this suite for prediction of three: dimensional structure of proteins.
This suite provides a handful of protocols for analyzing and mutating protein
structures. The simulation replies heavily on knowledge-based potentials. It is
a suite of libraries and tools for macromolecular ligand docking, to
thermo-stabilize proteins, to design a hydrogen-bond network, to design novel
protein folds, to create novel protein interfaces, and to design enzymes,
including some containing unnatural amino acids in the active sites.

\subsection{Phosphotriesterase} 
\label{sec:pte}

% This section focuses on PTE.
PTE is a homodimeric protein composed of two monomers, each of which contains a
metallo-active site.Phosphotriesterase (PTE) are enzymes, which hydrolyze
organophosphates (OPs) as well as synthetic esters.\cite{Ghanem2005a} OPs are a
synthetic class of small molecule that irreversibly inactivate
acetylcholinesterase (AChE), disrupting neural transmission. AChE is an enzyme
that degrades the neurotransmitter, acetylcholine, at the neuromuscular
junction in the cholinergic nervous system. After the acetylcholine is
hydrolyzed, the synaptic transmission would be terminated. Inhibition of AChE
lead to hyper-stimulation from toxic accumulation of
acetylcholine.\cite{Soreq2001} Army also adapted this protein for chemical
weapons neutralization. \cite{Yang2014a}

\subsection{Incorporation of Non-natural Amino Acids}
\label{sec:rsi}

Several methods have been developed for the incorporation of unnatural amino
acids into proteins: solid-phase synthesis (SPPS)\cite{Mahto2011}, in vivo and
in vitro site-specific incorporation, 16 and residue-specific incorporation
(Fig. 1)1d, 17. In SPPS, activated amino acids are immobilized on a solid
support and synthesized step-by-step in the reactant solution. This method is
convenient for the introduction of functional groups into peptides, but it is
still restricted to the yield and the expense of peptides. For example, if each
coupling step has 99\% yield, a 26-amino acid peptide would be synthesized in
77\% final yield. To synthesis longer chain peptides and proteins bearing UAAs,
biosynthetic methods have been developed.  There exists two contemporary
methods to biosynthetically incorporate non-natural amino acids into proteins:
site-specific incorporation and residue-specific incorporation. Schultz and
their coworkers18 have developed a general approach for the in vitro synthesis
of proteins. The approach relies on the suppression of an amber termination
codon (UAG) in the mRNA by an amber suppressor tRNA charged with the amino acid
analog. This method has been well studied and developed in research of protein
structures and functions19. 

Methods to incorporate amino acid analogues site-specifically into proteins in
vivo greatly expand research of unnatural amino acids. We are not only able to
synthesize large amounts of protein, but capable of overcoming potential
problems including post translational modifications. An in vivo site-specific
method to incorporation UAAs was developed by Schultz and coworkers21. A stop
codon at the position of interest is encoded in the mRNA. For in vivo
site-specific UAA incorporation, an orthogonal aminoacyl-tRNA synthetase
charges an orthogonal tRNA with particular UAA, and the suppressor tRNA would
help the incorporation of UAA with recognition of a stop codon. As cells
contain 20 aminoacyl-tRNA synthetase/suppressor tRNA pairs, a new one is
required for the incorporation. An orthogonal aminoacyl-tRNA
synthetase/suppressor tRNA pair based on a TyrRS/tRNATyr pair in the
\emph{Methanococcus jannaschii} has been engineered for use in \emph{E. coli}
for the incorporation of tyrosine analogs21a.

As an alternative to site-specific incorporation, residue-specific
incorporation has been developed in which a natural amino acid is replaced with
an UAA. Auxotrophic strains or organisms that cannot biosynthesize a particular
natural amino acid, has been used to introduce multiple UAAs throughout the
protein sequence. UAAs that are isosteric to natural amino acids are capable of
being recognized by the natural aminoacyl-tRNA synthetase (aaRS), charging the
appropriate tRNA enabling the introduction of UAA into the protein sequence
without alteration of the biosynthetic machinery. However, to introduce UAAs
with gross differences from the natural amino acids, further engineering of the
aaRS is required. To incorporate refractory methionine analogs, Tirrell and
coworkers engineered additional copied of the methionyl-tRNA synthetase (MetRS)
by adding the MetRS gene under constitutive promotor22. Alternatively, Schimmel
and coworkers mutated editing pocket of valyl-tRNA synthetase (ValRS) to
facilitate the incorporation of analogs that normally would not be accepted by
endogenous aaRS23. Finally, Kast and coworkers generated a mutated
phenylalanyl-tRNA synthetase (PheRS), ePheRS* under a constitutive promoter,
with a large binding pocket (T251G) and showed relaxed specificity.24

\subsection{Fluorinated Amino Acids In Proteins} 
\label{sec:faa}

% Describe the properties of faa in the proteins.
Fluorinated amino acids (FAAs), represent a unique class of UAAs. They have
different bond energies, electron distributions, and hydrophobzgicity 26 as
compared to their hydrogenated counterparts. As we compare the structure of
fluorocarbon groups, t(Bh)e C-F bond is highly dipolar while the hydrocarbon is
less. The C-F bond is roughly 0.24 {\AA} longer than C-H bond26 (Table 1). While in
some cases the global replacement of hydrophobic amino acids with fluorinated
analogs has led to the stabilization of protein structure27, it has all been
shown that in some cases they can reduce the thermodynamic stability28.

\subsection{Scope of Work}

The primary goals of this work were to done adapt Rosetta for
phosphotriesterase. Overall, with incorporation of \emph{p}FF into protein, we
will be able evaluate the performance of scoring function. In advance, we would
evaluate the shelf life and thermo-stability of phosphotriesterase.

\section{Methods}

\subsection{General}

All chemicals, reagents, and substrate were purchased from Sigma. T4 DNA ligase
was purchased from Roche. DNA sequence was confirmed by Eurofins MWG Operon.
96-well plates were purchased from Thermo Fisher Scientific (Waltham, MA).

\subsection{Recombinant Gene Construction}

pQE30-S5 was used as described before.\cite{Baker2011} The pQE30-104A plasmid
was prepared with forward primers (5’-GATGTGTCGACTGCCGATATCGGTCG-3’, Fisher
Scientific), reverse primers (5’-CGACCGATATCGGCAGTCGACACA-3’, Fisher
Scientific). The PCR parameters were set as follow for 18 cycles: initial
denaturation in \SI{95}{\celsius} for 30 seconds, sequential denaturation in
\SI{95}{\celsius} for 30 seconds, annealing in \SI{55}{\celsius} for 1 minute,
and extension in \SI{68}{\celsius} for 4 minutes. The mixture was then
incubated \SI{37}{\celsius} overnight with DpnI to digest methylated parent DNA
strands, which lack the desired mutation. DNA sequence was further confirmed by
Eurofins MWG Operon.

\subsection{Protein Expression}
Mutant and wild type plasmids were transformed into \latin{E. coli} phenylalanine
auxotrophic strains (AF-IQ cells).[5] Electroporation was done at
\SI{25}{\micro\farad}, \SI{100}{\ohm}, 2.5 kV (Biorad Gene Pulser II). Cells were
plated on agar plates containing 200 μg/mL ampicillin, 34 μg/mL
chloramphenicol. A Single colony was picked and grown in medium (M9 medium
supplemented with 0.2 wt \% glucose, 35 mg/L thiamine, 1mM \ch{MgSO4}, 0.1 mM
\ch{CaCl2}, 200 μg/mL ampicillin, and 34 μg/mL chloramphenicol) with 20 mg/L of
20 amino acids at \SI{37}{\celsius}, 300 r.p.m.  Afterwards, 250 mL of M9
medium for large-scale expression was innoculated 1:50 with an overnight
culture. After optical density reached 1.0 at 600 nm, media shift was carried
out by washing the cells three times with 0.9\% \SI{4}{\celsius} \ch{NaCl}.
Cells were then transferred to M9 minimal medium containing either 20 amino
acids or 19 amino acids (-Phe). \emph{p}FF-PTE and \emph{p}FF-104A expression
media were supplemented with and 3 mM of \emph{p}FF and 1 mM
isopropyl-$\beta$-D-thiogalactopyranoside (IPTG) to induce protein expression.
1mM of \ch{CoCl2} was added in each post-induction medium. After three hours
incubation at 37 °C, 300 r.p.m., the cells were harvested and then resuspended
with 20 mM Tris-HCl, 500 mM \ch{NaCl}, 5 mM imidazole, 10\% glycerol (pH 8.0)
and \SI{1}{\micro\moLar} \ch{CoCl2}. Cell lysate was sonicated on ice for 1.5
minutes and then a clarification spin was performed (20, 000 g,
\SI{4}{\celsius}, 30 min).  Clarified supernatants were loaded into a His
Trap column (G.E Healthcare, Piscataway, NJ) using ÄKTA FPLC purifier (G.E.
Healthcare, Piscataway, NJ).  Protein elution was generated using elution
buffer B (20 mM Tris- HCl, 500 mM sodium chloride, 500 mM imidazole (pH 8.0)).
The purified samples were then transferred for buffer exchange using
\SI{12}{\liter} 20 mM phosphate buffer (pH 8.0).  Dialyzed protein was
subjected to kinetic assays immediately.

\subsection{Thermo-stability and Secondary Structure of Phosphotriesterase}
\label{sec:thermo}

\subsubsection{Nano-DSC}

DSC (Nano-DSC, TA instrument, USA) was preformed by using 600 μL (0.1 mg/mL) of
protein right after dialysis. Measurements were conducted at a scan rate of
\SI{1}{\celsius}/min. Signals was blanked with buffer under the same condition.
The observed diagram was then analyzed by using NanoAnalyze software.

\subsubsection{Circular Dichroism}

CD spectra were recorded on a JASCO J-815 Spectropolarimeter (Easton, MD) using
Spectra Manager software. Temperature was controlled using a Fisher Isotemp
Model 3016S water bath. Proteins concentrations were 10 μM in 20 mM phosphate
buffer (pH 8.0). 20 mM phosphate buffer was used for blanking signals. To
calculate ellipticities, the following formula was used:
% use equation here
θmrw = MRW(θobs) / (10 * c * l)
where \emph{MRW} is the mean residue weight of the specific phosphotriesterase, θobs
is the observed ellipticities (mdeg), \emph{l} is the path length (cm), \emph{c} is the
concentration in μM. Spectra was recorded from 190 nm to 250 nm with a scan
speed of 1 nm/min.

\subsection{Enzyme Kinetics}

The protein was diluted to a final concentration of 30 nM in 20 mM sodium
phosphate (pH 8.0) by using the extinction coefficient 29,280
M\textsuperscript{-1} cm\textsuperscript{-1}. Reactions were monitored
spectrophotometrically (Synergy H1, Bio-Tek, Winooski VT) at 405 nm for
paraoxon (coefficient = 17,000 M\textsuperscript{-1}  cm\textsuperscript{-1} ).
Reactions for paraoxon (13 – 104 μM) was done in 0.4\% methanol. KM and kcat
values were determined by a Lineweaver-Burk plot (1/v vs 1/[S]).[5] The
equation used is shown below: 
% use equation here 
1/v = (KM/ VMax) ? (1/[S]) + 1/VMax S2
where [S] represents substrate concentration; KM represents the substrate
concentration at which the reaction rate is half of Vmax. The data reported is
the average of three trials and the error represents the standard deviation of
those trials.

\subsection{MALDI-TOF Mass Spectrometry}

To determine level of \emph{p}FF incorporation, \SI{20}{\micro\liter} of
purified PTE pFF-PTE, F104A, or \emph{p}FF-104A was incubated with 12.5 ng/μL
of trypsin solution (in 50 mM of ammonium bicarbonate) at \SI{37}{\celsius}
overnight. 2 μL of 10\% trifluoroacetic acid (TFA) was used to quench each
reaction. Reaction was then purified with a C\textsubscript{18} packed zip-tip
(Millipore, Billerica, MA).  Tips were wetted in 50\% acetonitirile (ACN),
equilibrated in 0.1\% TFA, and eluted with 0.1\% TFA in 75\% ACN. Matrix was
dissolved in 10 mg/mL $\alpha$-cyano-4-hydrocinnamic acid (CCA) in 50\% ACN,
0.05\% TFA. Theoretical trypsin digest were calculated from Peptide Mass
(www.expasy.org/tools/peptide-mass.html). Samples were added to the matrix at a
1:1 ratio and spotted on MALDI plate. Five standards were spotted separately
for calibration: angiotensin I (MW = 1295.69 g/mol), neurotensin (MW =
1671.92g/mol), ACTH (1-17) (MW = 2092.09 g/mol), ACTH (18-39) (MW = 2464.20
g/mol), and ACTH (7-38) (MW =3656.93 g/mol). Compass 1.4 for flex software was
then used to analyze the MALDI spectra (www.bruker.com/).

\section{Results and Discussion}

\subsection{Biosynthesis of phosphotriesterase}

The \emph{p}FF-F104A variant and the \emph{p}FF-PTE parent were biosynthesized
by residue-specific incorporation with the phenylalanine auxotrophic
\emph{Escherichia coli} strain AFIQ.[6] As controls, the non-fluorinated
counterparts, PTE and F104A, were expressed under conventional conditions. As
expected, all four proteins exhibited good expression in the presence of
phenylalanine or \emph{p}FF. \emph{p}FF-F104A and \emph{p}FF-PTE exhibited 80
and 92\% incorporation, respectively, as determined by MALDI-TOF mass
spectrometry. Notably, purified yields of \emph{p}FF-F104A were twofold higher
than for \emph{p}FF-PTE, thus indicating more soluble protein yield.

\subsection{Thermo-stability And Secondary Structure}

Circular dichroism (CD) was performed to determine whether the mutation had an
impact on the overall secondary structure and stability. Far-UV wavelength
scans of \emph{p}FF-F104A and \emph{p}FF-PTE showed a double minimum at 208 and
222 nm (\SI{25}{\celsius}), as expected for a
($\beta$/$\alpha$)\textsubscript{8}-barrel protein, thus suggesting that the
mutation did not affect the overall structure (Figure S3).  Surprisingly,
comparison of the non-fluorinated counterparts revealed that F104A was less
structured than PTE (Figure S3). To assess the stability, differential scanning
calorimetry (DSC) was performed (Figure 2). Upon heating the sample from 0 to
\SI{70}{\celsius}, \emph{p}FF-PTE exhibited two transitions
(T\textsubscript{m}1: 42.0 $\pm$ 0.1; T\textsubscript{m}2 : 48.6 ± 0.2 ; Table
S1); this is consistent with our previous studies.[6] This biphasic unfolding
was also observed by Grimsley et al. in a study of organophosphorus hydrolase
(EC 3.1.8.1), and was attributed to the presence of a dimeric unfolded
intermediate.[17] In contrast, \emph{p}FF-F104A exhibited a single transition
at (49.7 ± 0.2) 8C, which was higher than both \emph{p}FF-PTE values (by 7.7
and 1.18C; Figure 2B, Table S1).  Remarkably, after heating, pFF-F104A retained
the single Tm of (49.2 ± 0.1) \SI{8}{\celsius}, thus demonstrating regaining of
structure after undergoing thermal unfolding.  In the absence of \emph{p}FF,
F104A demonstrated two transitions similar to \emph{p}FF-PTE (Figure 2A), thus
suggesting that fluorination was critical for stability.  These data
demonstrate the overall thermodynamic stability of \emph{p}FF-F104A.

\subsection{Enzymatic Kinetics of PTE}

\subsection{Protein Design}

Although methods enabling the biosynthesis of artificial proteins bearing NCAAs
are abundant,[1] tools to help further improve the overall activity and
stability are needed. Mutagenesis and evolutionary approaches have been
employed successfully to identify variants with enhanced function; however,
these rely heavily on testing or screening several to millions of
constructs.[3c, 5, 20] We demonstrate the use of computational methods to
identify a fluorinated protein variant that exhibits superior heat stability
and half-life. Notably, the \emph{p}FF-F104A variant is only functional in the
fluorinated form, thus validating Rossetta-based design with \emph{p}FF. This
provides another useful tool for protein design and could be employed in
conjunction with the abov ementioned approaches.

\subsection{Future Work}

\printbibliography[heading=subbibliography]

\end{refsection}

\chapter{Effects of Phenylalanines Outside Dimer Interface of Phosphotriesterase}
\label{chap:dimer}
\begin{refsection}

\section{Introduction}

\subsection{Phosphotriesterase}

PTE is a homodimeric protein composed of two monomers, each of which contains a
metallo-active site. Phosphotriesterase (PTE) are enzymes, which hydrolyze
organophosphates (OPs) as well as synthetic esters (Figure
\ref{fig:pte-structure})\cite{Ghanem2005a}. The proenzyme form of PTE contains
29 amino acids signal peptide at the N-terminus. It is originally found as a
39kDa monomeric form in the solution\cite{Mulbry1989}. Later, the proenzyme of
PTE is engineered and expressed in the form of mature protein from \latin{E.
coli}. A ($\beta$/$\alpha$)\textsubscript{8} TIM-barrel structure forms the
monomeric PTE\cite{Roodveldt2005,Seibert2005}. The globular monomer is roughly
51\AA $\times$ 55\AA $\times$ 51\AA.  OPs are a synthetic class of small molecule
that irreversibly inactivate acetylcholinesterase (AChE), disrupting
neural transmission. AChE is an enzyme that degrades the neurotransmitter,
acetylcholine, at the neuromuscular junction in the cholinergic nervous system.
After the acetylcholine is hydrolyzed, the synaptic transmission would be
terminated. Inhibition of AChE lead to hyper-stimulation from toxic
accumulation of acetylcholine\cite{Soreq2001}. Army also adapted this protein
for chemical weapons neutralization \cite{Yang2014a}.

\subsection{Dimer Interface of Phosphotriesterase}



\subsection{Metal Ions Effects At Active Site}

\subsection{Side-chain Effects}

\section{Methods}

\subsection{General}

All chemicals, reagents, and substrate were purchased from Sigma. T4 DNA ligase
was purchased from Roche. DNA sequence was confirmed by Eurofins MWG Operon.
96-well plates were purchased from Thermo Fisher Scientific (Waltham, MA)\cite{Yang2014a}.

\subsection{Rosetta Design of Phosphotriesterase}

A symmetric starting model of wild type PTE from the B chain of PDB structure
1HZY\cite{Benning2001a} was built using the Rosetta suite of macromolecular
modeling tools\cite{Leaver-Fay2011}. Both active site \ch{Zn^{2+}} ions were
replaced with \ch{Co^{2+}} to reflect the metal used in the experimentally
produced mutants.  Distance constraints between the cobalt cations and the
coordinating residues were taken from PDB structure 3A4J\cite{Jackson2009b}.
Torsional and partial charge parameters for the non-standard carboxylated
lysine residue (Lys 169) were calculated quantum mechanically using the
HF/6-31G(d) level of theory in Gaussian09\cite{Frisch2009} with an overall
charge of -1.  Rotamer libraries for the carboxylated lysine were generated
with the Rosetta MakeRotLib\cite{Renfrew2012b} protocol.  Models were
constructed for each of the point mutations: F51L, F150M, F216L, F304L, F306L,
F327L, F335M, and F357L using the Rosetta fixbb (fixed backbone design)
protocol with symmetry\cite{DiMaio2011a}.  \ch{Co^{2+}} coordinating residues
were held fixed to their native rotamers. To propagate point mutation effects
throughout a mutant model, the Rosetta relax protocol was used to repack and
minimize the entire PTE structure with backbone flexibility. For each point
mutant, an ensemble of 500 relaxed decoys were generated. Interatomic distances
between \ch{Co^{2+}} and coordinating residues were enforced with harmonic
constraints.  The change in stability for a mutation was calculated as the
difference between the mutant and wild type ensemble averages of the total
Rosetta score. All protocols used here included the native rotamers and extra
rotamers sampling as additional parameters. All decoys were scored using the
talaris2013 score function\cite{Leaver-Fay2013a}.

\subsection{Variants of Phosphotriesterase}
Purified protein product was assayed for concentration by way of a Thermo

\subsection{Biosynthesis}

In anticipation for the need of large quantities of protein mass for the \ldots
delete comtent of the supernatant and storage at \SI{-20}{\celsius}.

\subsection{Protein Purification}
The purification was described previously in the section
\ref{sec:protein-expression-method}. All solutions used in the extraction and
purification of recombinant proteins \ldots delete content 5 CVs of buffer
prior to each injection.

\subsection{Enzyme Kinetics}

The protein was diluted to a final concentration of \SI{30}{\nano\Molar} in
\SI{20}{\milli\Molar} sodium phosphate (pH 8.0) by using the extinction
coefficient \SI{29280}{\per\Molar\per\cm}. Reactions were monitored
spectrophotometrically (Synergy H1, BioTek, Winooski VT) at \SI{405}{\nm} for
paraoxon (coefficient = \SI{17000}{\per\Molar\per\cm}).  Reactions for paraoxon
(\SIrange{13}{104}{\micro\Molar}) was done in 0.4\% methanol.
K\textsubscript{M} and k\textsubscript{cat} values were determined by a
Lineweaver-Burk plot.\cite{Baker2011b} The equation used is shown below
(Eq.~\ref{eqn:MM-chap2}): 
\begin{equation} 
    \frac{1}{v} =
    \frac{K\textsubscript{M}}{V\textsubscript{max}}\times\frac{1}{S} +
    \frac{1}{V\textsubscript{max}} 
    \label{eqn:MM-chap2}
\end{equation}
where S represents substrate concentration; K\textsubscript{M} represents the
substrate concentration at which the reaction rate is half of
V\textsubscript{max}. The data reported is the average of three trials and the
error represents the standard deviation of those trials.

\subsection{Thermo-stability and Secondary Structure of Phosphotriesterase}

\subsubsection{Nano-DSC}

The details are described in the section \ref{sec:dsc-method}. DSC (Nano-DSC,
TA instrument, USA) was preformed by using \SI{600}{\micro\L}
(\SI{0.1}{\mg\per\mL}) of protein right after dialysis. Measurements were
conducted at a scan rate of \SI{1}{\celsius\per\minute}. Signals was blanked with
buffer under the same condition.  The observed diagram was then analyzed by
using NanoAnalyze software.

\subsubsection{Circular Dichroism}

The details are described in the section \ref{sec:cd-method}. CD spectra were
recorded on a JASCO J-815 Spectropolarimeter (Easton, MD) using Spectra Manager
software. Temperature was controlled using a Fisher Isotemp Model 3016S water
bath. Proteins concentrations were \SI{10}{\micro\Molar} in
\SI{20}{\milli\Molar} phosphate buffer (pH 8.0). \SI{20}{\milli\Molar}
phosphate buffer was used for blanking signals. To calculate ellipticities, the
following formula was used(Eq.~\ref{eqn:CD-chap2}): 
\begin{equation}
    θmrw = MRW(θobs) / (10 * c * l)
    \label{eqn:CD-chap2}
\end{equation}
where \emph{MRW} is the mean residue weight of the specific phosphotriesterase,
θobs is the observed ellipticities (mdeg), \emph{l} is the path length (cm),
\emph{c} is the concentration in \SI{}{\micro\Molar}. Spectra was recorded from
\SIrange{190}{250}{\nm} with a scan speed of \SI{1}{\nano\meter\per\minute}.

\section{Results}

\subsection{DNA Alignments And PTE Variants}

Expression of COMPcc was carried out in auto-inducing media, encompassing a
\ldots delete content typical in inducible expression systems.

Auto-induction growth can be sustained in baffled shaker flasks, according to
\ldots delete content to purification.

\subsection{Variants Expression And Purification}

\ldots lots content delete here\ldots potential for COMPcc to behave as an
inhibitor of hypertrophic differentiation.

\subsection{Variants Enzyme Kinetics}

Endotoxin levels of the protein were measured using a limulus amebocyte lysate
catabolic events reported in Section.

\subsection{Thermo-stability and CD of PTE Variants}

\section{Discussion}

\printbibliography[heading=subbibliography]

\end{refsection}

%\chapter{Formulation of Phosphotriesterase Using $\alpha$-Lactose Monohydrate} 
\label{chap:lactose}

\begin{refsection}

\section{Introduction}

\subsection{Applications of Phosphotriesterase}

The unique catalytic properties of enzymes led to their rapid exploitation in
several industries as discussed in chapter 1. However, it quickly became
apparent that there were limitations to this technology owing to the
denaturation or inactivation of enzymes brought about by heat, proteolysis,
action of organic solvents, etc.  There was a major incentive to find solutions
to these problems in order to take advantage of enzymes active, specificity and
other attractive features.  Even in the early 70s there were already several
good reviews on the development of the new biochemistry based on immobilization
procedures.224-226

Entrapment is the method that has had the greatest use in large-scale enzyme
operations. Since most enzymes used commercially are intracellular, the
simplest method, which is not shown in the figure, is to use whole cell
preparations where the enzyme is never released from the bacteria in which it
is produced. One of the best examples of this is glucose isomerase, which has
been used in the commercial production of high fructose corn syrup (HFCS) since
1967.230 Most of the glucose isomerase used in the production of over 6,000,000
tons of HFCS per year is in the form of immobilized whole cells. 2 3 1 This is
often done by spray drying the harvested cells to give a granulated product
that is then treated with polymeric materials (such as polyethylenimine) to
stabilize them. To further improve reactivity, the cells may be permeabilized.
This removes the barrier for the free diffusion of the substrate/product across
the cell membrane, and also empties the cell of most of the small molecular
weight cofactors, etc., thus minimizing unwanted side reactions. Obviously,
this and all other entrapment techniques are most applicable for low molecular
weight substrates and simple bioconversions like hydrolysis, isomerization and
oxidation reactions that do not require a cofactor-regeneration system. 23 2

It is possible to use entrapment as a mean of immobilizing a cofactor requiring
system. This can be accomplished in several ways. Initial efforts involved
covalently attaching the cofactor (usually NAD or NADH) directly onto a solid
support or in a gel matrix along with the enzymes.233 However, while the
enzymes and cofactor would be stable, the level of activity was generally quite
low. Other researchers attached the cofactor directly to the enzyme by means of
a bifunctional linker such as modified polyethylene glycol (PEG). The linker
would ideally be short enough to keep the cofactor close to the active site.
However, it would also need to be long and flexible enough to permit its easy
entrance in and out of the active site.234 The most commonly used method has
been to attach the cofactor to PEG or other large polymer. This is then placed
with the enzyme(s) in ultrafiltration systems with a semipermeable membrane,
microencapsulated, or immobilized in membranes. In most cases, the cofactor
would be regenerated with a second enzyme such as formate or alcohol
dehydrogenase, for which a second substrate is required. When the process is
being used for the synthesis of a valuable product, this can be cost effective.
However, for use in biodegradation such an approach would be too expensive. Two
enzymes described recently could simplify the system and reduce costs. The
first involves a hydrogenase that in the presence of hydrogen can regenerate
NADH.235 This is still somewhat involved since hydrogen would need to be
provided. An even more interesting enzyme is an NADH Oxidase that can use
dissolved oxygen to regenerate NAD with the production of water.2 3 6 Thus,
only aeration of the system would be required and no unwanted products would be
generated.

While entrapment in a gel matrix has been extensively used for the
immobilization of cells, it has not been as common for free enzymes. The major
limitation of this technique for enzymes is the possible slow leakage during
continuous use.237 The primary natural polymers used for entrapment have been
agar, agarose and gelatin through thermoreversal polymerization, and alginate
and carrageenan by ionotropic gelation. In addition to possible enzyme leakage,
these are relatively soft materials that will deform in large packed columns.
They are also subject to deterioration if used in fluidized bed reactors. In
the case of alginate and carrageenan, the ionic species used in the
polymerization (usually Ca2+) needs to be present continuously to maintain the
integrity of the gels. With some enzymes and processes, this can be a problem.

In the past, members of this laboratory have tested lactose's ability to
incorporate all kinds of molecules: large organic dyes, green fluorescent
protein, bovine serum albumin, horse-radish peroxidase, and now
phosphotriesterase (PTE). PTE is a dimeric protein composed of identical
subunits which come together in the protein's active state. PTE is capable of
hydrolyzing a wide range of organophosphates, and has been shown to work in the
breakdown and neutralization of pesticides and herbicides, as well as nerve
agents like sarin gas. This makes PTE an excellent target for kinetic
stabilization and storage by lactose - consider field medics in a warzone,
stocked with tablets of PTE stored in lactose crystals, or the application of
PTE/LM crystals to the processing of crops or other chemically treated
foodstuff.

\subsection{Lactose Monohydrate}

Lactose (4-O-β-D-galactopyranosyl-D-glucopyranose, \ce{C12H22O11} ) is a
disaccharide consisting of a D - glucose and a D - galactose molecule joined by
a β - 1,4 - glycosidic linkage. (Figure \ref{fig:lactose-structure})  It has
been previously shown by members of this laboratory that $\alpha$-lactose
monohydrate (LM) is capable of incorporating various macromolecules and
biopolymers into its crystal structure; it has been reasoned that this is made
possible by LM's abundance of peripheral hydrogen atoms, which are capable of
hydrogen bonding to said macromolecules, trapping them, orienting them, and
ultimately allowing the forming crystal to overgrow and absorb them1.

% --------------------------
\begin{figure}[h!] \centering \includegraphics[width=0.5\textwidth]{fig3_01}
    \caption[Molecular structures of $\alpha$- and $\beta$- lactose.]{Molecular
    structures of $\alpha$- and $\beta$- lactose.}
    \label{fig:lactose-structure}
\end{figure}
% --------------------------

In the dairy industry, crystallization is an important separation
process used in the refining of lactose from whey solutions. In the refining
operation, lactose crystals are separated from the whey solution through
nucleation, growth, and/or aggregation. The rate of crystallization is
determined by the combined effect of crystallizer design, processing
parameters, and impurities on the kinetics of the process. This review
summarizes studies on lactose crystallization, including the mechanism, theory
of crystallization, and the impact of various factors affecting the
crystallization kinetics. In addition, an overview of the industrial
crystallization operation highlights the problems faced by the lactose
manufacturer. The approaches that are beneficial to the lactose manufacturer
for process optimization or improvement are summarized in this review. Over the
years, much knowledge has been acquired through extensive research. However,
the industrial crystallization process is still far from optimized. Therefore,
future effort should focus on transferring the new knowledge and technology to
the dairy industry.

% --------------------------
\begin{figure}[h!] \centering \includegraphics[width=0.7\textwidth]{fig3_02} 
    \caption[Crystals of lactose monohydrate (LM) as hosts for the guest green
    fluorescent protein (GFP)]{Crystals of lactose monohydrate (LM) as hosts
    for the guest green fluorescent protein (GFP)}
    \label{fig:lm-intro}
\end{figure}
% --------------------------

\subsubsection{Protein conjugation}

% This is such a depressing section to write about. I hate doing this shitty
% research in the lab. Why on earth that I am still not getting any job. There
% must be something there I did not solve. 

\section{Methods}

\subsection{Biosynthesis And Protein Purification}

PTE DNA, pQE30-PTE, was transformed into AFIQ cells as described in our previous
work\cite{Yang2014a}. Cells were plated on agar plates containing
\SI{200}{\ug\per\mL} ampicillin, \SI{34}{\ug\per\mL} chloramphenicol. A single
colony was picked and grown in LB with \SI{200}{\ug\per\mL} ampicillin, and
\SI{34}{\ug\per\mL} chloramphenicol) at \SI{37}{\celsius}, 300 r.p.m for 16
hours \SI{37}{\celsius} incubation.  Afterwards, \SI{250}{\mL} of LB medium for
large-scale expression was innoculated 1:50 with the overnight culture.  After
optical density reached 1.0 at 600 nm, the expression media were supplemented
with \SI{1}{\milli\Molar} isopropyl-$\beta$-D-thiogalactopyranoside (IPTG) to
induce protein expression.  \SI{1}{\milli\Molar} of \ce{CoCl2} was added in
each post-induction medium.  After three hours incubation at \SI{37}{\celsius},
300 r.p.m., the cells were harvested by using 4000 r.p.m centrifugation at
\SI{4}{\celsius} for 15 minutes and then resuspended with \SI{20}{\milli\Molar}
Tris-HCl, \SI{500}{\milli\Molar} \ce{NaCl}, \SI{5}{\milli\Molar} imidazole,
10\% glycerol (pH 8.0) and \SI{1}{\micro\Molar} \ce{CoCl2}. Cell lysate was
immediately sonicated for 1.5 minutes at \SI{4}{\celsius} and then a
clarification spin was performed (20,000 g, \SI{4}{\celsius}, 30 minutes).
Clarified supernatants were loaded into a \SI{5}{\mL} His Trap column (G.E
Healthcare, Piscataway, NJ) using AKTA FPLC purifier (G.E.  Healthcare,
Piscataway, NJ).  Protein elution was generated using elution buffer B
(\SI{20}{\milli\Molar} Tris-HCl, \SI{500}{\milli\Molar} sodium chloride,
\SI{500}{\milli\Molar} imidazole (pH 8.0)).  The purified samples were then
transferred for buffer exchange using \SI{12}{\L} \SI{20}{\milli\Molar}
phosphate buffer (pH 8.0).  Dialyzed protein was subjected to crystallization
immediately.

\subsection{Lactose and Crystallization}

To prepare over-saturated lactose solution, \SI{5}{\mL} of a deionized lactose
solution (\SI{0.30}{\gram\per\mL}) was added to \SI{1}{\mL} of an approximately
\SI{0.2}{\mg\per\mL} solution of the purified PTE. PTE was prepared according
to the procedure shown above. The mixture was incubated at \SI{6}{\celsius} for
roughly 2.5 weeks until crystals of a suitable size were obtained. Powder
lactose monohydrate was used as the seed for crystallization.  The crystals
were harvested and washed with distilled water.

\subsection{Protein Conjugation}

FITC kit was purchased from Sigma. Fluorescein isothiocyanate (FITC) was used
for labeling of PTE. FITC dissolved in dimethylformamide (DMF)
(\SI{1}{\mg\per\mL}) was added to the enzyme in PBS (pH 8.0) to a final ratio
of 1:5 (PTE:FITC). The reaction mixture was incubated for \SI{1.5}{hour} at room
temperature, then dialyzed against \SI{2}{\liter} of phosphate buffer
(\SI{20}{\milli\Molar}, pH 8.0). 

\subsection{Enzyme Kinetics}

The crystal was reconstituted in \SI{200}{\micro\liter} sodium phosphate (pH
8.0). Reactions were monitored spectrophotometrically (Synergy H1, BioTek,
Winooski VT) at \SI{405}{\nm} for paraoxon (coefficient =
\SI{17000}{\per\Molar\per\cm}).  Reactions for paraoxon
(\SIrange{13}{104}{\micro\Molar}) was done in 0.4\% methanol.
K\textsubscript{M} and V\textsubscript{max} values were determined by a
Lineweaver-Burk plot.\cite{Baker2011b} The equation used is shown below
(Eq.~\ref{eqn:MM-chap2}): 
\begin{equation} 
    \frac{1}{v} =
    \frac{K\textsubscript{M}}{V\textsubscript{max}}\times\frac{1}{S} +
    \frac{1}{V\textsubscript{max}} 
    \label{eqn:MM-chap3}
\end{equation}
where S represents substrate concentration; K\textsubscript{M} represents the
substrate concentration at which the reaction rate is half of
V\textsubscript{max}. The data reported is the average of three trials and the
error represents the standard deviation of those trials.

\section{Results}

The results was amazing.

\section{Future work}

\printbibliography[heading=subbibliography]

\end{refsection}

%\appendix
%\chapter{Supplementary Data}
\begin{refsection}

\section{Transmission electron microscopy of non-fluorinated block copolymers}
% --------------------------
\begin{figure}[h!] \centering \includegraphics[width=0.95\textwidth]{f_c_08}
    \caption[]{Transmission electron micrographs of (d) wt-EC, (e) wt-CE, and
        (f) wt-ECE, prepared to \SI{0.45}{\mg\per\mL}, prepared and analyzed
        according to the methods in Section \ref{sec:TEM_method}. Samples were
        prepared and pre-incubated at \SI{4}{\celsius} prior to analysis. Images
        were collected ${110000 \times}$ and ${19500 \times}$ (inset)
        magnification. This figure is reproduced from the work by Haghpanah and
        Yuvienco for comparative purposes.\cite{Haghpanah2010}}
        \label{fig:block_EM_wt} \end{figure}
% --------------------------
In studies carried out previously by Haghpanah and Yuvienco, wild-type variants
of the block copolymers - EC, CE, and ECE - were characterized via transmission
electron microscopy. These studies indicated particle morphologies for all three
constructs, as well as a particular propensity for the ECE protein to flocculate
and form larger supramolecular features (shown in Figure \ref{fig:block_EM_wt}f
(inset)).

\section{Solid-state \NMR*{19,F} on \emph{p}FF-CE\textsubscript{3} protein}
\label{sec:ss_nmr}
% --------------------------
\begin{figure}[h!] \centering \includegraphics[width=0.8\textwidth]{f_c_05}
    \caption{\NMR*{19,F} spectra obtained at \SI{10}{\celsius} and
    \SI{35}{\celsius} for \emph{p}FF-CE\textsubscript{3} protein.}
    \label{fig:nmr_result_1} \end{figure}
% --------------------------
% --------------------------
\begin{figure}
    \centering
    \begin{subfigure}[b]{0.4\textwidth}
        \includegraphics[width=\textwidth]{f_c_06a}
    \end{subfigure}
    \begin{subfigure}[b]{0.4\textwidth}
        \includegraphics[width=\textwidth]{f_c_06b}
    \end{subfigure}
    \caption{\NMR*{19,F} peak intensities for the residual Teflon (green) and
    \emph{p}FF-CE\textsubscript{3} (red) on the \textsuperscript{1}H decoupling
    power parameter aHtppm at (a) \SI{10}{\celsius} and (b)
    \SI{35}{\celsius}.  Spectra suggest a dependence of the peak intensities for
    the residual Teflon (green) and \emph{p}FF-CE\textsubscript{3} (red) on the
    \textsuperscript{1}H decoupling power parameter \emph{aHtppm} at
    \SI{10}{\celsius} and \SI{35}{\celsius}.}\label{fig:nmr_result_2}
\end{figure}
% --------------------------
% --------------------------
\begin{figure}
    \centering
    \begin{subfigure}[b]{0.4\textwidth}
        \includegraphics[width=\textwidth]{f_c_07a}
    \end{subfigure}
    \begin{subfigure}[b]{0.4\textwidth}
        \includegraphics[width=\textwidth]{f_c_07b}
    \end{subfigure}
    \caption{\NMR*{19,F} peak intensities for the (a) residual Teflon and (b)
    \emph{p}FF-CE\textsubscript{3} as a function of the time between spin
    echoes, d1, at \SI{10}{\celsius} (green) and \SI{35}{\celsius} (red).}
    \label{fig:nmr_result_3}
\end{figure}
% --------------------------
All experiments were carried out using a Varian 500 MHz spectrometer operating
at \SI{470.6}{\mega\hertz} for \textsuperscript{19}F.\footnote{\NMR*{19,F}
experiments were carried out with the generous assistance and supervision of Dr.
Christopher Klug (Naval Research Laboratory, Chemistry Division).}
% --------------------------
The sample was packed into a \SI{3.2}{\mm} SiN rotor and spun at
\SI{20}{\kilo\hertz}.  The mass of the sample was \SI{10.5}{\mg}.  All data was
acquired using a rotor synchronized spin echo, with varying amounts of 1H
decoupling. The measurements were run at two temperatures, \SI{10}{\celsius} and
\SI{35}{\celsius}.

Typical spectra are shown in Figure \ref{fig:nmr_result_1}.  The narrow peak at
roughly \SI{-122}{\ppm} is assigned to residual Teflon present in the Torlon
components of the NMR rotor.  The relatively broad peak at ~\SI{115}{\ppm}
associated with \emph{p}FF-CE\textsubscript{3} reflects a distribution of local
environments for the \textsuperscript{19}F nuclei of these side groups.  (Note
that the other peaks in the spectra are spinning sidebands.) There is no
significant difference in the results obtained at the two temperatures.

As the decoupling power increases there is first a minimum peak intensity for
the \emph{p}FF-CE\textsubscript{3} sample at 1000 followed by a gradual
increase (see Figure \ref{fig:nmr_result_2}).  As expected, the Teflon peak was
relatively insensitive to \textsuperscript{1}H decoupling.  Higher decoupling
powers were not attempted to avoid excessive sample heating.

Figure \ref{fig:nmr_result_3} shows the dependence of the peak intensities in
the \NMR*{19,F} spectra on time between spin echoes, d1.  This gives a measure
of the spin-lattice relaxation time, T1.  The solid lines are fits to a simple
single-exponential recovery corresponding to a single T1.  While there is a
slight decrease in the calculated T1 for the Teflon from ~2.1 s to 1.8 s as the
temperature increased from \SI{10}{\celsius} to \SI{35}{\celsius}.  There is no
clear change in the calculated T1 for \emph{p}FF-CE\textsubscript{3}, which
remains at ~2.2-2.3 s.  This somewhat unexpected result may be due to a broad
distribution in relaxation behavior for the 19F in
\emph{p}FF-CE\textsubscript{3}.

\section{The cytotoxic effects of protein constructs}
\begin{figure}[h!] \centering \includegraphics[width=0.95\textwidth]{f_c_01}
    \caption{The effects of COMP and block copolymer protein constructs on the
        viability of MC3T3-E1 preosteoblasts, assessed via the MTT assay method,
        \emph{n}=4. wt-CE, \emph{p}FF-CE, and COMP were prepared to a final
        concentration of \SI{200}{\ug\per\mL} and \emph{p}FF was prepared to a
        final concentration of \SI{0.8}{\ug\per\mL}.}\label{fig:mtt_results}
    \end{figure}
Cytotoxicity of the protein constructs were evaluated by
3-(4,5-dimethylthiazol-2-yl)-2,5-diphenyl tetrasodium bromide, MTT assay.
MC3T3-E1 preosteoblast cells (
\SI[scientific-notation=true,retain-unity-mantissa=true]{1e6}{\per\ml}) in
\SI{100}{\uL} of ${\alpha}$MEM (Invitrogen) supplemented with \SI{10}{\volper}
FBS were seeded in 96-well plates and incubated overnight. The
\SI{5}{\mg\per\mL} MTT reagent in PBS was added into the plates and incubated
for 4 h. After incubation, the medium was aspirated and dimethyl sulfoxide
(\SI{100}{\uL} per well) was added to stop the reaction. The optical density was
then quantified in a microplate reader, Synergy HT at \SI{570}{\nm} wavelength.
The percentage of cell viability was calculated by comparing the groups to
control cells, which did not contain any of the sample reagents. The results are
shown in Figure \ref{fig:mtt_results}. the majority of the constructs
demonstrate a statistically insignificant effect on the viability of cells as
compared to the control.

\section{Transmission electron microscopy study of COMP coiled-coil protein}
\label{sec:comp_tem_results}
% --------------------------
\begin{figure}
    \centering
    \begin{subfigure}[b]{0.31\textwidth}
        \includegraphics[width=\textwidth]{f_c_02a}
        \caption{}
    \end{subfigure}
    \begin{subfigure}[b]{0.31\textwidth}
        \includegraphics[width=\textwidth]{f_c_02b}
        \caption{}
    \end{subfigure}
    \begin{subfigure}[b]{0.31\textwidth}
        \includegraphics[width=\textwidth]{f_c_02c}
        \caption{}
    \end{subfigure}
    \begin{subfigure}[b]{0.31\textwidth}
        \includegraphics[width=\textwidth]{f_c_02d}
        \caption{}
    \end{subfigure}
    \begin{subfigure}[b]{0.31\textwidth}
        \includegraphics[width=\textwidth]{f_c_02e}
        \caption{}
    \end{subfigure}
    \begin{subfigure}[b]{0.31\textwidth}
        \includegraphics[width=\textwidth]{f_c_02f}
        \caption{}
    \end{subfigure}
    \begin{subfigure}[b]{0.31\textwidth}
        \includegraphics[width=\textwidth]{f_c_02g}
        \caption{}
    \end{subfigure}
    \begin{subfigure}[b]{0.31\textwidth}
        \includegraphics[width=\textwidth]{f_c_02h}
        \caption{}
    \end{subfigure}
    \begin{subfigure}[b]{0.31\textwidth}
        \includegraphics[width=\textwidth]{f_c_02i}
        \caption{}
    \end{subfigure}
    \caption{Electron micrographs of COMP protein prepared to (a-c)
        \SI{900}{\micro\moLar}, (d-f) \SI{500}{\micro\moLar}, and
        (g-i) \SI{100}{\micro\moLar}. All micrographs provide evidence of the
    supramolecular fiber assembly of the protein at concentrations applicable to
    \latin{in vitro} experiments documented in Chapter \ref{chap:comp}. As
concentration of the preparations increases, the protein tends to form larger
fibers, isolated to fewer spots of the TEM sample grid.}\label{fig:COMP_EM_3}
\end{figure}
% --------------------------
Transmission electron micrographs were collected using a Philips CM12
transmission electron microscope equipped with a Gatan 4k ${\times}$ 2.7k
digital camera.\footnote{These experiments were conducted with the assistance of
Richard Hwang and Eric Roth at the New York University Skirball Institute.}Lyophilized protein samples were prepared from COMP protein
desalting extensively against MilliQ water. Dry protein was dissolved with
various volumes of \SI{10}{\milli\moLar} Gomori buffer, pH 8.0, to yield the
appropriate concentrations dictated by the experiment, shown in Figure
\ref{fig:COMP_EM_3}. The samples were negatively stained, consistent with
the adhesion drop method, previously documented in \ref{sec:TEM_method}. 

\section{Limulus amebocyte lysate (LAL) assay of COMP protein preparation}
\label{sec:lal_assay}
% --------------------------
\begin{figure}[h!] \centering \includegraphics[width=0.6\textwidth]{f_c_03}
    \caption{Quantitation of endotoxin levels in COMP protein preparations
        following endotoxin removal using a Detoxi-Gel column. Open square
        markers denote standard curve data points, derived from an
        \latin{E.coli} O111:B4 standard. Closed markers indicate COMP elution
        samples from the column separation. Samples were applied to the assay
        kit as a 1:120, 1:25, and 1:10 dilution samples for pre-endotoxin
        separation, \SI{300}{\milli\moLar}, and \SI{500}{\milli\moLar} elutions,
    respectively. Correlation coefficient of 0.92881 was calculated from a
linear regression fit along the standard data points.}\label{fig:LAL_assay}
\end{figure}
% --------------------------
LAL assays were performed using a Pierce LAL Chromogenic Endotoxin Quantitation
Kit (Pierce). The kit measures endotoxin levels by measuring the activity of the
proenzyme Factor C against lipopolysaccharides (endotoxins) derived from the
outer cell membrane of gram-negative bacteria. The active protease derived from
the presence of endotoxins results in the release of \iupac{\p-nitroaniline}
(pNA) after proteolysis, which produces a maximal absorption at \SI{405}{\nm}.
Version 2445.3 of the kit was used.
\section{Size-exclusion standard curves}
% --------------------------
\begin{figure}
    \centering
    \begin{subfigure}[b]{0.8\textwidth}
        \includegraphics[width=\textwidth]{f_c_04a}
        \caption{}
    \end{subfigure}
    \begin{subfigure}[b]{0.8\textwidth}
        \includegraphics[width=\textwidth]{f_c_04b}
        \caption{}
    \end{subfigure}
    \caption{Size-exclusion chromatograms collected for lysozyme and bovine
        serum albumin, each run as separate \SI{2}{\micro\liter} injections and
        prepared to \SI{3}{\mg\per\mL} in \SI{100}{\milli\moLar} Gomori
        phosphate buffer, pH 8.0, \SI{10}{\volper} methanol, consistent with
        mobile phase conditions. (a) Overlay of chromatograms of protein
        standards in the absence of BMS493 in the mobile phase buffer. (b)
        Overlay of chromatograms of protein standards in the presence of BMS493
        in the mobile phase buffer.}\label{fig:sec_standards}
\end{figure}
% --------------------------


Size exclusion standard curves (shown in Figure \ref{fig:sec_standards}) were collected on a Acquity H-Class UPLC
(Waters), using a \SI{4.6}{\mm} ${\times}$ \SI{300}{\mm} BEH125 SEC column. The
column was used with an isocratic mobile phase consisting of
\SI{100}{\milli\moLar} Gomori phosphate buffer, pH 8.0;
\SI{500}{\milli\moLar} \iupac{\L-arginine}${\cdot}$\ch{HCl};
\SI{10}{\volper} methanol, 
in either the absence or presence of BMS493 (supplemented with BMS493 to a final
concentration of \SI{50}{\micro\moLar}). Buffer flowed through the column at a
constant rate of \SI{250}{\uL\per\minute} with a typical back pressure of
\SI{8000}{\psi}. Chromatograms were collected over the
course of \SI{25}{\minute}.

\section{UPLC method development}
\label{sec:rp_method}
% --------------------------
\begin{figure} \centering \begin{subfigure}[b]{0.7\textwidth}
        \includegraphics[width=\textwidth]{f_2_13a} \caption{UPLC gradient for
        chromatographic separation} \label{fig:uplc_gradient} \end{subfigure}
    \begin{subfigure}[b]{0.7\textwidth}
        \includegraphics[width=\textwidth]{f_2_13b} \caption{Separation of BMS493, ATRA, and retinyl palmitate, collected at
            \SI{329}{\nm}, with noted retention times of \SI{2.176}{\minute},
            \SI{2.372}{\minute}, and \SI{7.151}{\minute}, respectively.}
        \label{fig:std_chromatogram} \end{subfigure}
    \begin{subfigure}[b]{0.7\textwidth}
        \includegraphics[width=\textwidth]{f_2_13c} \caption{Cross-sectional
            spectra
            at points of peak maxima, corresponding to ATRA, BMS493, and retinyl
        palmitate (top to bottom).}
        \label{fig:retinoid_spectra} \end{subfigure}
    \caption[Chromatographic separation of BMS493, ATRA, and retinyl
    palmitate]{Chromatographic separation of BMS493, ATRA, and retinyl palmitate
        using the UPLC gradient shown in (a). The resultant chromatogram (b)
        demonstrate successful separation of this control sample. From 3D
        spectra (data not shown), single retention time peak spectra were
    extracted for the identification of optimal wavelengths for
analytes.}\label{fig:uplc_report} \end{figure}
% --------------------------
% --------------------------
\begin{figure}
    \centering
    \begin{subfigure}[b]{0.75\textwidth}
        \includegraphics[width=\textwidth]{f_2_14a}
        \caption{MilliQ, \SI{0.3366}{\ng}, \SI{92.46}{\percent} recovery}
        \label{fig:milliq_extract}
    \end{subfigure}
    \begin{subfigure}[b]{0.8\textwidth}
        \includegraphics[width=\textwidth]{f_2_14b}
        \caption{DMEM, \SI{0.335}{\ng}, \SI{92.02}{\percent} recovery}
        \label{fig:lonza_extract}
    \end{subfigure} \caption{UPLC quantitation of BMS493 (\SI{3.1}{\minute}) in
    extract controls prepared in different media. Each sample was spiked with
    \SI{0.1}{\volper} \SI{1}{\milli\moLar} BMS493 prepared in
DMSO}\label{fig:extract_controls} \end{figure}
% --------------------------
UPLC method was optimized by first attempting separation of BMS493, retinoic
acid, and retinyl palmitate using a binary system of acetonitrile and water.
However, this resulted in poor resolution of retinoic acid. The addition of
methanol was then added, in agreement with previously reported
methods.\cite{DeLeenheer1982,Kane2008b,Wang2001a,Schaffer2010} However, BMS493
and retinoic were shown to co-elute easily with each other. To optimize the
resolution of BMS493 and endogenous retinoids, but also to maintain maximum
attainable peak heights (and signal/noise), a linear scouting gradient was
performed from 10:10:80 (ACN:MetOH:\ch{H2O}, \SI{0.1}{\percent} formic acid) to
45:45:10 (ACN:MetOH:\ch{H2O}, \SI{0.1}{\percent} formic acid) over
\SI{30}{\minute} (data not shown). From this optimization, maximal resolution
and minimum peak width were determined to be obtained using an isocratic mobile
phase consisting of 41.2:41.2:17.6 (ACN:MetOH:\ch{H2O}, \SI{0.1}{\percent}
formic acid). Elution of retinyl palmitate requires a relatively abrupt gradient
to \SI{100}{\percent} ACN, then held isocratically for approximately
\SI{3}{\minute} (Figure \ref{fig:uplc_gradient}). The cross-sectional spectra
presented in Figure \ref{fig:retinoid_spectra} allowed for single channels -
\SI{355}{\nm}, \SI{329}{\nm}, and \SI{324}{\nm} for ATRA, BMS493, and retinyl
palmitate, respectively - to be collected to obtain the best sensitivity of
detection as well as cross-channel calibration against retinyl palmitate. This
method was then applied to cell culture extracts, prepared using liquid-liquid
extraction techniques (see Figure \ref{fig:extract_controls}).

\printbibliography[heading=subbibliography]

\end{refsection}

%\singlespacing
%\chapter{Sequence Data}
\label{chap:seq_data}

\lstset{basicstyle=\footnotesize,breakatwhitespace=true,breaklines=true,title=\lstname}

\begin{figure}[h!] \centering \includegraphics[width=0.95\textwidth]{f_a_01}
    \caption{Plasmid sequence for ELP-COMP block copolymer}\label{fig:plasmid_pQE30-EC} \end{figure}
%EC Sequence Start
\texttt{\lstinputlisting{../sequences/pQE30-EC}}
%EC Sequence End

\begin{figure}[h!] \centering \includegraphics[width=0.95\textwidth]{f_a_02}
    \caption{Plasmid sequence for COMP-ELP block copolymer}\label{fig:plasmid_pQE30-CE} \end{figure}
%CE Sequence Start
\texttt{\lstinputlisting{../sequences/pQE30-CE}}
%CE Sequence End

\begin{figure}[h!] \centering \includegraphics[width=0.95\textwidth]{f_a_03}
    \caption{Plasmid sequence for ELP-COMP-ELP block copolymer}\label{fig:plasmid_pQE30-ECE} \end{figure}
%ECE Sequence Start
\texttt{\lstinputlisting{../sequences/pQE30-ECE}}
%ECE Sequence End

\begin{figure}[h!] \centering \includegraphics[width=0.95\textwidth]{f_a_04}
    \caption{Plasmid sequence for COMP block copolymer}\label{fig:plasmid_pQE9-COMP} \end{figure}
%C Sequence Start
\texttt{\lstinputlisting{../sequences/pQE9-COMP}}
%C Sequence End

%\chapter{Microrheology Subroutines}

\lstset{
    basicstyle=\footnotesize,
    breakatwhitespace=true,
    breaklines=true,
    title=\lstname}

\lstinputlisting[language=IDL]{../source_code/batch_process.pro}

\lstinputlisting[language=IDL]{../source_code/tiff_pretrack.pro}

\lstinputlisting[language=IDL]{../source_code/msd_vb.pro}

\lstinputlisting[language=IDL]{../source_code/micrheo.pro}

\end{document}

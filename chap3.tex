\chapter{Phosphotriesterase Stabilization Via $\alpha$-Lactose Monohydrate Formulation} 
\label{chap:lactose}

\begin{refsection}

\section{Introduction}

\subsection{Applications of Phosphotriesterase}

The catalytic properties of enzymes led to extensive development in several
industries as briefly discussed in \ref{chap:uaa}. However, the limitations of
applications were resulted from the inactivation of enzymes. The factors
included heat, proteolysis, organic solvents, etc. The limitations motivated
scientists to find solutions in order to take advantage of enzymes activity,
specificity and other attractive features.

Entrapment is one of the methods used in industrial scale enzyme
operations. As enzymes of interest are intracellular, it is easier to use whole cell
immobilization where the enzyme is synthesized in bacteria. \cite{Trelles2013}
(Figure \ref{fig:enzyme-entrapment})

% --------------------------
\begin{figure}[h!] \centering \includegraphics[width=0.7\textwidth]{fig3_04}
    \caption[Schematic drawing of the agar or agarose immobilization
    procedure]{Schematic drawing of the agar or agarose immobilization
        procedure.\cite{Trelles2013}}
    \label{fig:enzyme-entrapment} 
\end{figure}
% --------------------------

One of examples of this is glucose isomerase (GI), which has been used in the
commercial production of high fructose corn syrup (HFCS).\cite{Bhosale1996} Most of
the glucose isomerase used in the production of over 6 million tons of
HFCS per year is in the form of immobilized whole cells. [231] This is often
done by spray drying the harvested cells to give a granulated product that is
then treated with polymeric materials (such as polyethylenimine) to stabilize
them. To further improve reactivity, the cells may be permeabilized.  This
removes the barrier for the free diffusion of the substrate/product across the
cell membrane, and also empties the cell of most of the small molecular weight
cofactors, etc., thus minimizing unwanted side reactions. Obviously, this and
all other entrapment techniques are most applicable for low molecular weight
substrates and simple bioconversions like hydrolysis, isomerization and
oxidation reactions that do not require a cofactor-regeneration system. [232]

It is possible to use entrapment as a mean of immobilizing a cofactor requiring
system. This can be accomplished in several ways. Initial efforts involved
covalently attaching the cofactor directly onto a solid
support or in a gel matrix along with the enzymes.[233] However, while the
enzymes and cofactor would be stable, the level of activity was generally quite
low. Other researchers attached the cofactor directly to the enzyme by means of
a bifunctional linker such as modified polyethylene glycol (PEG). The linker
would ideally be short enough to keep the cofactor close to the active site.
However, it would also need to be long and flexible enough to permit its easy
entrance in and out of the active site.[234] The most commonly used method has
been to attach the cofactor to PEG or other large polymer. This is then placed
with the enzyme(s) in ultrafiltration systems with a semipermeable membrane,
microencapsulated, or immobilized in membranes. In most cases, the cofactor
would be regenerated with a second enzyme such as formate or alcohol
dehydrogenase, for which a second substrate is required. When the process is
being used for the synthesis of a valuable product, this can be cost effective.
However, for use in biodegradation such an approach would be too expensive. Two
enzymes described recently could simplify the system and reduce costs. The
first involves a hydrogenase that in the presence of hydrogen can regenerate
NADH.[235] This is still somewhat involved since hydrogen would need to be
provided. An even more interesting enzyme is an NADH Oxidase that can use
dissolved oxygen to regenerate NAD with the production of water.[236] Thus,
only aeration of the system would be required and no unwanted products would be
generated.

The primary natural polymers used for entrapment have been
agar, agarose and gelatin through thermoreversal polymerization, and alginate
and carrageenan by ionotropic gelation. In addition to possible enzyme leakage,
these are relatively soft materials that will deform in large packed columns.
They are also subject to deterioration if used in fluidized bed reactors. In
the case of alginate and carrageenan, the ionic species used in the
polymerization (usually Ca2+) needs to be present continuously to maintain the
integrity of the gels. With some enzymes and processes, this can be a problem.

In the past, members of this laboratory have tested lactose ability to
incorporate all kinds of molecules: large organic dyes, green fluorescent
protein, bovine serum albumin, horse-radish peroxidase, and now
phosphotriesterase (PTE). PTE is a dimeric protein composed of identical
subunits which come together in the protein's active state. PTE is capable of
hydrolyzing a wide range of organophosphates, and has been shown to work in the
breakdown and neutralization of pesticides and herbicides, as well as nerve
agents like sarin gas. This makes PTE an excellent target for kinetic
stabilization and storage by lactose - consider field medics in a warzone,
stocked with tablets of PTE stored in lactose crystals, or the application of
PTE/LM crystals to the processing of crops or other chemically treated
foodstuff.

\subsection{Lactose Monohydrate}

Lactose (\iupac{4-\O-\chembeta-\D-galactopyranosyl-\D-glucopyranose},
\ce{C12H22O11}) is a disaccharide consisting of a \iupac{\D-glucose} and a
\iupac{\D-galactose} joined by a \iupac{\chembeta-1,4-glycosidic} bond. (Figure
\ref{fig:lactose-structure})  It has been previously shown by members of this
laboratory that $\alpha$-lactose monohydrate (LM) is capable of incorporating
various macromolecules and biopolymers into its crystal
structure\cite{Wang2001a,Kurimoto1999}. (Figure \ref{fig:lactose-structure}) It
has been reasoned that this is made possible by LM's abundance of peripheral
hydrogen atoms, which are capable of hydrogen bonding to said macromolecules,
trapping them, orienting them, and ultimately allowing the forming crystal to
overgrow and absorb them1.

% --------------------------
\begin{figure}[h!] \centering \includegraphics[width=0.5\textwidth]{fig3_01}
    \caption[Molecular structures of $\alpha$- and $\beta$- lactose.]{Molecular
    structures of $\alpha$- and $\beta$- lactose.}
    \label{fig:lactose-structure}
\end{figure}
% --------------------------

In the dairy industry, crystallization is an important separation
process used in the refining of lactose from whey solutions. In the refining
operation, lactose crystals are separated from the whey solution through
nucleation, growth, and/or aggregation. The rate of crystallization is
determined by the combined effect of crystallizer design, processing
parameters, and impurities on the kinetics of the process. This review
summarizes studies on lactose crystallization, including the mechanism, theory
of crystallization, and the impact of various factors affecting the
crystallization kinetics. In addition, an overview of the industrial
crystallization operation highlights the problems faced by the lactose
manufacturer. The approaches that are beneficial to the lactose manufacturer
for process optimization or improvement are summarized in this review. Over the
years, much knowledge has been acquired through extensive research. However,
the industrial crystallization process is still far from optimized. Therefore,
future effort should focus on transferring the new knowledge and technology to
the dairy industry.

% --------------------------
\begin{figure}[h!] \centering \includegraphics[width=0.7\textwidth]{fig3_02} 
    \caption[Crystals of lactose monohydrate (LM) as hosts for the guest green
    fluorescent protein (GFP)]{Crystals of lactose monohydrate (LM) as hosts
        for the guest green fluorescent protein (GFP)\cite{Wang2001a}.}
    \label{fig:lm-intro}
\end{figure}
% --------------------------

\subsubsection{Protein conjugation}

Protein conjugation had been deployed for different purposes, including
stabilization and visualization. 

\section{Methods}

\subsection{Materials}

FITC kit and paraoxon were purchased from Sigma. All other chemicals,
including \ch{NaCl}, sodium phosphates monobasic, sodium phosphate dibasic,
were purchased from Sigma or VWR. 96-well plates were purchased from Thermo
Fisher Scientific (Waltham, MA).

\subsection{Biosynthesis And Protein Purification}
\label{sec:pte-chap3}

PTE DNA, pQE30-PTE, was transformed into AFIQ cells as described in our previous
work\cite{Yang2014a} and Chapter \ref{chap:uaa}. Cells were plated on agar
plates containing \SI{200}{\ug\per\mL} ampicillin, \SI{34}{\ug\per\mL}
chloramphenicol. A single colony was picked and grown in LB with
\SI{200}{\ug\per\mL} ampicillin, and \SI{34}{\ug\per\mL} chloramphenicol) at
\SI{37}{\celsius}, 300 r.p.m for 16 hours \SI{37}{\celsius} incubation.
Afterwards, \SI{250}{\mL} of LB medium for large-scale expression was
innoculated 1:50 with the overnight culture.  After optical density reached 1.0
at 600 nm, the expression media were supplemented with \SI{1}{\milli\Molar}
isopropyl-$\beta$-D-thiogalactopyranoside (IPTG) to induce protein expression.
\SI{1}{\milli\Molar} of \ce{CoCl2} was added in each post-induction medium.
After three hours incubation at \SI{37}{\celsius}, 300 r.p.m., the cells were
harvested by using 4000 r.p.m centrifugation at \SI{4}{\celsius} for 15 minutes
and then resuspended with \SI{20}{\milli\Molar} Tris-HCl,
\SI{500}{\milli\Molar} \ce{NaCl}, \SI{5}{\milli\Molar} imidazole, 10\% glycerol
(pH 8.0) and \SI{1}{\micro\Molar} \ce{CoCl2}. Cell lysate was immediately
sonicated for 1.5 minutes at \SI{4}{\celsius} and then a clarification spin was
performed (20,000 g, \SI{4}{\celsius}, 30 minutes).  Clarified supernatants
were loaded into a \SI{5}{\mL} His Trap column (G.E Healthcare, Piscataway, NJ)
using AKTA FPLC purifier (G.E.  Healthcare, Piscataway, NJ).  Protein elution
was generated using elution buffer B (\SI{20}{\milli\Molar} Tris-HCl,
\SI{500}{\milli\Molar} sodium chloride, \SI{500}{\milli\Molar} imidazole (pH
8.0)).  The purified samples were then transferred for buffer exchange using
\SI{12}{\L} \SI{20}{\milli\Molar} phosphate buffer (pH 8.0).  Dialyzed protein
was subjected to crystallization immediately.

\subsection{Lactose and Crystallization}

To prepare PTE.LM crystals, a total of \SI{0.20}{\milli\gram} purified PTE was
added to \SI{2}{\mL} \SI{0.28}{\gram\per\mL} lactose solution
(\SI{20}{\milli\Molar} phosphate buffer, pH 7.4).  PTE protein was prepared
according to the procedure in Section \ref{sec:pte-chap3}. The mixture was
incubated at \SI{6}{\celsius} for roughly 2.5 weeks until PTE.LM crystals of a
suitable size were obtained.  Powder lactose monohydrate was used as the seed
for crystallization. The crystals were then harvested and washed with distilled
water and dried under room temperature. PTE.LM crystals were then stored in
vials at \SI{4}{\celsius} or room temperature for the rest of the microscope or
paraoxon hydrolysis experiments.

\subsection{Protein Conjugation}

Fluorescein isothiocyanate (FITC) was used
for labeling of PTE. FITC dissolved in dimethylformamide (DMF)
(\SI{1}{\mg\per\mL}) was added to the enzyme in phosphate buffer (pH 8.0) to a
final ratio of 1:5 (PTE:FITC). The reaction mixture was incubated for
\SI{1.5}{hour} at room temperature, then dialyzed against \SI{2}{\liter} of
phosphate buffer (\SI{20}{\milli\Molar}, pH 8.0). 

\subsection{Enzyme Kinetics}

The crystal was reconstituted in \SI{20}{\micro\liter} sodium phosphate (pH
8.0). Reactions were monitored spectrophotometrically (Synergy H1, BioTek,
Winooski VT) at \SI{405}{\nm} for paraoxon (coefficient =
\SI{17000}{\per\Molar\per\cm}).  Reactions for paraoxon
(\SIrange{13}{104}{\micro\Molar}) was done in 0.4\% methanol.
K\textsubscript{M} and V\textsubscript{max} values were determined by a
Lineweaver-Burk plot.\cite{Baker2011b} The equation used is shown below
(Eq.~\ref{eqn:MM-chap3}): 
\begin{equation} 
    \frac{1}{v} =
    \frac{K\textsubscript{M}}{V\textsubscript{max}}\times\frac{1}{S} +
    \frac{1}{V\textsubscript{max}} 
    \label{eqn:MM-chap3}
\end{equation}
where S represents substrate concentration; K\textsubscript{M} represents the
substrate concentration at which the reaction rate is half of
V\textsubscript{max}. The data reported is the average of three trials and the
error represents the standard deviation of those trials.

\section{Results}

\subsection{Biosynthesis of PTE And Crystallization of PTE.LM}

PTE protein synthesis detail was described in Chapter \ref{chap:uaa}. After
dialysis, we determined the protein concentration \SI{0.139}{\mg\per\mL}. Using
over-saturated lactose solution, we then deposited purified PTE into the
lactose crystal. 

% --------------------------
\begin{figure}[h!] \centering \includegraphics[width=0.7\textwidth]{fig3_03} 
    \caption[Microscope image of PTE.LM.]{Microscope image of PTE.LM.} 
    \label{fig:ptelm-image} 
\end{figure}
% --------------------------

\subsection{Reconstituted Crystal Hydrolysis Reaction}

After roughly 3.5 months later, the reconstituted crystal showed paraoxon
hydrolysis reaction.

% --------------------------
\begin{figure}[h!] \centering \includegraphics[width=0.8\textwidth]{fig3_05} 
    \caption[The hydrolysis of reconstituted PTE.LM after stored at room
    temperature for 3.5 months.]{The hydrolysis of reconstituted PTE.LM after
    stored at room temperature for 3.5 months.} \label{fig:ptelm-hydrolysis} 
\end{figure}
% --------------------------

After storing at room temperature for ~3.5 months, \SI{0.08}{\gram\per\mL}
PTE.LM crystals was then reconstituted in \SI{20}{\milli\Molar} phosphate
buffer. \SIrange{50}{100}{\micro\Molar} paraoxon (0.4\% methanol) was then used
for assays. Within three minutes, reactions reached the maximum absorbance at
\SI{405}{\nm}. (Figure \ref{fig:ptelm-hydrolysis}) Focusing on the linear range
of the reaction, we then quantified the V\textsubscript{max} 0.0043 $\pm$
\SI{0.0012}{\micro\Molar\per\second} and K\textsubscript{M} 1186 $\pm$
\SI{336}{\micro\Molar} of PTE.LM. 

\subsection{FITC Conjugation And Crystallization}

\section{Future work}

Due to the limitation of detection methods that we previously used, we were not
able to determine protein concentration via spectrophotometer or BCA. Further
quantification is needed for k\textsubscript{cat}/K\textsubscript{M}.

\printbibliography[heading=subbibliography]

\end{refsection}

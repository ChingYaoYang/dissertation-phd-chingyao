\chapter{Formulation of Phosphotriesterase Using alpha-Lactose Monohydrate} 
\label{chap:lactose}

\begin{refsection}

\section{Introduction}

\subsection{Protein-based materials}

The lazy brown fox.

\subsection{Incorporation of non-natural amino acids}

\subsubsection{Site-specific incorporation methods}

The lazy brown fox.

\subsubsection{Global incorporation methods}

The lazy brown fox.

\subsection{Phosphotriesterase}

PTE is a homodimeric protein composed of two monomers, each of which contains a
metallo-active site. Phosphotriesterase (PTE) are enzymes, which hydrolyze
organophosphates (OPs) as well as synthetic esters (Figure
\ref{fig:pte-structure})\cite{Ghanem2005a}. The proenzyme form of PTE contains
29 amino acids signal peptide at the N-terminus. It is originally found as a
39kDa monomeric form in the solution\cite{Mulbry1989}. Later, the proenzyme of
PTE is engineered and expressed in the form of mature protein from \latin{E.
coli}. A ($\beta$/$\alpha$)\textsubscript{8} TIM-barrel structure forms the
monomeric PTE\cite{Roodveldt2005,Seibert2005}. The globular monomer is roughly
51\AA $\times$ 55\AA $\times$ 51\AA.  OPs are a synthetic class of small molecule
that irreversibly inactivate acetylcholinesterase (AChE), disrupting
neural transmission. AChE is an enzyme that degrades the neurotransmitter,
acetylcholine, at the neuromuscular junction in the cholinergic nervous system.
After the acetylcholine is hydrolyzed, the synaptic transmission would be
terminated. Inhibition of AChE lead to hyper-stimulation from toxic
accumulation of acetylcholine\cite{Soreq2001}. Army also adapted this protein
for chemical weapons neutralization \cite{Yang2014a}.

\subsection{Fluorous effect}

The lazy brown fox.

\subsection{Block protein polymers}

\subsubsection{Elastin-like peptides}
The lazy brown fox.

\subsubsection{Cartilage oligomeric matrix protein}

The lazy brown fox.

\section{Methods}

The lazy brown fox.

\section{Results}

The lazy brown fox.

\section{Discussion}

The lazy brown fox.

\section{Conclusion}

The lazy brown fox.

\printbibliography[heading=subbibliography]

\end{refsection}

\chapter{Phosphotriesterase Stabilization Via Lactose Monohydrate Formulation}
\label{chap:lactose}

\begin{refsection}

\section{Introduction}

\subsection{Enzymes Applications}

The catalytic properties of enzymes led to extensive development in several
industries as briefly discussed in the previous chapter. However, the
limitations of applications were resulted from the inactivation of enzymes. The
factors included heat \cite{Shirley1995}, proteolysis \cite{VandenBurg2002a},
organic solvents \cite{LeJeune1997a}, etc. The limitations motivated scientists
to find solutions in order to take advantage of enzymes activity, specificity
and other attractive features.

Entrapment is one of the methods used in industrial scale enzyme operations. As
enzymes of interest are intracellular, it is easier to use whole cell
immobilization where the enzyme is synthesized in bacteria. \cite{Trelles2013}
(Figure \ref{fig:enzyme-entrapment})
% --------------------------
\begin{figure}[h!] \centering \includegraphics[width=0.7\textwidth]{fig3_04}
    \caption[Schematic drawing of the agar or agarose immobilization
    procedure]{Schematic drawing of the agar or agarose immobilization
        procedure.\cite{Trelles2013}}
    \label{fig:enzyme-entrapment} 
\end{figure}
% --------------------------

One of examples of this is glucose isomerase (GI), which has been used in the
commercial production of high fructose corn syrup (HFCS).\cite{Bhosale1996} Most of
the glucose isomerase used in the production of over 6 million tons of
HFCS per year is in the form of immobilized whole cells.

%The primary natural polymers used for entrapment have been
%agar, agarose and gelatin through thermoreversal polymerization, and alginate
%and carrageenan by ionotropic gelation. In addition to possible enzyme leakage,
%these are relatively soft materials that will deform in large packed columns.
%They are also subject to deterioration if used in fluidized bed reactors. In
%the case of alginate and carrageenan, the ionic species used in the
%polymerization (usually Ca2+) needs to be present continuously to maintain the
%integrity of the gels. With some enzymes and processes, this can be a problem.

\subsection{PTE Application}

PTE is a dimeric protein composed of identical
subunits, and capable of hydrolyzing a wide range of organophosphates. It has
been shown to neutralize organophosphates, such pesticides and chemical
weapons. This makes PTE an excellent target for kinetic stabilization and
storage by lactose - consider field medics in a warzone, stocked with tablets
of PTE stored in lactose crystals, or the application to the processing of
crops or other chemically treated food products.

One of the PTE application was developed for bio-sensor of organophosphates. 

The other application was developed for the decontamination of
organophosphates
\cite{Cheng1996,LeJeune1997,Little1989,Chen1998,Gill2000,Havens1993,Masson2009a}.
NATO project 31 previously conducted research on several methods for PTE
stabilization \cite{Defrank}, such as foam \cite{LeJeune1997}, microemulsion
\cite{Cheng1996}, and polyurethane \cite{Defrank}. One advantage is that this
enzyme-based decontamination would provide a significant logistical reduction
during the processing (25 to 50 fold) \cite{Defrank}. One of the examples is
the foam formulation that was developed by LeJeune \latin{et al.}
\cite{LeJeune1997} (Figure \ref{fig:pte-foam}).
% -------------------------- pte-foam 
\begin{figure}[htbp]
    \centering \includegraphics[width=0.7\textwidth]{fig3_06}
    \caption[Photographs of synthesized polymers. Polymers were synthesized
    with Hypol 3000 surfactant. A 1:1 wt ratio of prepolymer to aqueous phase
    was used with 1 wt \% surfactant in the aqueous phase. PTE
    (\SI{0.6}{\mg\per\mL}) was added (\SI{10}{\mL}) to a buffered aqueous
    solution (\SI{0.12}{\Molar} Hepes buffer, pH 7.4,
    \SI{50}{\milli\Molar}\ch{CoCl2}) containing 1 \% Pluronic P-65 surfactant
    (\SI{5}{\mL}). Approximately \SI{4}{\mL} of Hypol 3000, a toluene
    diisocyanate based polyurethane prepolymer, were added to the PTE
solution(\SI{0.6}{\mg\per\mL}).]{Photographs of synthesized polymers. Polymers
    were synthesized with Hypol 3000 surfactant. A 1:1 wt ratio of prepolymer
    to aqueous phase was used with 1 wt \% surfactant in the aqueous phase. PTE
    (\SI{0.6}{\mg\per\mL}) was added (\SI{10}{\mL}) to a buffered aqueous
    solution (\SI{0.12}{\Molar} Hepes buffer, pH 7.4,
    \SI{50}{\milli\Molar}\ch{CoCl2}) containing 1\% Pluronic P-65 surfactant
    (\SI{5}{\mL}). Approximately \SI{4}{\mL} of Hypol 3000, a toluene
    diisocyanate based polyurethane prepolymer, were added to the PTE
    solution(\SI{0.6}{\mg\per\mL}) \cite{LeJeune1997}.} 
    \label{fig:pte-foam}
\end{figure}
% --------------------------*

The immobilization extended the storage to more than 25 days. They further
demonstrated that the foam also stabilizes PTE at \SI{50}{\celsius} for more
than 158 hours in comparison with soluble enzyme at \SI{50}{\celsius} for
merely 1.5 hours. 

\subsection{Lactose Monohydrate}

Lactose (\iupac{4-\O-\chembeta-\D-galactopyranosyl-\D-glucopyranose},
\ce{C12H22O11}) is a disaccharide consisting of a \iupac{\D-glucose} and a
\iupac{\D-galactose} joined by a \iupac{\chembeta-1,4-glycosidic} bond (Figure
\ref{fig:lactose-structure}). 
% --------------------------
\begin{figure}[h!] \centering \includegraphics[width=0.5\textwidth]{fig3_01}
    \caption[Molecular structures of $\alpha$- and $\beta$- lactose.]{Molecular
    structures of $\alpha$- and $\beta$- lactose.}
    \label{fig:lactose-structure}
\end{figure}
% --------------------------

In the dairy industry, crystallization is an important separation
process used in the refining of lactose from whey solutions. In the refining
operation, lactose crystals are separated from the whey solution through
nucleation, growth, and/or aggregation. The rate of crystallization is
determined by the combined effect of crystallizer design, processing
parameters, and impurities on the kinetics of the process. This review
summarizes studies on lactose crystallization, including the mechanism, theory
of crystallization, and the impact of various factors affecting the
crystallization kinetics. In addition, an overview of the industrial
crystallization operation highlights the problems faced by the lactose
manufacturer. The approaches that are beneficial to the lactose manufacturer
for process optimization or improvement are summarized in this review. Over the
years, much knowledge has been acquired through extensive research. However,
the industrial crystallization process is still far from optimized. 

It has been previously shown that $\alpha$-lactose monohydrate (LM) is capable
of incorporating various macromolecules and biopolymers into its crystal
structure\cite{Wang2001a,Kurimoto1999}. (Figure \ref{fig:lactose-structure}) It
has been reasoned that this is made possible by LM abundance of peripheral
hydrogen atoms, which are capable of hydrogen bonding to said macromolecules,
trapping them, orienting them, and ultimately allowing the forming crystal to
overgrow and absorb them.

% --------------------------
\begin{figure}[h!] \centering \includegraphics[width=0.7\textwidth]{fig3_02} 
    \caption[Crystals of lactose monohydrate (LM) as hosts for the guest green
    fluorescent protein (GFP)]{Crystals of lactose monohydrate (LM) as hosts
        for the guest green fluorescent protein (GFP)\cite{Wang2001a}.}
    \label{fig:lm-intro}
\end{figure}
% --------------------------

\subsubsection{Protein conjugation}

Protein conjugation had been deployed for different purposes, including
stabilization and visualization. Different fluorescein derivatives, such as
fluorescein isothiocyanate 1 (FITC), fluorescein amidite (FAM), or eosin, are
frequently used to label protein. 

\section{Methods}

\subsection{Materials}

FITC FluoroTag kit and paraoxon were purchased from Sigma (St. Louis, MO).
$\alpha$-lactose monohydrate was also purchased from Sigma (St. Louis, MO). All
other chemicals, including \ch{NaCl}, \ch{CoCl2}, Tris-HCl, tryptone, yeast
extract, paraoxon, ampicillin, chloramphenicol, sodium phosphates monobasic,
sodium phosphate dibasic, or imidazole were purchased from Sigma (St. Louis,
MO) or VWR (Radnor, PA). 96-well plates were purchased from Thermo Fisher
Scientific (Waltham, MA).

\subsection{Biosynthesis And Protein Purification}
\label{sec:pte-chap3}

PTE DNA, pQE30-PTE, was transformed into AFIQ cells as described in our previous
work\cite{Yang2014a} and Chapter \ref{chap:uaa}. Cells were plated on agar
plates containing \SI{200}{\ug\per\mL} ampicillin, \SI{34}{\ug\per\mL}
chloramphenicol. A single colony was picked and grown in LB with
\SI{200}{\ug\per\mL} ampicillin, and \SI{34}{\ug\per\mL} chloramphenicol) at
\SI{37}{\celsius}, 300 r.p.m for 16 hours \SI{37}{\celsius} incubation.
Afterwards, \SI{250}{\mL} of LB medium for large-scale expression was
innoculated 1:50 with the overnight culture.  After optical density reached 1.0
at 600 nm, the expression media were supplemented with \SI{1}{\milli\Molar}
isopropyl-$\beta$-D-thiogalactopyranoside (IPTG) to induce protein expression.
\SI{1}{\milli\Molar} of \ce{CoCl2} was added in each post-induction medium.
After three hours incubation at \SI{37}{\celsius}, 300 r.p.m., the cells were
harvested by using 4000 r.p.m centrifugation at \SI{4}{\celsius} for 15 minutes
and then resuspended with \SI{20}{\milli\Molar} Tris-HCl,
\SI{500}{\milli\Molar} \ce{NaCl}, \SI{5}{\milli\Molar} imidazole, 10\% glycerol
(pH 8.0) and \SI{1}{\micro\Molar} \ce{CoCl2}. Cell lysate was immediately
sonicated for 1.5 minutes at \SI{4}{\celsius} and then a clarification spin was
performed (20,000 g, \SI{4}{\celsius}, 30 minutes).  Clarified supernatants
were loaded into a \SI{5}{\mL} His Trap column (G.E Healthcare, Piscataway, NJ)
using AKTA FPLC purifier (G.E.  Healthcare, Piscataway, NJ).  Protein elution
was generated using elution buffer B (\SI{20}{\milli\Molar} Tris-HCl,
\SI{500}{\milli\Molar} sodium chloride, \SI{500}{\milli\Molar} imidazole (pH
8.0)).  The purified samples were then transferred for buffer exchange using
\SI{12}{\L} \SI{20}{\milli\Molar} phosphate buffer (pH 8.0).  Dialyzed protein
was subjected to crystallization immediately.

\subsection{Lactose and Crystallization}

To prepare PTE.LM crystals, a total of \SI{0.20}{\milli\gram} purified PTE was
added to \SI{2}{\mL} \SI{0.28}{\gram\per\mL} lactose solution
(\SI{20}{\milli\Molar} PBS buffer, pH 7.4).  PTE protein was prepared
according to the procedure in Section \ref{sec:pte-chap3}. The mixture was
incubated at \SI{6}{\celsius} for roughly 2.5 weeks until PTE.LM crystals of a
suitable size were obtained.  Powder lactose monohydrate was used as the seed
for crystallization. The crystals were then harvested and washed with distilled
water and dried under room temperature. PTE.LM crystals were then stored in
vials at \SI{4}{\celsius} or room temperature for the rest of the microscope or
paraoxon hydrolysis experiments.

\subsection{Protein Conjugation}

Fluorescein isothiocyanate (FITC) kit was used
for fluorescein conjugation with PTE. FITC dissolved in dimethylformamide (DMF)
(\SI{1}{\mg\per\mL}) was added to the enzyme in \SI{20}{\milli\Molar} phosphate
buffer (pH 8.0) to a final ratio of 1:5 (PTE:FITC). The reaction mixture was
incubated for \SI{1.5}{hour} at room temperature, then dialyzed against
\SI{2}{\liter} of phosphate buffer (\SI{20}{\milli\Molar}, pH 8.0) at
\SI{4}{\celsius} overnight.  The sample was then transferred to lactose
solution for crystallization as described above. The crystal (PTE.LM-FITC) was
then first observed with Zeiss Axioskop 40 Fluorescence Microscope (Peabody,
MA).  Wavelength \SI{495}{\nm} was used for excitation, and \SI{525}{\nm} was
used for emission. 

\subsection{Enzyme Kinetics}

The crystal (both PTE.LM and PTE.LM-FITC) was reconstituted in
\SI{20}{\micro\liter} sodium phosphate (pH 8.0). Reactions were monitored
spectrophotometrically (Synergy H1, BioTek, Winooski VT) at \SI{405}{\nm} for
paraoxon (coefficient = \SI{17000}{\per\Molar\per\cm}).  Reactions for paraoxon
(\SIrange{13}{104}{\micro\Molar}) was done in 0.4\% methanol.
K\textsubscript{M} and V\textsubscript{max} values were determined by a
Lineweaver-Burk plot.\cite{Baker2011b} The equation used is shown below
(Eq.~\ref{eqn:MM-chap3}): 
\begin{equation} 
    \frac{1}{v} =
    \frac{K\textsubscript{M}}{V\textsubscript{max}}\times\frac{1}{S} +
    \frac{1}{V\textsubscript{max}} 
    \label{eqn:MM-chap3}
\end{equation}
where S represents substrate concentration; K\textsubscript{M} represents the
substrate concentration at which the reaction rate is half of
V\textsubscript{max}. The data reported is the average of three trials and the
error represents the standard deviation of those trials.

\section{Results and Discussion}

\subsection{Biosynthesis of PTE And Crystallization of PTE.LM}

PTE protein synthesis detail was described in Chapter \ref{chap:uaa}. Using
FPLC, we were able to isolate PTE from \latin{E. coli} expression. After
dialysis in \SI{20}{\milli\Molar} phosphate buffer (pH 8.0), we determined the
protein concentration at \SI{0.139}{\mg\per\mL}by using Nano-Drop. The purity
of PTE was determined by SDS-PAGE gel (Chapter \ref{chap:uaa}). To prepare the
lactose solution of crystallization, we then supersaturated solutions of LM,
and incubated with PTE protein at \SI{40}{\celsius}. At the rate of
\SI{0.5}{\celsius\per\minute}, the mixture was then cooled down to room
temperature. The crystals measuring \SIrange{0.5}{3}{\mm} in length were
harvested after roughly 2.5 weeks incubation at \SI{4}{\celsius} (Figure
\ref{fig:ptelm-bottle}). 
% --------------------------ptelm-bottle
\begin{figure}[htbp] \centering \includegraphics[width=0.7\textwidth]{fig3_07}
    \caption[The crystal of PTE.LM. The left-hand side bottle contains
        \SI{79}{\mg} of PTE.LM after 2.5 weeks incubation at \SI{4}{\celsius}.
\SI{1.4}{\gram} and \SI{1}mg PTE were combined in \SI{20}{\milli\Molar}
phosphate buffer (pH 7.4).]{The crystal of PTE.LM. The left-hand side bottle
    contains \SI{79}{\mg} of PTE.LM after 2.5 weeks incubation at
    \SI{4}{\celsius} \SI{1.4}{\gram} and \SI{1}mg PTE were combined in
    \SI{20}{\milli\Molar} PBS (pH 7.4).} 
    \label{fig:ptelm-bottle} 
\end{figure}
% --------------------------*

Using over-saturated lactose solution,  we were able to deposit purified PTE
into the lactose crystal. Using microscope, we looked at the morphology of
PTE.LM crystal (Figure \ref{fig:ptelm-image}) While PTE.LM crystals exhibited
amorphous shape, we estimated the size at roughly \SI{3.0}{\mm} (h) $\times$
\SI{1.0}{\mm} (w) $\times$ \SI{0.5}{\mm} (d).
% --------------------------
\begin{figure}[h!] \centering \includegraphics[width=0.7\textwidth]{fig3_03} 
    \caption[Microscope image of PTE.LM. The estimated dimension of amorphous
    PTE.LM crystal are estimated as  \SI{3.0}{\mm} (h) $\times$ \SI{1.0}{\mm}
(w) $\times$ \SI{0.5}{\mm} (d).]{Microscope image of PTE.LM. The estimated
    dimension of amorphous PTE.LM crystal are estimated as  \SI{3.0}{\mm} (h)
    $\times$ \SI{1.0}{\mm} (w) $\times$ \SI{0.5}{\mm} (d).}
    \label{fig:ptelm-image} 
\end{figure}
% --------------------------

\subsection{Reconstituted PTE.LM Hydrolysis Reaction}

Previously, we already demonstrated that PTE
was able to hydrolyze organophosphates substrate \cite{Yang2014a,Baker2011b}.
However, the soluble PTE exhibited short shelf life within one week (Chapter
\ref{chap:uaa}). Further evidences demonstrated that PTE required stabilization
through embodiments \cite{Chen1998,Gill2000,Havens1993,Masson2009a}. To further
stabilize PTE, we deposit this enzyme into LM and determine the half life of
PTE.LM through paraoxon hydrolysis. Different batches of crystal were
synthesized for measurements of enzyme hydrolysis efficiency. After the
crystallization, PTE.LM was stored at room temperature or \SI{4}{\celsius}. The
preliminary result indicated that one month old PTE.LM stored at
\SI{4}{\celsius} was still active to paraoxon after reconstituted in
\SI{20}{\milli\Molar} pH 8 phosphate buffer (Figure \ref{fig:ptelm-one-month}). 
% --------------------------
\begin{figure}[htbp] \centering \includegraphics[width=0.7\textwidth]{fig3_08}
    \caption[Reconstituted one month old PTE.LM hydrolysis of paraoxon.
        \SI{10}{\mg} of PTE.LM was dissolved in \SI{200}{\micro\liter} of
        \SI{20}{\milli\Molar} phosphate buffer. Paraoxon was prepared at
        \SI{200}{\micro\liter} in \SI{20}{\milli\Molar} phosphate buffer. The
    hydrolysis reaction was measured at \SI{405}{\nm}.] {Reconstituted one month
        old PTE.LM hydrolysis of paraoxon.  \SI{10}{\mg} of PTE.LM was
        dissolved in \SI{200}{\micro\liter} of \SI{20}{\milli\Molar} phosphate
        buffer.  Paraoxon was prepared at \SI{200}{\micro\liter} in
        \SI{20}{\milli\Molar} phosphate buffer. The hydrolysis reaction was
        measured at \SI{405}{\nm}.} 
        \label{fig:ptelm-one-month} 
\end{figure}
% --------------------------

While the hydrolysis of paraoxon required catalyst in the solution, we then
confirmed the reaction in comparison of control experiment (saturated lactose
solution without PTE).The absorbance at \SI{405}{\nm} clearly demonstrated the
increased amounts of \emph{p}-nitrophenol after hydrolysis. The data from
crystals after 2 months storage at \SI{4}{\celsius} also demonstrated
hydrolysis of paraoxon when the protein stored in buffer is inactive. (Figure
\ref{fig:ptelm-two-month})
% --------------------------
\begin{figure}[htbp] \centering \includegraphics[width=0.5\textwidth]{fig3_09}
    \caption[Reconstituted two month old PTE.LM hydrolysis of paraoxon.
    \SI{10}{\mg} of PTE.LM was dissolved in \SI{200}{\micro\liter} of
\SI{20}{\milli\Molar} phosphate buffer. Paraoxon was prepared at
\SI{200}{\micro\liter} in \SI{20}{\milli\Molar} phosphate buffer. The
hydrolysis reaction was measured at \SI{405}{\nm}. Reconstituted PTE represents
the air-dried PTE that was stored at room temperature and reconstituted in
\SI{20}{\milli\Molar} phosphate buffer.] {Reconstituted two month old PTE.LM
    hydrolysis of paraoxon. \SI{10}{\mg} of PTE.LM was dissolved in
    \SI{200}{\micro\liter} of \SI{20}{\milli\Molar} phosphate buffer. Paraoxon
    was prepared at \SI{200}{\micro\liter} in \SI{20}{\milli\Molar} phosphate
    buffer. The hydrolysis reaction was measured at \SI{405}{\nm}.
    Reconstituted PTE represents the air-dried PTE that was stored at room
    temperature and reconstituted in \SI{20}{\milli\Molar} phosphate buffer.}
    \label{fig:ptelm-two-month} 
\end{figure}
% --------------------------

To further extend the shelf life of PTE.LM, we scaled up the process and
prolong the storage timeline to more than three months. In the meanwhile, we
stored the PTE.LM at room temperature to study the thermostability of PTE.LM.
After storing at room temperature for ~3.5 months, \SI{0.08}{\gram\per\mL}
PTE.LM crystals was then reconstituted in \SI{20}{\milli\Molar} phosphate
buffer. \SIrange{50}{100}{\micro\Molar} paraoxon (0.4\% methanol) was then used
for assays. Within three minutes, reactions reached the maximum absorbance at
\SI{405}{\nm}. (Figure \ref{fig:ptelm-hydrolysis}) Focusing on the linear range
of the reaction, we then quantified the V\textsubscript{max} 0.0043 $\pm$
\SI{0.0012}{\micro\Molar\per\second} and K\textsubscript{M} 1186 $\pm$
\SI{336}{\micro\Molar} of PTE.LM. 
% --------------------------
\begin{figure}[htbp] \centering \includegraphics[width=0.8\textwidth]{fig3_05} 
    \caption[The hydrolysis of reconstituted PTE.LM after stored at room
    temperature for 3.5 months.]{The hydrolysis of reconstituted PTE.LM after
    stored at room temperature for 3.5 months.} \label{fig:ptelm-hydrolysis} 
\end{figure}
% --------------------------

While we successfully extended the shelf life of PTE through the formulation,
we encounter the challenge of measurement of PTE inside the PTE.LM crystal.
After we incubated \SI{1}{\mg} of PTE in the lactose solution, we estimated
roughly \SI{0.014}{\mg} of protein deposited in the crystal. Using the
spectrophotometer, we were unable to determine the accurate amounts of PTE
after reconstituted in solution due to extremely low concentration of protein.
To further quantify the protein concentration, we would like to conduct and
evaluate the protein conjugation experiment. 

\subsection{FITC Conjugation And Crystallization}

To estimate the amounts of non-GFP protein in the crystal, Wang \latin{et al.}
conjugated fluorescein into protein, such as RNAse A, RNAse B, Avidin, and
NeutrAvidin \cite{Wang2001a}. With help from fluorescein, one could visualize
the location of protein at the (010) growth sector. To adapt the similar
strategy for PTE.LM crystal, we performed protein conjugation with Fluorescein
isothiocyanate (FTIC). The kit was originally developed for antibody
conjugation. The FITC probe has an absorption at \SI{495}{\nm} and emission at
\SI{525}{\nm}, which forms stable conjugation with free amino groups of
proteins. After the purification of PTE, we then combined this probe with
protein, and performed dialysis at \SI{4}{\celsius} overnight. After dialysis,
we deposited the conjugation product into LM with the same crystallization
procedure. After two weeks, we collected the crystal (PTE.LM-FITC) and use
fluorescence microscope to locate the probe conjugated with PTE. (Figure
\ref{fig:ptelm-fitc})
% --------------------------
\begin{figure}[htbp] \centering \includegraphics[width=0.8\textwidth]{fig3_10} 
    \caption[The image of PTE.LM-FITC crystal. An absorption at \SI{495}{\nm}
    and an emission at \SI{525}{\nm} were used for FITC probe. The crystal was
collected after two weeks crystallization.]{The image of PTE.LM-FITC crystal.
    An absorption at \SI{495}{\nm} and an emission at \SI{525}{\nm} were used
for FITC probe. The crystal was collected after two weeks crystallization.}
\label{fig:ptelm-fitc} 
\end{figure}
% --------------------------

From Figure \ref{fig:ptelm-fitc}, the crystal demonstrated the distribution of
PTE-FITC. Unfortunately, we did not find the (010) pattern inside the crystal.
This might be resulted from several factors, such as temperatures and pH during
the crystallization \cite{Wong2014}. As crystal visualization served as one of
goal of formulation, we also investigated the PTE stability in this study.
After collecting the PTE.LM-FITC, we reconstituted the crystal with
\SI{20}{\milli\Molar} pH 8 phosphate buffer. Unlike the hydrolysis from PTE.LM,
we were not able to measure the catalytic reaction. The loss of enzyme activity
is mainly caused from the FITC conjugation, which was previously reported by
Rogers \latin{et al.} \cite{Rogers1999}. To visualize the functional PTE inside
the crystal, especially at extremely low concentration, alternative methodology
is required in this study.

\section{Future work}

Due to the limitation of detection methods that we previously used, we were not
able to determine protein concentration via spectrophotometer or BCA. Further
quantification is needed for k\textsubscript{cat}/K\textsubscript{M}.

\printbibliography[heading=subbibliography]

\end{refsection}

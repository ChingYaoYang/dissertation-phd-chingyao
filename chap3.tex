\chapter{Phosphotriesterase Stabilization Via Lactose Monohydrate Formulation}
\label{chap:lactose}

\begin{refsection}

\section{Introduction}

\subsection{Enzyme Applications}

The catalytic properties of enzymes has led to extensive development in industry
as briefly discussed in the previous chapter \ref{tab:protein-app} (Table
\ref{tab:protein-app}). For example, protease has been developed for the use in
detergent \cite{Kirk2002}. After several iterations of development, protease
were engineered to meet requirements in detergents for laundry.  However,
inactivation of enzymes result in limitations of use. The inactivation factors
include: heat \cite{Shirley1995,Perdana2012,Etzel1996,Gouda2003}, proteolysis
\cite{VandenBurg2002a,Ahmad2012}, organic solvents
\cite{LeJeune1997a,Izutsu2009,Stepankova2013}. Several approaches have been
developed to address these limitations. The followings are discussion of
these limitations and solutions, including salts, mutagenesis, and
immobilization.

\subsubsection{Heat Inactivation}

Enzymes are sensitive to temperatures, especially under elevated high
temperatures \cite{Perdana2012,Etzel1996}. For example, glucose oxidase
\cite{Gouda2003} is susceptible to heat inactivation. Gouda \latin{et al.}
demonstrated that glucose oxidase was irreversibly inactivated over the
temperature above \SI{60}{\celsius} \cite{Gouda2003} (Figure
\ref{fig:heat-inactivation-exmaple}). 
% --------------------------heat-inactivation-exmaple
\begin{figure}[h!] \centering \includegraphics[width=0.7\textwidth]{fig3_12}
    \caption[The heat inactivation of glucose oxidase (in the absence of
    additives). Diamond, \SI{56}{\celsius};  square, \SI{60}{\celsius};
triangle, \SI{63}{\celsius}; circle, \SI{67}{\celsius}. Samples were incubated
at the desinated temperatures.]{The heat inactivation of glucose oxidase (in
    the absence of additives). Diamond, \SI{56}{\celsius};  square,
    \SI{60}{\celsius}; triangle, \SI{63}{\celsius}; circle, \SI{67}{\celsius}.
    Samples were incubated at the desinated temperatures \cite{Gouda2003}.}
    \label{fig:heat-inactivation-exmaple}
\end{figure}
% -------------------------- 

To improve the thermal stability of glucose oxidase, additives such as salts,
mono- and polyhydric alcohols, as well as polyelectrolytes were added
\cite{Appleton1997,Gouda2003}. To understand the stabilization by salts
such as \ce{NaCl} and \ce{K2SO4}, Gouda \latin{et al.} determined thermal
unfolding transitions of glucose oxidase in the presence of \SI{0.6}{\Molar}
\ce{NaCl} and \SI{0.2}{\Molar} \ce{K2SO4} by circular dichroism (CD)
measurements \cite{Gouda2003} (Figure \ref{fig:heat-inactivation-improve}). 
% --------------------------heat-inactivation-improvement
\begin{figure}[htbp] \centering \includegraphics[width=0.7\textwidth]{fig3_13} 
    \caption[Effect of salts on the thermal unfolding of  glucose oxidase. (A)
    ellipticity at 274 nm. (B) ellipticity at 222 nm. Solid line, native enzyme
in \SI{20}{\milli\Molar} phosphate (pH 6.0); dotted line, enzyme in the
presence of \SI{0.2}{\Molar} \ce{K2SO4} ; dashed line, enzyme in the presence
of \SI{0.6}{\Molar} \ce{NaCl}. Thermal inactivation of enzyme was followed in
the temperature range of \SIrange{25}{90}{\celsius}.]{Effect of salts on the thermal
    unfolding of  glucose oxidase. (A) ellipticity at 274 nm. (B) ellipticity
    at 222 nm. Solid line, native enzyme in \SI{20}{\milli\Molar} phosphate (pH
    6.0); dotted line, enzyme in the presence of \SI{0.2}{\Molar} \ce{K2SO4} ;
    dashed line, enzyme in the presence of \SI{0.6}{\Molar} \ce{NaCl}. Thermal
    inactivation of enzyme was followed in the temperature range of
    \SIrange{25}{90}{\celsius} \cite{Gouda2003}.}
    \label{fig:heat-inactivation-improve} 
\end{figure}
% --------------------------

The melting temperature (T\textsubscript{m}) measured at 274 nm were shifted
from \SI{62}{\celsius} for native enzyme to \SI{68}{\celsius} in the presence
of \SI{0.2}{\Molar} \ce{K2SO4} and to \SI{72}{\celsius} in the presence of
\SI{0.6}{\Molar} \ce{NaCl}, leading to stabilization. Using such stabilizing
effects of salts, they successfully demonstrated how the heat limitations could
be overcame. 

$\beta$-galactosidase (\iupac{\chembeta-\D-galactohydrolase}; EC 3.2.1.23)
has been demonstrated its enzymatic half-life at \SI{60}{\celsius} 0.5 hour
\cite{Melchers1970,Chen2008}. To generate hyper-thermostabilized
$\beta$-galactosidase, Xiong \latin{et al.} have developed a variant YG6762 via
directed evolution. With mutations, T29A, V213I, L217M, N277H, I387V, E414D,
R491C, and N496D, YG6762 demonstrates greater activity on its
substrate, \iupac{pNP-\chembeta-\D-glucuronide (pNPG)}, than the wild-type
enzyme (YH4502) \cite{Xiong2007}. Notably, YG6762 variant exhibits nearly 50\%
higher activity at \SI{37}{\celsius} Figure \ref{fig:yg6762}.
% --------------------------
\begin{figure}[htbp] \centering \includegraphics[width=0.7\textwidth]{fig3_20}
    \caption[The relative activity of purified mutant YG6762 and wild-type
    YH4502 enzymes at various temperatures. $\beta$-Glucuronidase activity was
measured using \iupac{pNP-\chembeta-\D-glucuronide} (pNPG) as a substrate.] {The
    relative activity of purified mutant YG6762 and wild-type YH4502 enzymes at
    various temperatures. $\beta$-Glucuronidase activity was measured using
    \iupac{pNP-\chembeta-\D-glucuronide} (pNPG) as a substrate \cite{Xiong2007}.}
    \label{fig:yg6762}
\end{figure}
% --------------------------

\subsubsection{Proteolysis}

Proteolytic resistance of a protein limits its application
\cite{Ottesen1967,Daniel1982,Fontana2004}. Investigations into relationship
between protein stability and proteolytic resistance reveales that the
resistance is dependent on stability of the protein \cite{Daniel1982,
Parsell1989}. For example, \latin{B. subtilis} lipase has been developed for the
synthesis of biopolymers and biodiesel, and the production of enantiopure
pharmaceuticals \cite{Jaeger2002,Ahmad2012}. Ahmad \latin{et al.} determined
the protease resistance of lipase around 55 minutes. Upon the inactivation from
protease, residual activity of wild-type lipase exhibited 55.2\% after four
hours incubation at \SI{37}{\celsius} \cite{Ahmad2012}. (Figure
\ref{fig:protease-resistance})
% --------------------------
\begin{figure}[htbp] \centering \includegraphics[width=0.5\textwidth]{fig3_14}
    \caption[Proteolytic resistance of wild-type lipase and its variants
    measured as post proteolysis residual activity, after incubation with
Subtilisin A. Mutations listed in the figure were generated by site-saturation
mutagenesis.]{Proteolytic resistance of wild-type lipase and its variants
    measured as post proteolysis residual activity, after incubation with
    Subtilisin A. Mutations listed in the figure were generated by
    site-saturation mutagenesis \cite{Ahmad2012}.}
    \label{fig:protease-resistance}
\end{figure}
% --------------------------

To increase both stability and protease resistance of lipase, Ahmad \latin{et al.}
generated a library of lipase mutants through site-saturated mutagenesis
(Figure \ref{fig:protease-resistance}).  Focusing on the loop and surface
residues of this enzyme, they then compared unfolding profiles and residual
activities of wild-type and mutants (Figure \ref{fig:protease-resistance}). By
introducing the mutation M137P, the lipase T\textsubscript{m} was raised by
\SI{6.8}{\celsius}, and the protease resistance was increased to ~955 minutes.
The residual activity was also elevated to 92.5\% \cite{Ahmad2012} (Figure
\ref{fig:protease-resistance}).

\subsubsection{Organic Solvent Inactivation}

One of the limitations for the use of enzymes is instability under processing
conditions, such as organic solvent systems \cite{Stepankova2013}.
Although water serves as a ideal solvent, it is a rather poor solvent for
synthetic reactions \cite{Serdakowski2008}. However, enzymes become inactive at
an organic co-solvent concentration of 60\% to 70\% \cite{Stepankova2013}. For
example, Stepankova \latin{et al.} have investigated the co-solvent effects on
haloalkane dehalogenases, DbjA, DhaA, and LinB \cite{Koudelakova2013}. These
enzyme catalyze the hydrolytic cleavage of the carbon-halogen bonds of
halogenated aliphatic compounds, and have been engineered for degradation
industry \cite{Stepankova2013a,Koudelakova2013} (Figure \ref{fig:hld}).
% --------------------------
\begin{figure}[htbp] \centering \includegraphics[width=0.5\textwidth]{fig3_16} 
    \caption[Simplified scheme of the reaction mechanism of haloalkane
    dehalogenase. Hydrolytic cleavage of a carbon-halogen bond proceeds by the
SN2, followed by the addition of water. Water is the only co-factor required
for catalysis]{Simplified scheme of the reaction mechanism of haloalkane
    dehalogenase. Hydrolytic cleavage of a carbon-halogen bond proceeds by the
    SN2, followed by the addition of water. Water is the only co-factor
    required for catalysis \cite{Koudelakova2013}.} 
    \label{fig:hld}
\end{figure}
% --------------------------

Stepankova \latin{et al.} demonstrated that addition of high concentrations of
organic solvents such as dimethylformamide, dimethyl sulfoxide and
tetrahydrofuran, resulted in variations in the CD spectra, indicating a loss of
$\alpha$-helical structure \cite{Stepankova2013a} (Figure
\ref{fig:organic-effect}). A decrease in activity of Dha A was also observed
with increasing co-solvent concentration. LinB was inactivated by most
co-solvents at low concentrations (less than 10\%).
% --------------------------
\begin{figure}[htbp] \centering \includegraphics[width=0.5\textwidth]{fig3_15} 
    \caption[Circular dichroism spectra of (A) DbjA and (B) DhaA in the
    presence of organic co-solvents. The spectra were measured at
\SI{37}{\celsius} in phosphate buffer (\SI{50}{\milli\Molar}, pH 7.5) and
various organic co-solvents at concentrations that caused reductions in enzyme
activity of more than 90\%. DMF, dimethylformamide; DMSO, dimethyl sulfoxide;
THF, tetrahydrofuran.]{Circular dichroism spectra of (A) DbjA and (B) DhaA in
    the presence of organic co-solvents. The spectra were measured at
    \SI{37}{\celsius} in phosphate buffer (\SI{50}{\milli\Molar}, pH 7.5) and
    various organic co-solvents at concentrations that caused reductions in
    enzyme activity of more than 90\%. DMF, dimethylformamide; DMSO, dimethyl
    sulfoxide; THF, tetrahydrofuran \cite{Stepankova2013a}.}
    \label{fig:organic-effect} 
\end{figure}
% --------------------------

To address this issue, Dravis \latin{et al.} have developed strategies to
stabilize enzymes in organic solvents \cite{Dravis2001}. The use of immobilized
enzymes represents the most common method to improve enzyme stability toward
organic solvents \cite{Koudelakova2013,Dravis2001}. The enzyme is coupled to an
inorganic polyethyleneimine impregnated alumina (PEI-alumina) with a
glutaraldehyde linker (Figure \ref{fig:pei}). The results demonstrate that the
soluble dehalogenase was completely inactivated (half life was less than 10
minutes) in presence of \SI{12}{\milli\Molar} 1,2,3-Trichloropropane (TCP). On
the contrary, The immobilized enzyme exhibits activity in the pure
1,2,3-trichloropropane (TCP) solvent, demonstrating a half-life of nearly 10
hours \cite{Dravis2001}.
% --------------------------
\begin{figure}[htbp] \centering \includegraphics[width=0.7\textwidth]{fig3_21}
    \caption[Stabilization of polyethyleneimine(PEI)/enzyme composites via
    glutaraldehyde crosslinking to prevent subunit dissociation.]{Stabilization
    of polyethyleneimine (PEI)/enzyme composites via glutaraldehyde
    crosslinking to prevent subunit dissociation \cite{Barbosa2014}.}
    \label{fig:pei}
\end{figure}
% --------------------------

Subtilisin E is an another example of inactivation due to organic solvent
\cite{You1996}. In the presence of 20\% of dimethylformamide (DMF), the
wild-type enzyme exhibits only less than 10\% of activity of that in the
absence of DMF. To enhance to catalytic activity of Subtilisin E, the Arnold
group have developed error-prone PCR strategy to generate variants
\cite{You1996}. After three generations of random mutagenesis, You \latin{et
al.} have demonstrated that variant 13M exhibited 10-fold greater activity that
wild-type in the absence of DMF. Notably, 13M was further improved by 16-fold
in the presence of 20\% DMF \cite{You1996} (Figure \ref{fig:arnold}). 
% --------------------------
\begin{figure}[htbp] \centering \includegraphics[width=0.7\textwidth]{fig3_22}
    \caption[Comparison of the catalytic efficiencies of wild type and 13M
    subtilisin E towards hydrolysis of succinyl-Ala-Ala-Pro-Phe-p-nitroanilide
(s-AAPF-pNa) in \SI{0.1}{\Molar} Tris-HCl, pH 8.0,
\SI{10}{\milli\Molar}{\ce{CaCl2}} with dimethylformamide (DMF), at
\SI{37}{\celsius}.]{Comparison of the catalytic efficiencies of wild type and
    13M subtilisin E towards hydrolysis of
    succinyl-Ala-Ala-Pro-Phe-p-nitroanilide (s-AAPF-pNa) in \SI{0.1}{\Molar}
    Tris-HCl, pH 8.0, \SI{10}{\milli\Molar}{\ce{CaCl2}} with dimethylformamide
    (DMF), at \SI{37}{\celsius}  \cite{You1996}.} \label{fig:arnold}
\end{figure}
% --------------------------

\subsection{Stabilization of Proteins in Solid State}

The stability of enzymes presents challenges in developments. Factors include
elevated temperatures \cite{Rupley1991} or organic solvents
\cite{Stepankova2013} may result in protein denaturation or aggregation. While
many enzymes do not anticipate long-term stability in solution, a solid-state
strategy, such as freeze-drying (lyophilized) \cite{Carpenter1993} or
crystallization, has been developed \cite{Taylor2010}. However, solid-sate
stabilization methods generates different stresses to proteins, which degrades
proteins during processes \cite{Taylor2010}. For example, Carpenter \latin{et
al.} have demonstrated that both freeze-drying and freeze-thawing processes
greatly impacted activity recovery of phosphofructokinase (PFK)
\cite{Carpenter1993}. Without any additive, PFK recovered less than 10 \% of
activity \cite{Carpenter1993} (Figure \ref{fig:pfk}).  
% --------------------------
\begin{figure}[htbp] \centering \includegraphics[width=0.5\textwidth]{fig3_23}
    \caption[Effect of polyethylene glycerol on phosphofructokinase stability
    during freeze-thawing (circle) and freeze-drying (triangle).] {Effect of
        polyethylene glycerol on phosphofructokinase stability during
        freeze-thawing (circle) and freeze-drying (triangle)
        \cite{Carpenter1993}.} 
    \label{fig:pfk}
\end{figure}
% --------------------------

To resolve the stabilization dilemma from solid state, proteins are formulated
with additives to improve their stability. Figure \ref{fig:pfk} illustrates
that polyethylene glycerol (PEG) results in significant recovery of activity.
The concentration of 1\% PEG completely protected the catalytic activity of PFK
during the freeze-thawing process. Moreover, Carpenter \latin{et al.} have
demonstrated that the combination of sugar and PEG further improved the
stability of freeze-drying PFK (Figure \ref{fig:pfk2}). Notably, the activity
recovery from the freeze-drying process raised to 70 \% via additives of
\SI{25}{\milli\Molar} glucose and 1\% PEG. 
% --------------------------
\begin{figure}[htbp] \centering \includegraphics[width=0.5\textwidth]{fig3_24}
    \caption[Comparison of influence of glucose alone (circle) and glucose with
    1\% PEG (square) on stability of freeze-dried PFK.]{Comparison of influence
        of glucose alone (circle) and glucose with 1\% PEG (square) on
        stability of freeze-dried PFK \cite{Carpenter1993}.} \label{fig:pfk2}
    \end{figure}
% --------------------------

\subsection{PTE and Its Applications}

Enzyme-based decontamination would provide a significant logistical reduction
during the processing (25 to 50 fold) \cite{Defrank}. Phosphotriesterase (PTE)
is a dimeric protein that hydrolyzes a wide range of organophosphates (OPs).
PTE has been demonstrated to neutralize OPs, such as pesticides and chemical
weapons (Section \ref{sec:pte-intro}), rendering PTE an excellent
agent for detoxification. While PTE can catalyze OPs, it is unstable. Thus,
approaches to stable PTE for prolonged use outside the biological confines of
the cell have been developed. 

To stabilize PTE, it has been immobilized on trityl
agarose resin, allowing decontamination of OPs such as paraoxon, parathion, and
coumaphos \cite{Caldwell1991}. Caldwell \latin{et al.} employs trityl agarose
because of the high coupling properties and the high retention of activity upon
the immobilization of enzymes. Figure \ref{fig:pte-app-imm} demonstrates the
correlation between the flow rate, paraoxon concentration and the percentage of
hydrolysis. A 50\% hydrolysis of paraoxon is achieved with flow rate of
\SI{11.5}{\mL\per\hour}. Experiments has been also conducted at the 1.0 and 10-unit
reactors. Unfortunately, the organic solvent dissociated the enzyme from the
reactor.
% --------------------------pte-app
\begin{figure}[htbp] \centering \includegraphics[width=0.7\textwidth]{fig3_19}
    \caption[The hydrolysis of paraoxon in \SI{125}{\milli\Molar}
        2-(N-cyclohexylamino)ethanesulfonic acid (CHES), 10\% methanol at
    varied flow rates with an immobilized phosphotriesterase fixed reactor (0.1
unit) (triangle, \SI{0.92}{\milli\Molar}, closed circle,
\SI{0.1}{\milli\Molar}, open circle, \SI{0.046}{\milli\Molar}); (1.0 unit)
(closed square, \SI{0.17}{\milli\Molar}, open square,
\SI{0.087}{\milli\Molar}); (10.0 unit) (Diamond, \SI{0.97}{\milli\Molar}).]{The
    hydrolysis of paraoxon in \SI{125}{\milli\Molar} 2-(N-cyclohexylamino)
    ethanesulfonic acid (CHES), 10\% methanol at varied flow rates with an
    immobilized phosphotriesterase fixed reactor (0.1 unit) (triangle,
    \SI{0.92}{\milli\Molar}, closed circle, \SI{0.1}{\milli\Molar}, open
    circle, \SI{0.046}{\milli\Molar}); (1.0 unit) (closed square,
    \SI{0.17}{\milli\Molar}, open square, \SI{0.087}{\milli\Molar}); (10.0
    unit) (Diamond, \SI{0.97}{\milli\Molar}) \cite{Caldwell1991}.}
    \label{fig:pte-app-imm}
\end{figure}
% --------------------------

NATO project 31 previously conducted research on several methods for PTE
stabilization \cite{Defrank}, such as foam \cite{LeJeune1997a}, microemulsion
\cite{Cheng1996}, and polyurethane \cite{Defrank}. One of the examples is
the foam formulation that was developed by LeJeune \latin{et al.}
\cite{LeJeune1997a} (Figure \ref{fig:pte-foam}).
% -------------------------- pte-foam 
\begin{figure}[htbp]
    \centering \includegraphics[width=0.7\textwidth]{fig3_06}
    \caption[Photographs of synthesized polymers. Polymers were synthesized
    with Hypol 3000 surfactant. A 1:1 wt ratio of prepolymer to aqueous phase
    was used with 1 wt \% surfactant in the aqueous phase. PTE
    (\SI{0.6}{\mg\per\mL}) was added (\SI{10}{\mL}) to a buffered aqueous
    solution (\SI{0.12}{\Molar} Hepes buffer, pH 7.4,
    \SI{50}{\milli\Molar}\ch{CoCl2}) containing 1 \% Pluronic P-65 surfactant
    (\SI{5}{\mL}). Approximately \SI{4}{\mL} of Hypol 3000, a toluene
    diisocyanate based polyurethane prepolymer, were added to the PTE
solution(\SI{0.6}{\mg\per\mL}).]{Photographs of synthesized polymers. Polymers
    were synthesized with Hypol 3000 surfactant. A 1:1 wt ratio of prepolymer
    to aqueous phase was used with 1 wt \% surfactant in the aqueous phase. PTE
    (\SI{0.6}{\mg\per\mL}) was added (\SI{10}{\mL}) to a buffered aqueous
    solution (\SI{0.12}{\Molar} Hepes buffer, pH 7.4,
    \SI{50}{\milli\Molar}\ch{CoCl2}) containing 1\% Pluronic P-65 surfactant
    (\SI{5}{\mL}). Approximately \SI{4}{\mL} of Hypol 3000, a toluene
    diisocyanate based polyurethane prepolymer, were added to the PTE
    solution(\SI{0.6}{\mg\per\mL}) \cite{LeJeune1997a}.} 
    \label{fig:pte-foam}
\end{figure}
% --------------------------*

The PTE was immobilized with Hypol polyurethane prepolymers. The immobilization
extended the storage to more than 25 days. They further demonstrated that the
foam also stabilizes PTE at \SI{50}{\celsius} for more than 158 hours in
comparison with soluble enzyme at \SI{50}{\celsius} for merely 1.5 hours. 

\subsubsection{PTE Heat I/activation}

Heat inactivation affected PTE functionality in solution. The thermostability
studies had been carried out to characterize PTE
\cite{Rochu2002b,Grimsley1997b}. Rochu \latin{et al.} demonstrated the
heat-induced unfolding profile of PTE \cite{Rochu2002b} (Figure
\ref{fig:pte-thermo-inactive}). 
% --------------------------pte-heat-inactivation
\begin{figure}[htbp] \centering \includegraphics[width=0.7\textwidth]{fig3_17}
    \caption[Effect of heat treatment on the catalytic activity of PTE in
    \SI{200}{\milli\Molar} borate (pH 9.4). Samples were heated in a water bath
for 15 min at the desired temperature, then cooled to \SI{25}{\celsius} and the
PTE activity immediately measured. This validates the reality of the zig-zag
plot with an activation phase near \SI{46}{\celsius}. The inset depicts the
thermal inactivation profile of the enzyme as a function of time at
\SI{60}{\celsius}. The percent residual activity is shown for treated
(triangle) or without (square) previous incubation for 15 min at
\SI{60}{\celsius}.]{Effect of heat treatment on the catalytic activity of PTE
    in \SI{200}{\milli\Molar} borate (pH 9.4). Samples were heated in a water
    bath for 15 min at the desired temperature, then cooled to
    \SI{25}{\celsius} and the PTE activity immediately measured. This validates
    the reality of the zig-zag plot with an activation phase near
    \SI{46}{\celsius}. The inset depicts the thermal inactivation profile of
    the enzyme as a function of time at \SI{60}{\celsius}. The percent
    residual activity is shown for treated (triangle) or without (square)
    previous incubation for 15 min at \SI{60}{\celsius} \cite{Rochu2002b}.}
    \label{fig:pte-thermo-inactive} 
\end{figure}
% --------------------------

Upon elevated temperatures, Rochu \latin{et al.} demonstrated that denaturation
irreversible, leading to aggregates at high temperatures with an apparent
T\textsubscript{m} ~ \SI{75}{\celsius}. Above \SI{46}{\celsius}, a rapid loss
of activity occurred with complete inactivation at \SI{60}{\celsius}. 

\subsubsection{PTE Organic Solvent Inactivation}

Organic solvent has been demonstrated to impact enzymes stabilities and
activities. LeJeune \latin{et al.} investigated the inactivation of PTE by
contacting soluble enzyme with different concentrations of dimethyl sulfoxide
(DMSO) \cite{Lejeune1997a} (Figure \ref{fig:pte-organic-inactive}).
% --------------------------pte-organic
\begin{figure}[htbp] \centering \includegraphics[width=0.5\textwidth]{fig3_18}
    \caption[Dependence of soluble PTE activity in the
    presence of DMSO during enzyme assay. Assay conditions were
\SI{0.5}{\milli\Molar} paraoxon at \SI{25}{\celsius}, enzyme concentration =
0.9 pM.]{Dependence of soluble PTE activity in the presence of
    DMSO during enzyme assay. Assay conditions were \SI{0.5}{\milli\Molar}
    paraoxon at \SI{25}{\celsius}, enzyme concentration = 0.9 pM
    \cite{Lejeune1997a}.}
    \label{fig:pte-organic-inactive}
\end{figure}
% --------------------------

After DMSO was added to the hydrolysis of paraoxon in
aqueous solution, PTE activity decreased dramatically. LeJeune \latin{et al.}
attributed this phenomenon to sources such as environmental dielectric changes
and competitive and noncompetitive inhibition \cite{Lejeune1997a}. The results
also demonstrated PTE with DMSO concentrations above 40\% exhibited substantial
irreversible effects \cite{Lejeune1997a}.

\subsection{Lactose Monohydrate}

Lactose (\iupac{4-\O-\chembeta-\D-galactopyranosyl-\D-glucopyranose},
\ce{C12H22O11}) is a disaccharide consisting of a \iupac{\D-glucose} and a
\iupac{\D-galactose} joined by a \iupac{\chembeta-1,4-glycosidic} bond (Figure
\ref{fig:lactose-structure}). In the dairy industry, crystallization is a
separation process that refines lactose from whey solutions
\cite{Hourigan2013}. During this operation, the crystallization is considered
as a two-step process; the first is called the nucleation
\cite{Schmitt1999,Wong2014}, and second is the growth of the nucleus.  This
nuclei from the first step grows depending on the condition of supersaturation
of lactose solution.  The lactose crystals are affected by process parameters
and solutions, including, temperature, viscosity, pH, and presence of
impurities \cite{Bhargava1996}. The pH changes the rate of mutarotation, which
is an important factor in lactose crystallization \cite{Hourigan2013}. 
% --------------------------
\begin{figure}[h!] \centering \includegraphics[width=0.5\textwidth]{fig3_01}
    \caption[Molecular structures of $\alpha$- and $\beta$- lactose.]{Molecular
    structures of $\alpha$- and $\beta$- lactose.}
    \label{fig:lactose-structure}
\end{figure}
% --------------------------

A diagram of a typical $\alpha$-lactose monohydrate (LM) crystal is shown in
Figure \ref{fig:lm-crystal}. The crystals are shaped as tomahawk with one
axis of symmetry. The solvent compositions and temperatures during the
crystallization result in different yields, purities, shapes, and sizes of the
lactose crystals \cite{Hourigan2013}.
% --------------------------ALM-crystal
\begin{figure}[h!] \centering \includegraphics[width=0.5\textwidth]{fig3_11}
    \caption[Tomahawk crystal of $\alpha$-lactose monohydrate showing faceted
    structure.]{Tomahawk crystal of $\alpha$-lactose monohydrate showing
        faceted structure \cite{Wong2014}.} 
    \label{fig:lm-crystal} 
\end{figure}
% --------------------------

It has been previously shown that $\alpha$-lactose monohydrate (LM) is capable
of incorporating various macromolecules and biopolymers into its crystal
structure\cite{Wang2001a,Kurimoto1999}. (Figure \ref{fig:lactose-structure}) It
has been hypothesized that this is due to peripheral hydrogen atoms of LM,
which are capable of forming hydrogen bonds to neighboring macromolecules.
% --------------------------
\begin{figure}[h!] \centering \includegraphics[width=0.7\textwidth]{fig3_02} 
    \caption[Crystals of lactose monohydrate (LM) as hosts for the guest green
    fluorescent protein (GFP)]{Crystals of lactose monohydrate (LM) as hosts
        for the guest green fluorescent protein (GFP) \cite{Wang2001a}.}
    \label{fig:lm-intro}
\end{figure}
% --------------------------

Wang \latin{et al.} demonstrated that multiple guests molecules were
crystallized with LM \cite{Wang2001a} such as GFP, Cyt C, lysozyme and RNAse B.
Kurimoto \latin{et al.} investigated the stability of GFP inside the crystals.
Surprisingly, they oriented and stabilized GFP in its native conformation, and
subsequently reconstituted crystals into solution in its native state
\cite{Kurimoto1999}.

In this chapter, the enhanced stability and extended shelf life at room
temperature of PTE within the crystallization of LM would be demonstrated. This
is the first report that uses enzyme kinetic parameter to evaluate the
stability in LM crystal. The formulation provides effective storage of PTE, and
ultimately can be used for multiple applications in agriculture and military. 

\section{Methods}

\subsection{General}

Fluorescein isothiocyanate (FITC) FluoroTag kit and paraoxon were purchased
from Sigma (St. Louis, MO).  $\alpha$-lactose monohydrate was also purchased
from Sigma (St. Louis, MO). All other chemicals, including \ch{NaCl},
\ch{CoCl2}, Tris-HCl, tryptone, yeast extract, paraoxon, ampicillin,
chloramphenicol, sodium phosphates monobasic, sodium phosphate dibasic, or
imidazole were purchased from Sigma (St. Louis, MO) or VWR (Radnor, PA).
96-well plates were purchased from Thermo Fisher Scientific (Waltham, MA). FPLC
column was purchased from G.E Healthcare (Piscataway, NJ). 

\subsection{Protein Expression}
\label{sec:pte-chap3}

PTE DNA, pQE30-PTE, was transformed into AFIQ cells as described in our previous
work\cite{Yang2014a} and Chapter \ref{chap:uaa}. Cells were plated on agar
plates containing \SI{200}{\ug\per\mL} ampicillin, \SI{34}{\ug\per\mL}
chloramphenicol. A single colony was picked and grown in Lysogeny broth (LB)
with \SI{200}{\ug\per\mL} ampicillin, and \SI{34}{\ug\per\mL} chloramphenicol)
at \SI{37}{\celsius}, 300 r.p.m for 16 hours \SI{37}{\celsius} incubation.
Afterwards, \SI{250}{\mL} of LB medium for large-scale expression was
innoculated 1:50 with the overnight culture.  After optical density reached 1.0
at 600 nm, the expression media was supplemented with \SI{1}{\milli\Molar}
isopropyl-$\beta$-D-thiogalactopyranoside (IPTG) to induce protein expression.
In addition, \SI{1}{\milli\Molar} of \ce{CoCl2} was added in each
post-induction medium.  After three hours incubation at \SI{37}{\celsius}, 300
r.p.m., the cells were harvested by using 4000 r.p.m centrifugation (Beckman
Coulter, Jersey City, NJ.  F10 rotor) at \SI{4}{\celsius} for 15 minutes and
then resuspended with \SI{20}{\milli\Molar} Tris-HCl, \SI{500}{\milli\Molar}
\ce{NaCl}, \SI{5}{\milli\Molar} imidazole, 10\% glycerol (pH 8.0) and
\SI{100}{\micro\Molar} \ce{CoCl2}. Cell lysate was immediately sonicated at 400 kJ
for 1.5 minutes at \SI{4}{\celsius} (Q500 sonicator, Qsonica, Newtown, CT) and then a
clarification spin was performed (20,000 g, \SI{4}{\celsius}, 30 minutes).
Clarified supernatants were loaded into a \SI{5}{\mL} His Trap column (G.E
Healthcare, Piscataway, NJ) using AKTA FPLC purifier (G.E.  Healthcare,
Piscataway, NJ).  Protein elution was generated using 30\% elution buffer B
(\SI{20}{\milli\Molar} Tris-HCl, \SI{500}{\milli\Molar} sodium chloride,
\SI{500}{\milli\Molar} imidazole (pH 8.0)). The purified samples were then
transferred into 3.5K MWCO dialysis SnakeSkin (Life Technologies, Carlsbad, CA)
for buffer exchange using \SI{12}{\L} \SI{20}{\milli\Molar} phosphate buffer
(pH 8.0) at \SI{4}{\celsius} overnight. The purify of protein was
determined by sodium dodecyl sulfate polyacrylamide gel electrophoresis analysis
(SDS-PAGE). The protein concentration was measured by Nano-Drop Thermo
Scientific (Waltham, MA) by using the extinction coefficient
\SI{29575}{\per\Molar\per\cm} for PTE \cite{Gasteiger2005, Pace1995}. Dialyzed
protein was subjected to crystallization immediately.

\subsection{Protein Conjugation}

Fluorescein isothiocyanate (FITC) kit was used for fluorescein conjugation with
PTE. FITC dissolved in \SI{0.1}{\Molar} sodium carbonate-bicarbonate buffer (pH
9.0).  \SI{1}{\mg\per\mL} was added to the \SI{0.14}{\mg\per\mL} enzyme in
\SI{20}{\milli\Molar} phosphate buffer (pH 8.0, \SI{100}{\micro\Molar}
\ce{CoCl2}) to a final ratio of 1:5 (PTE:FITC). The reaction mixture was
incubated for \SI{1.5}{hour} at room temperature in tube rack, then dialyzed
against \SI{2}{\liter} of phosphate buffer (\SI{20}{\milli\Molar}, pH 8.0,
\SI{100}{\micro\Molar} \ce{CoCl2}) at \SI{4}{\celsius} overnight.  The sample
was then transferred to a lactose solution for crystallization.

\subsection{Lactose and Crystallization}

To prepare PTE.LM crystals, a total of \SI{0.20}{\milli\gram} purified PTE was
added to \SI{2}{\mL} \SI{0.28}{\gram\per\mL} lactose solution
(\SI{20}{\milli\Molar} PBS buffer, pH 7.4). The super saturated lactose was
prepared by dissolving \SI{0.6}{\gram} $\alpha$-lactose monohydrate into boiled
\SI{20}{\milli\Molar} PBS buffer (pH 7.4). PTE protein (in
\SI{20}{\milli\Molar} phosphate buffer, \SI{100}{\micro\Molar} \ce{CoCl2}, pH
8.0) was prepared according to the procedure in Section \ref{sec:pte-chap3}.
After the lactose solution was cooled down to ~\SI{40}{\celsius},
\SI{0.20}{\gram} PTE was then The mixture was incubated at \SI{6}{\celsius} for
roughly 2.5 weeks until PTE.LM crystals of a suitable size were obtained.
Powder lactose monohydrate (less than \SI{1}{\mg})was used as the seed for
crystallization. The crystals were then harvested by filtration and washed with
distilled water and dried under room temperature. PTE.LM crystals were then
stored in vials at \SI{4}{\celsius} or room temperature for subsequent
microscope studies or paraoxon hydrolysis experiments.

\subsection{Fluorescence Phase Contrast Microscopy}

The crystal (FITC-PTE.LM) was then first observed with Zeiss Axioskop 40
Fluorescence Microscope (Peabody, MA).  Wavelength \SI{495}{\nm} was used for
excitation, and \SI{525}{\nm} was used for emission. The 4X object lens was used to
collect images of F-PTE.LM.

\subsection{Enzyme Kinetics}

The crystal (both PTE.LM and PTE.LM-FITC) was reconstituted in
\SI{20}{\micro\liter} sodium phosphate (pH 8.0, \SI{100}{\micro\Molar}
\ce{CoCl2}). Reactions were monitored spectrophotometrically (Synergy H1,
BioTek, Winooski VT) at \SI{405}{\nm} for paraoxon (coefficient =
\SI{17000}{\per\Molar\per\cm}).  Reactions for paraoxon
(\SIrange{13}{104}{\micro\Molar}) was done in 0.4\% methanol.
K\textsubscript{M} and V\textsubscript{max} values were determined by a
Lineweaver-Burk plot \cite{Baker2011b}. The equation used is shown below
(Eq.~\ref{eqn:MM-chap3}): 
\begin{equation} 
    \frac{1}{v} =
    \frac{K\textsubscript{M}}{V\textsubscript{max}}\times\frac{1}{S} +
    \frac{1}{V\textsubscript{max}} 
    \label{eqn:MM-chap3}
\end{equation}
where S represents substrate concentration in \si{\Molar}; K\textsubscript{M}
represents the substrate concentration at which the reaction rate is half of
V\textsubscript{max}. The data reported is the average of three trials and the
error represents the standard deviation of those trials.

\section{Results and Discussion}

\subsection{Biosynthesis of PTE And Crystallization of PTE.LM}

Wild-type PTE was biosynthesized from the pQE30-PTE plasmid \cite{Yang2014a}.
After transformation into AFIQ \latin{E. coli} \cite{Baker2011b}, expression
was performed in LB media. After three hours of induction with
\SI{1}{\milli\Molar} IPTG, pure PTE protein was isolated via FPLC (Chapter
\ref{chap:uaa}), and the purity of protein was analyzed via SDS-PAGE. After
dialysis in \SI{20}{\milli\Molar} phosphate buffer (pH 8.0,
\SI{100}{\micro\Molar} \ce{CoCl2}), the protein concentration was determined to
be \SI{0.139}{\mg\per\mL} by using Nano-Drop.

To prepare the lactose solution for crystallization, supersaturated
solutions of LM was incubated with PTE protein at \SI{40}{\celsius}. The
mixture was then cooled down to room temperature at the rate of
\SI{0.5}{\celsius\per\minute}. The resulting PTE.LM crystals were harvested
after roughly 2.5 weeks incubation at \SI{4}{\celsius} (Figure
\ref{fig:ptelm-bottle}). 
% --------------------------ptelm-bottle
\begin{figure}[htbp] \centering \includegraphics[width=0.35\textwidth]{fig3_07}
    \caption[The crystal of PTE.LM. (Left) A bottle contains
        \SI{79}{\mg} of PTE.LM after 2.5 weeks incubation at \SI{4}{\celsius}.
        \SI{0.2}mg PTE were crystallized in \SI{0.28}{\gram\per\mL} lactose
        solution (\SI{20}{\milli\Molar} PBS buffer, pH 7.4).]{The
            crystal of PTE.LM. (Left) A bottle contains \SI{79}{\mg} of PTE.LM
            after 2.5 weeks incubation at \SI{4}{\celsius}.  \SI{0.2}mg PTE
            were crystallized in \SI{0.28}{\gram\per\mL} lactose solution
            (\SI{20}{\milli\Molar} PBS buffer, pH 7.4).}
        \label{fig:ptelm-bottle} 
\end{figure}
% --------------------------*

Using supersaturated lactose solution, purified PTE was successfully deposited
onto the lactose crystals yielding PTE.LM crystals measuring
\SIrange{0.5}{3}{\mm} in length. Using phase contrast microscope, the dimension
of a PTE.LM crystal was calculated at \SI{3.0}{\mm} (h) $\times$ \SI{1.0}{\mm}
(w) $\times$ \SI{0.5}{\mm} (d) (Figure \ref{fig:ptelm-image}). While Kurimoto
\latin{et al.} \cite{Kurimoto1999} and Wang \latin{et al.} \cite{Wang2001a}
reported the similar sizes of crystal, the shape of PTE.LM crystals did not
exhibit tomahawk form. Previously, protein co-crystals were expected to display
a hatchet morphology having a broad base (010) further bounded by (100), (110)
and (011) \cite{Kurimoto1999}.  However, due to the fluctuations from factors,
pH and the drying process, the irregular shape of PTE.LM was expected. Hourigan
\latin{et al.} reported amorphous crystals during the crystallization, the pH
and drying process of PTE.LM should be adjusted. 
% --------------------------
\begin{figure}[h!] \centering \includegraphics[width=0.7\textwidth]{fig3_03} 
    \caption[Microscope image of PTE.LM. The estimated dimension of amorphous
    PTE.LM crystal are estimated as  \SI{3.0}{\mm} (h) $\times$ \SI{1.0}{\mm}
(w) $\times$ \SI{0.5}{\mm} (d).]{Microscope image of PTE.LM. The estimated
    dimension of amorphous PTE.LM crystal are estimated as  \SI{3.0}{\mm} (h)
    $\times$ \SI{1.0}{\mm} (w) $\times$ \SI{0.5}{\mm} (d).}
    \label{fig:ptelm-image} 
\end{figure}
% --------------------------

\subsection{Reconstituted PTE.LM Hydrolysis Reaction}

While wild-type PTE in solution hydrolyzed OP substrates \cite{Yang2014a,Baker2011b},
it exhibited short half life, losing activity within one week (Chapter
\ref{chap:uaa}, Figure \ref{fig:kinetics-fig}). Further evidence demonstrated
that PTE required stabilization through embodiments, such as immobilization and
formulation discussed in introduction
\cite{Chen1998,Gill2000,Havens1993,Masson2009a}. To further stabilize PTE, it
was further crystallized in the presence of LM, and its ability to hydrolyze
paraoxon after storage was assessed. After the crystallization, PTE.LM was
stored at room temperature or \SI{4}{\celsius}. One month old PTE.LM stored at
\SI{4}{\celsius} revealed activity for paraoxon hydrolysis after reconstituted
in \SI{200}{\micro\liter} \SI{20}{\milli\Molar} (pH 8, \SI{100}{\micro\Molar}
\ce{CoCl2}) phosphate buffer (Figure \ref{fig:ptelm-one-month}) at room
temperature. 
% --------------------------
\begin{figure}[htbp] \centering \includegraphics[width=0.7\textwidth]{fig3_08}
    \caption[Reconstituted one month old PTE.LM hydrolysis of paraoxon.
        \SI{10}{\mg} of PTE.LM was dissolved in \SI{200}{\micro\liter} of
        \SI{20}{\milli\Molar} phosphate buffer. Paraoxon was prepared at
        \SI{200}{\micro\liter} in \SI{20}{\milli\Molar} phosphate buffer. The
    hydrolysis reaction was measured at \SI{405}{\nm}.] {Reconstituted one month
        old PTE.LM hydrolysis of paraoxon.  \SI{10}{\mg} of PTE.LM was
        dissolved in \SI{200}{\micro\liter} of \SI{20}{\milli\Molar} phosphate
        buffer.  Paraoxon was prepared at \SI{200}{\micro\liter} in
        \SI{20}{\milli\Molar} phosphate buffer. The hydrolysis reaction was
        measured at \SI{405}{\nm}.} 
        \label{fig:ptelm-one-month} 
\end{figure}
% --------------------------

The absorbance at \SI{405}{\nm} clearly demonstrated the
increased hydrolysis of \emph{p}-nitrophenol by the one-month old PTE.LM
(Figure \ref{fig:ptelm-one-month}). PTE.LM crystals after 2 months
storage at \SI{4}{\celsius} also demonstrated hydrolysis of paraoxon when the
protein stored in buffer is inactive (Figure \ref{fig:ptelm-two-month}). By
contrast, paraoxon saturated in the absence of PTE revealed no hydrolysis,
indicating that the PTE.LM crystals indeed were active. Assuming the same
amount of PTE within the crystal, the 2-month sample exhibited less activity
(Figure \ref{fig:ptelm-two-month}). When compared to a dread PTE sample
dissolved in buffer, the 2-month sample exhibited activity while no detectable
hydrolysis was observed with the dried stored PTE(Figure
\ref{fig:ptelm-two-month}). This suggested that the crystallization of PTE in
LM provided stability.
% --------------------------
\begin{figure}[htbp] \centering \includegraphics[width=0.5\textwidth]{fig3_09}
    \caption[Reconstituted two month old PTE.LM hydrolysis of paraoxon.
        \SI{10}{\mg} of PTE.LM was dissolved in \SI{200}{\micro\liter} of
        \SI{20}{\milli\Molar} phosphate buffer. Paraoxon was prepared at
        \SI{200}{\micro\liter} in \SI{20}{\milli\Molar} phosphate buffer. The
        hydrolysis reaction was measured at \SI{405}{\nm}. Dread PTE represents
        the air-dried PTE that was stored at room temperature and dissolved in
        \SI{20}{\milli\Molar} phosphate buffer (pH 8, \SI{100}{\micro\Molar}
    \ce{CoCl2}).] {Reconstituted two month old PTE.LM hydrolysis of paraoxon.
        \SI{10}{\mg} of PTE.LM was dissolved in \SI{200}{\micro\liter} of
        \SI{20}{\milli\Molar} phosphate buffer. Paraoxon was prepared at
        \SI{200}{\micro\liter} in \SI{20}{\milli\Molar} phosphate buffer. The
        hydrolysis reaction was measured at \SI{405}{\nm}.  Dread PTE
        represents the air-dried PTE from the soluble sample stored at room
        temperature. The dread PTE was then dissolved in \SI{20}{\milli\Molar}
        phosphate buffer (pH 8, \SI{100}{\micro\Molar} \ce{CoCl2}).}
    \label{fig:ptelm-two-month} 
\end{figure}
% --------------------------

In the meanwhile, PTE.LM was stored at room temperature to investigate
half-life and stability of PTE.LM.  After storing at room temperature for ~3.5
months, \SI{0.08}{\gram\per\mL} PTE.LM crystals was then reconstituted in
\SI{20}{\milli\Molar} phosphate buffer (pH 8, \SI{100}{\micro\Molar}
\ce{CoCl2}). \SIrange{13}{104}{\micro\Molar} paraoxon was then used for assays.
Within three minutes, reactions reached the maximum absorbance at
\SI{405}{\nm} (Figure \ref{fig:ptelm-hydrolysis}). Within the linear range
of the reaction, enzyme kinetics was calculated with V\textsubscript{max} and
K\textsubscript{M} of 0.0043 $\pm$ \SI{0.0012}{\micro\Molar\per\second} and 1186
$\pm$ \SI{336}{\micro\Molar}, respectively (Table \ref{tab:ptelm-table}).
Compared to the frsshly prepared wild-type PTE in solution, reconstituted
PTE.LM demonstrated decreased activity on both K\textsubscript{M} and Vmax
values. Due to the elevated temperature at \SI{40}{\celsius} during the
crystallization process, the PTE activity may be partially impaired from heat
inactivation as observed from Masson and Tawfik group
\cite{Rochu2002b,Roodveldt2005}. However, this subtle changes in temperature
was not the only factor. The dilution of \ce{Co^{2+}} in the active site also
impacted the PTE catalytic efficiency. Raushel group demonstrated that the
metal ions deprivation in the active site led to the loss of activity
\cite{Benning1995,Samples2005}. 
% --------------------------
\begin{figure}[htbp] \centering \includegraphics[width=0.8\textwidth]{fig3_05} 
    \caption[The hydrolysis of reconstituted PTE.LM after stored at room
    temperature for 3.5 months.]{The hydrolysis of reconstituted PTE.LM after
    stored at room temperature for 3.5 months.} \label{fig:ptelm-hydrolysis} 
\end{figure}
% --------------------------

% --------------------------ptelm-table
\begin{table}[h!]
    \centering
    \begin{tabular}{ lll }
        \hline
        Protein & K\textsubscript{M} & Vmax \\
        \hline
        PTE & 342 $\pm$ 102 & 0.51 $\pm$ 0.13 \\
        PTE.LM & 1186 $\pm$ 336 & 0.0043 $\pm$ 0.0012 \\
        \hline
    \end{tabular}
    \caption[Kinetics comparisons of PTE and PTE.LM.]{Kinetics comparisons of PTE and PTE.LM.} 
    \label{tab:ptelm-table} 
\end{table}
% --------------------------

\subsection{FITC Conjugation And Crystallization}

To calculate the amounts of non-GFP protein in the crystal, Wang \latin{et al.}
conjugated fluorescein into protein, such as RNAse A, RNAse B, Avidin, and
NeutrAvidin \cite{Wang2001a}. Using fluorescein, the location of protein at the
(010) growth sector was visulaized. The PTE was conjugated with fluorescein
isothiocyanate (FTIC) to label with the lysine and C-terminus of PTE
\cite{Rogers1999}. The FITC probe has an absorption at \SI{495}{\nm} and
emission at \SI{525}{\nm}, which forms stable conjugation with free amino
groups of proteins. 
The FITC was incubated with PTE and performed dialysis at \SI{4}{\celsius}. After
dialysis, we deposited the conjugation product into LM with the same
crystallization procedure. After two weeks, we collected the crystal
(PTE.LM-FITC) and use fluorescence microscope to locate the probe conjugated
with PTE. (Figure \ref{fig:ptelm-fitc})
% --------------------------
\begin{figure}[htbp] \centering \includegraphics[width=0.8\textwidth]{fig3_10} 
    \caption[The image of PTE.LM-FITC crystal. An absorption at \SI{495}{\nm}
    and an emission at \SI{525}{\nm} were used for FITC probe. The crystal was
collected after two weeks crystallization.]{The image of PTE.LM-FITC crystal.
    An absorption at \SI{495}{\nm} and an emission at \SI{525}{\nm} were used
for FITC probe. The crystal was collected after two weeks crystallization.}
\label{fig:ptelm-fitc} 
\end{figure}
% --------------------------

From Figure \ref{fig:ptelm-fitc}, the crystal demonstrated the distribution of
PTE-FITC. Unfortunately, we did not find the (010) pattern inside the crystal.
This might be resulted from several factors, such as temperatures and pH during
the crystallization \cite{Wong2014}. As crystal visualization served as one of
goal of formulation, we also investigated the PTE stability in this study.
After collecting the PTE.LM-FITC, we reconstituted the crystal with
\SI{20}{\milli\Molar} pH 8 phosphate buffer. Unlike the hydrolysis from PTE.LM,
we were not able to measure the catalytic reaction. The loss of enzyme activity
may be due to the FITC conjugation, which was previously reported by
Rogers \latin{et al.} \cite{Rogers1999}. To visualize the functional PTE inside
the crystal, especially at extremely low concentration, alternative methodology
is required in this study.

\section{Future work}

Due to the limitation of detection methods that we previously used, we were not
able to determine protein concentration via spectrophotometer or BCA. Further
quantification is needed for k\textsubscript{cat}/K\textsubscript{M}.

% removed content
%\subsubsection{Other Methods}
%
%While there are limitations of uses of enzymes, other methodologies are
%developed to stabilize enzymes, such as entrapment
%\cite{Trelles2013,Bhosale1996,Etzel1996}. Entrapment is used in industrial
%scale enzyme operations. As enzymes of interest are intracellular, it is easier
%to use whole-cell entrapment where the enzyme is synthesized in bacteria.
%\cite{Trelles2013} (Figure \ref{fig:enzyme-entrapment})
%% --------------------------
%\begin{figure}[h!] \centering \includegraphics[width=0.7\textwidth]{fig3_04}
%    \caption[Schematic drawing of the agar or agarose entrapment
%    procedure]{Schematic drawing of the agar or agarose entrapment
%        procedure.\cite{Trelles2013}}
%    \label{fig:enzyme-entrapment} 
%\end{figure}
%% --------------------------
%
%One of examples of this is glucose isomerase (GI), which has been used in the
%commercial production of high fructose corn syrup (HFCS) \cite{Bhosale1996}.
%There are reported more than 6 millions tons of HFCS per year produced by GI.
%Compared with chemical isomerization, using enzyme not only increased the
%purify of the product fructose \cite{Barker1975}, but also enhanced sweetness
%\cite{Bhosale1996}. To save the cost of isolating GI, the whole-cell
%entrapment was developed to adapt the increased demand of such enzyme
%\cite{Trelles2013}. 
%
%The other application was developed for the decontamination of
%organophosphates
%\cite{Cheng1996,LeJeune1997a,Little1989,Chen1998,Gill2000,Havens1993,Masson2009a}.

\printbibliography[heading=subbibliography]

\end{refsection}

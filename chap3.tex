\chapter{Phosphotriesterase Stabilization Via Lactose Monohydrate Formulation}
\label{chap:lactose}

\begin{refsection}

\section{Introduction}

\subsection{Enzyme Applications}

The catalytic properties of enzymes has led to extensive development in industry
as briefly discussed in the previous chapter \ref{tab:protein-app} (Table
\ref{tab:protein-app}). For example, protease has been developed for the use in
detergent \cite{Kirk2002}. After several iterations of development, protease
has been engineered to meet requirements in detergents for laundry.  However,
inactivation of enzymes continue to limit their use. The inactivation factors
include: heat \cite{Shirley1995,Perdana2012,Etzel1996,Gouda2003}, proteolysis
\cite{VandenBurg2002a,Ahmad2012}, and organic solvents
\cite{LeJeune1997a,Izutsu2009,Stepankova2013}. Several approaches have been
developed to address these limitations
\cite{Caldwell1991,LeJeune1997a,You1996,Gouda2003,Ahmad2012}. The following is
a discussion of these limitations and solutions that include addition of salts
\cite{Appleton1997,Gouda2003}, directed evolution / or mutagenesis
\cite{Kamerzell2008,Xiong2007,Ahmad2012,You1996,Tsai2012b}, and immobilization
\cite{Stepankova2013,Dravis2001,LeJeune1997a,Barbosa2014}.

\subsubsection{Heat Inactivation}

Enzymes are sensitive to temperatures, especially under elevated temperatures
\cite{Perdana2012,Etzel1996}. For example, glucose oxidase \cite{Gouda2003} is
susceptible to heat inactivation over \SI{60}{\celsius} \cite{Gouda2003} (Figure
\ref{fig:heat-inactivation-exmaple}). To improve the thermal stability of
glucose oxidase, additives such as salts, mono- and polyhydric alcohols, as
well as polyelectrolytes have been added \cite{Appleton1997,Gouda2003}. To
understand the stabilization by salts such as \ce{NaCl} and \ce{K2SO4}, Gouda
\latin{et al.} have determined thermal unfolding transitions of glucose oxidase in
the presence of \SI{0.6}{\Molar} \ce{NaCl} and \SI{0.2}{\Molar} \ce{K2SO4} by
circular dichroism (CD) measurements \cite{Gouda2003} (Figure
\ref{fig:heat-inactivation-improve}). The melting temperature
(T\textsubscript{m}) measured at 274 nm was shifted from \SI{62}{\celsius} for
native enzyme to \SI{68}{\celsius} in the presence of \SI{0.2}{\Molar}
\ce{K2SO4} and to \SI{72}{\celsius} in the presence of \SI{0.6}{\Molar}
\ce{NaCl}, leading to stabilization (Figure
\ref{fig:heat-inactivation-improve}). Using such stabilizing effects of salts,
they successfully demonstrated how the heat limitations could be overcame. 

% --------------------------heat-inactivation-exmaple
\begin{figure}[htbp] \centering \includegraphics[width=0.7\textwidth]{fig3_12}
    \caption[The heat inactivation of glucose oxidase (in the absence of
        additives) at \SI{56}{\celsius} (diamond); \SI{60}{\celsius} (square);
        \SI{63}{\celsius} (triangle); and \SI{67}{\celsius} (circle). Samples
        were incubated at the desinated temperatures.]{The heat inactivation of
            glucose oxidase (in the absence of additives) at \SI{56}{\celsius}
            (diamond); \SI{60}{\celsius} (square); \SI{63}{\celsius}
            (triangle); and \SI{67}{\celsius} (circle). Samples were incubated
            at the desinated temperatures \cite{Gouda2003}.}
    \label{fig:heat-inactivation-exmaple}
\end{figure}
% --------------------------  
% --------------------------heat-inactivation-improvement
\begin{figure}[htbp] \centering \includegraphics[width=0.7\textwidth]{fig3_13} 
    \caption[Effect of salts on the thermal unfolding of  glucose oxidase.
        Ellipticity at (A) 274 nm and (B) 222 nm as a function of time. Solid
        line, native enzyme in \SI{20}{\milli\Molar} phosphate (pH 6.0); dotted
        line, enzyme in the presence of \SI{0.2}{\Molar} \ce{K2SO4} ; dashed
        line, enzyme in the presence of \SI{0.6}{\Molar} \ce{NaCl}. Thermal
        inactivation of enzyme was followed in the temperature range of
    \SIrange{25}{90}{\celsius}.]{Effect of salts on the thermal unfolding of
        glucose oxidase. Ellipticity at (A) 274 nm and (B) 222 nm as a function
        of time. Solid line, native enzyme in \SI{20}{\milli\Molar} phosphate
        (pH 6.0); dotted line, enzyme in the presence of \SI{0.2}{\Molar}
        \ce{K2SO4} ; dashed line, enzyme in the presence of \SI{0.6}{\Molar}
        \ce{NaCl}. Thermal inactivation of enzyme was followed in the
        temperature range of \SIrange{25}{90}{\celsius} \cite{Gouda2003}.}
    \label{fig:heat-inactivation-improve} 
\end{figure}
% --------------------------

$\beta$-galactosidase (\iupac{\chembeta-\D-galactohydrolase}; EC 3.2.1.23)
possesses an enzymatic half-life at \SI{60}{\celsius} of 0.5 hour
\cite{Melchers1970,Chen2008}. To generate hyper-thermostabilized
$\beta$-galactosidase, Xiong \latin{et al.} have identified a variant YG6762 via
directed evolution. Using \emph{P. woesei} $\beta$-galactosidase gene as a
template, DNA shuffling is performed to generate a library. After
transformation into \emph{E. coli} and incubation, colonies have been absorbed onto
a nitrocellulose filter and transferred to a Petri dish
\cite{Xiong2007,Xiong2007a}. The filter papers are then incubated with
\iupac{5-bromo-4-chloro-3-indolyl-\chembeta-\D-glucuronic acid} (X-GlcA) for
screening. One colony, YG6762, among 200,000 colonies after five rounds of
screening exhibits the fastest rate of reaction with X-GlcA at
\SI{37}{\celsius}. With mutations, T29A, V213I, L217M, N277H, I387V, E414D,
R491C, and N496D, YG6762 also demonstrates greater activity on its substrate,
\iupac{pNP-\chembeta-\D-glucuronide (pNPG)}, than the wild-type enzyme (YH4502)
from \SI{25}{\celsius} to \SI{80}{\celsius} \cite{Xiong2007}. Notably, the
YG6762 variant exhibits nearly 50\% higher activity at \SI{37}{\celsius}
(Figure \ref{fig:yg6762}).
% --------------------------
\begin{figure}[htbp] \centering \includegraphics[width=0.7\textwidth]{fig3_20}
    \caption[The
    relative activity of purified mutant YG6762 and wild-type YH4502 enzymes at
    various temperatures. $\beta$-Glucuronidase activity was measured using
    \iupac{pNP-\chembeta-\D-glucuronide} (pNPG) as a substrate. With pNPG as a
    substrate and testing at pH 7.0, the specific activities of the variant
    YG6762 are approximately 3.13 U/mg at \SI{25}{\celsius}, 11.96 U/mg at
    \SI{37}{\celsius}, 61.95 U/mg at \SI{90}{\celsius}, and 55.72 U/mg at
    \SI{100}{\celsius}. YH4502 are approximately 0.42 U/mg at
    \SI{25}{\celsius}, 2.43 U/mg at \SI{37}{\celsius}, 60.32 U/mg at
    \SI{90}{\celsius}, and 59.45 U/mg at \SI{100}{\celsius}. All relative
    activity measurements are normalized to each activity at
\SI{90}{\celsius}.] {The relative activity of purified mutant YG6762 and
    wild-type YH4502 enzymes at various temperatures. $\beta$-Glucuronidase
    activity was measured using \iupac{pNP-\chembeta-\D-glucuronide} (pNPG) as
    a substrate. With pNPG as a substrate and testing at pH 7.0, the specific
    activities of the variant YG6762 are approximately 3.13 U/mg at
    \SI{25}{\celsius}, 11.96 U/mg at \SI{37}{\celsius}, 61.95 U/mg at
    \SI{90}{\celsius}, and 55.72 U/mg at \SI{100}{\celsius}. YH4502 are
    approximately 0.42 U/mg at \SI{25}{\celsius}, 2.43 U/mg at
    \SI{37}{\celsius}, 60.32 U/mg at \SI{90}{\celsius}, and 59.45 U/mg at
    \SI{100}{\celsius}. All relative activity measurements are normalized to
    each activity at \SI{90}{\celsius} \cite{Xiong2007}.} \label{fig:yg6762}
\end{figure}
% --------------------------

Xylanase (XynA, $\beta$-1,4-xylan xylanohydrolases; EC 3.2.1.8), a recombinant
enzyme from \emph{B. subtilis}, catalyzes the hydrolysis of $\beta$-1,4-xylan
\cite{Milessi2015,Manrich2010}. Using the lysine residues or N-terminal end of
XynA, the enzyme has been immobilized on agarose support activated with glyoxal
groups \cite{Blanco1991}. Milessi \latin{et al.} have demonstrated an 8600-fold
increase of the thermal stability via the immobilization of glyoxal-agarose
(\SI{56}{\celsius}, pH 5.5) \cite{Milessi2015} (Figure \ref{fig:xyna}).
% --------------------------
\begin{figure}[htbp] \centering \includegraphics[width=0.8\textwidth]{fig3_31}
    \caption[Inactivation courses of XynA soluble and immobilized in different
        supports. Inactivation was performed at \SI{56}{\celsius} in 50 mM citrate buffer pH
        5.5. (A) Soluble XynA (initial activity = 193 $\pm$ 5.9 UI/mL); (B)
        agarose-glyoxal-XynA (initial activity = 44.84 $\pm$ 3.7 UI/g
    gel)]{Inactivation courses of XynA soluble and immobilized in different
        supports. Inactivation was performed at \SI{56}{\celsius} in 50 mM citrate buffer pH
        5.5. (A) Soluble XynA (initial activity = 193 $\pm$ 5.9 UI/mL); (B)
    agarose-glyoxal-XynA (initial activity = 44.84 $\pm$ 3.7 UI/g gel)}
    \label{fig:xyna}
\end{figure}
% --------------------------

\subsubsection{Proteolysis}

Proteolytic resistance of a protein limits its application
\cite{Ottesen1967,Daniel1982,Fontana2004}. As proteolytic enzymes break down
protein by cleaving peptide bonds, a frequent used strategy is to add inhibitor
agents \cite{Ryan2013,Rawlings2012}. For example, in the presence of a serine
protease, phenylmethanesulfonyl fluoride (PMSF) is added to react with its
active-site \cite{Ryan2013,James1978,Gold1965}. The active site of the protease
consists of a serine, a histidine, and an aspartate residue. During a
proteolysis reaction, the histidine residue deprotonates the serine hydroxyl
group, rendering the serine side chain a nucleophile \cite{James1978,Gold1965}.
Gold \latin{et al.} have demonstrated that the sulfonyl fluorides acted as
inhibitors on the serine residue by using phenylmethanesulfonyl
$\alpha$-chymotrypsin with 2-mercaptoethylamine in \SI{8}{\Molar} urea solution
at pH 8.0, leading to a product of S-aminoethyl-cysteine \cite{Gold1965}
(Figure \ref{fig:pmsf}). 
% --------------------------
\begin{figure}[htbp] \centering \includegraphics[width=0.6\textwidth]{fig3_32}
    \caption[Interaction of PMSF (phenylmethylsulfonyl fluoride) with the
    active-site serine hydroxyl group of a serine proteinase.]{Interaction of
        PMSF (phenylmethylsulfonyl fluoride) with the active-site serine
        hydroxyl group of a serine proteinase \cite{James1978,Gold1965}.}
    \label{fig:pmsf}
\end{figure}
% --------------------------

Investigations into relationship between protein stability and proteolytic
resistance reveals that the resistance is dependent on stability of the protein
\cite{Daniel1982, Parsell1989}. For example, \latin{B. subtilis} lipase has
been developed for the synthesis of biopolymers and biodiesel, and the
production of enantiopure pharmaceuticals \cite{Jaeger2002,Ahmad2012}. The
protease resistance of lipase at \SI{37}{\celsius} in \SI{100}{\milli\Molar}
sodium phosphate buffer (pH 7.2) is approximately 55 minutes \cite{Ahmad2012}.
In addition, upon inactivation by protease, residual activity of wild-type
lipase is 55.2\% after four hours incubation at \SI{37}{\celsius}
\cite{Ahmad2012} (Figure \ref{fig:protease-resistance}). To increase both
stability and protease resistance of lipase, Ahmad \latin{et al.} have
generated a library of lipase mutants through site-saturated mutagenesis
(Figure \ref{fig:lipase}).  While loops and surface regions of proteins display
conformational dynamics \cite{Kamerzell2008}, they have produced a collection
of site-saturated mutagenesis to compare unfolding profiles and residual
activities of wild-type and mutants (Figure \ref{fig:protease-resistance}). By
introducing the mutation M137P, the lipase T\textsubscript{m} is raised by
\SI{6.8}{\celsius}, and the protease resistance is increased to 955 minutes at
\SI{37}{\celsius} in \SI{100}{\milli\Molar} sodium phosphate buffer (pH 7.2).
The residual activity is also elevated to 92.5\% \cite{Ahmad2012} after four
hours incubation at \SI{37}{\celsius} (Figure \ref{fig:protease-resistance}).
Figure \ref{fig:protease-resistance} and Figure \ref{fig:lipase} illustrate
that 16 out of 17 single mutants in the loop region of lipase demonstrate
improved resistance to proteolysis.  This example highlights the role of loops
in protein stability as well as proteolytic resistance via mutagenesis.
% --------------------------
\begin{figure}[htbp] \centering \includegraphics[width=0.5\textwidth]{fig3_14}
    \caption[Proteolytic resistance of wild-type lipase and its variants
        measured as post proteolysis residual activity, after incubation with
        subtilisin A. Mutations listed in the figure have been generated by
        site-saturation mutagenesis. Highlighted is M137P variant that exhibits
        enhanced protease resistance.]{Proteolytic resistance of wild-type
            lipase and its variants measured as post proteolysis residual
            activity, after incubation with subtilisin A. Mutations listed in
            the figure have been generated by site-saturation mutagenesis.
            Highlighted is M137P variant that exhibits enhanced protease
            resistance \cite{Ahmad2012}.}
    \label{fig:protease-resistance}
\end{figure}
% --------------------------
% --------------------------
\begin{figure}[htbp] \centering \includegraphics[width=0.5\textwidth]{fig3_25}
    \caption[Ribbon diagram of \emph{Bacillus subtilis} lipase, depicting 15
    thermostabilizing positions, along with corresponding preferred
substitutions.] {Ribbon diagram of 
    \emph{Bacillus subtilis} lipase, depicting 15 thermostabilizing positions,
    along with corresponding preferred substitutions \cite{Ahmad2012}.}
    \label{fig:lipase}
\end{figure}
% --------------------------

Using 3-aminopropyltriethoxysilane (APTES) , Sarkar \latin{et al.}
immobilized laccase (\emph{p}-diphenol oxidase, EC 1.10.3.2) onto silt loam
soil (pH 6.4; silt 59\%; sand 17\%; clay 24\%; and organic matter 3.2\%)
\cite{Sarkar1989,Vashist2014}. About 50 units of free and immobilized enzymes
were incubated separately with protease in the presence of
\SI{50}{\milli\Molar} Tris-HCI buffer (pH 8.0) at \SI{37}{\celsius}. In the
presence of protease, approximately 75\% decrease of the activity of free
lactase was measured after a 5-hour exposure to protease (Figure
\ref{fig:clay}). Nearly 5\% of its activity remained after a 12-hour
incubation.  In contrast, immobilized laccase demonstrates retained 65\% and
37\% of its original activity after 5 and 12 hours of protease exposure,
respectively (Figure \ref{fig:clay}).
% --------------------------
\begin{figure}[htbp] \centering \includegraphics[width=0.5\textwidth]{fig3_33}
    \caption[Effect of protease on the activity of various free and immobilized
    enzymes.]{Effect of protease on the activity of various free and immobilized
        enzymes \cite{Sarkar1989}.} \label{fig:clay}
\end{figure}
% --------------------------

\subsubsection{Organic Solvent Inactivation}
\label{sec:osi}

One limitation for the use of enzymes is instability under processing
conditions, such as organic solvent systems \cite{Stepankova2013}.
Although water serves as an ideal solvent, it is a rather poor solvent for
synthetic reactions \cite{Serdakowski2008}. However, enzymes become inactive at
an organic co-solvent concentration of 60\% to 70\% \cite{Stepankova2013}. For
example, Stepankova \latin{et al.} have investigated the co-solvent effects on
haloalkane dehalogenases, DbjA, DhaA, and LinB \cite{Koudelakova2013}. These
enzyme catalyze the hydrolytic cleavage of the carbon-halogen bonds of
halogenated aliphatic compounds, and have been engineered for degradation of
environmental pollutants \cite{Stepankova2013a,Koudelakova2013} (Figure
\ref{fig:hld}). They have demonstrated that addition of high concentrations of
organic solvents such as dimethylformamide, dimethyl sulfoxide and
tetrahydrofuran, results in a loss of $\alpha$-helical structure in DbjA and
DhaA \cite{Stepankova2013a} (Figure \ref{fig:organic-effect}). While a decrease
in activity of Dha A has been observed with increasing co-solvent
concentration,  LinB is inactivated by most co-solvents even at low
concentrations (less than 10\%).
% --------------------------
\begin{figure}[htbp] \centering \includegraphics[width=0.5\textwidth]{fig3_16} 
    \caption[Simplified scheme of the reaction mechanism of haloalkane
    dehalogenase. Hydrolytic cleavage of a carbon-halogen bond proceeds by the
S\textsubscript{N}2, followed by the addition of water. Water is the only
co-factor required for catalysis]{Simplified scheme of the reaction mechanism
    of haloalkane dehalogenase. Hydrolytic cleavage of a carbon-halogen bond
    proceeds by the S\textsubscript{N}2, followed by the addition of water.
    Water is the only co-factor required for catalysis \cite{Koudelakova2013}.}
    \label{fig:hld}
\end{figure}
% --------------------------
% --------------------------
\begin{figure}[htbp] \centering \includegraphics[width=0.5\textwidth]{fig3_15} 
    \caption[Circular dichroism spectra of (A) DbjA and (B) DhaA in the
    presence of organic co-solvents. The spectra are measured at
\SI{37}{\celsius} in phosphate buffer (\SI{50}{\milli\Molar}, pH 7.5) and
various organic co-solvents at concentrations that cause reductions in enzyme
activity of more than 90\%. DMF, dimethylformamide; DMSO, dimethyl sulfoxide;
THF, tetrahydrofuran.]{Circular dichroism spectra of (A) DbjA and (B) DhaA in
    the presence of organic co-solvents. The spectra are measured at
    \SI{37}{\celsius} in phosphate buffer (\SI{50}{\milli\Molar}, pH 7.5) and
    various organic co-solvents at concentrations that cause reductions in
    enzyme activity of more than 90\%. DMF, dimethylformamide; DMSO, dimethyl
    sulfoxide; THF, tetrahydrofuran \cite{Stepankova2013a}.}
    \label{fig:organic-effect} 
\end{figure}
% --------------------------

While the ionization in organic solvents can be optimized, using buffer pairs
of acids and conjugated bases increases the enzymatic activity
\cite{Klibanov2001,Blackwood1994}. Blackwood \latin{et al.} have demonstrated
that the combinations of triisooctylamine with its hydrochloride salt can be
used for catalysis in organic media \cite{Blackwood1994} (Figure
\ref{fig:organic-salt}). Triisooctylamine is first dissolved in pentan-3-one,
and then shaken with water to saturate it. To convert the solution to the
hydrochloride form, it is prepared with an equal volume of \SI{0.1}{\Molar}
aqueous \ch{HC1}. \emph{Rhizomucor miehei} lipase activity is performed with a
range of pH of triisooctylamine buffer (Figure \ref{fig:organic-salt}), showing
the reaction rate is dependent on the ratio of buffer species. At the ratio of
90\% of base in buffer, lipase demonstrates a 2-fold increase of activity
relative to 0\%.
% --------------------------
\begin{figure}[htbp] \centering \includegraphics[width=0.5\textwidth]{fig3_34}
    \caption[Activity of lipase as a function of organic phase buffer ratio.
        Activity was determined in pentanone in the presence of the
        triisooctylamine buffer system, using the percentage of free base
        shown. The rate shown is that of consumption of ethyl
    decanoate.]{Activity of lipase as a function of organic phase buffer ratio.
        Activity was determined in pentanone in the presence of the
        triisooctylamine buffer system, using the percentage of free base
        shown. The rate shown is that of consumption of ethyl decanoate
        \cite{Blackwood1994}.} \label{fig:organic-salt}
\end{figure}
% --------------------------

Subtilisin E represents an another example where the enzyme is inactivated due
to organic solvent \cite{You1996}. In the presence of 20\% of dimethylformamide
(DMF), the wild-type enzyme exhibits only less than 10\% of activity of that in
the absence of DMF \cite{You1996}. To enhance to catalytic activity of
subtilisin E, the Arnold group have developed an error-prone PCR strategy to
generate variants \cite{You1996}. An agar plate used for screening is covered
with two membranes; one organic solvent-resistant Biodyne nylon membrane and
one nitrocellulose membrane. After incubation, the resulting Biodyne nylon
membrane is then transferred to 35-45\% DMF for evaluation of activity on 1\%
casein. After three generations of random mutagenesis, You \latin{et al.}
identifies that the variant 13M (A48R, D97G, Q103R, G131D, E156G, N181S, S182G,
S188P, Q206L, T255A) which exhibits 10-fold greater activity than wild-type in
the absence of DMF. The specific activity of 13M is further improved by
16-fold in the presence of 20\% DMF \cite{You1996} (Figure \ref{fig:arnold}). 
% --------------------------
\begin{figure}[htbp] \centering \includegraphics[width=0.7\textwidth]{fig3_22}
    \caption[Comparison of the catalytic efficiencies of wild type and 13M
    subtilisin E towards hydrolysis of succinyl-Ala-Ala-Pro-Phe-p-nitroanilide
    (s-AAPF-pNa) in \SI{0.1}{\Molar} Tris-HCl, pH 8.0,
    \SI{10}{\milli\Molar}{\ce{CaCl2}} with varying concentration of
    dimethylformamide (DMF), at \SI{37}{\celsius}.]{Comparison of the catalytic
        efficiencies of wild type and 13M subtilisin E towards hydrolysis of
        succinyl-Ala-Ala-Pro-Phe-p-nitroanilide (s-AAPF-pNa) in
        \SI{0.1}{\Molar} Tris-HCl, pH 8.0, \SI{10}{\milli\Molar}{\ce{CaCl2}}
        with varying concentration of dimethylformamide (DMF), at
        \SI{37}{\celsius}  \cite{You1996}.} \label{fig:arnold}
\end{figure}
% --------------------------

To address the issue of organic solvent, Dravis \latin{et al.} have developed
strategies to stabilize enzymes in organic solvents \cite{Dravis2001}. The use
of immobilized enzymes represents the most common method to improve enzyme
stability toward organic solvents \cite{Koudelakova2013,Dravis2001}. The
enzyme, haloalkane dehalogenase, is coupled to an inorganic polyethyleneimine
impregnated alumina (PEI-alumina) with a glutaraldehyde linker (Figure
\ref{fig:pei}). The results demonstrate that the soluble dehalogenase is
completely inactivated (half-life is less than 10 minutes) in the presence of
\SI{12}{\milli\Molar} 1,2,3-trichloropropane (TCP). By contrast, the
immobilized enzyme exhibits activity in the pure TCP solvent, demonstrating a
half-life of nearly 10 hours \cite{Dravis2001}.
% --------------------------
\begin{figure}[htbp] \centering \includegraphics[width=0.7\textwidth]{fig3_21}
    \caption[Stabilization of polyethyleneimine(PEI)/enzyme composites via
    glutaraldehyde crosslinking to prevent subunit dissociation.]{Stabilization
    of polyethyleneimine (PEI)/enzyme composites via glutaraldehyde
    crosslinking to prevent subunit dissociation \cite{Barbosa2014}.}
    \label{fig:pei}
\end{figure}
% --------------------------

\subsection{Stabilization of Proteins in Solid State}

The stability of enzymes presents challenges for use in non-biological
environments \cite{Carpenter1993,Taylor2010}. Elevated temperatures
\cite{Rupley1991} or organic solvents \cite{Stepankova2013} may result in
protein denaturation or aggregation. While many enzymes do not anticipate
long-term stability in solution, a solid-state strategy, such as freeze-drying
(lyophilization) \cite{Carpenter1993,Luthra2007}, has been
developed \cite{Taylor2010}. Lyophilization involves three steps, including
freezing, primary-drying, and secondary-drying \cite{Luthra2007}. In the
freezing stage, the solution is frozen at a temperature below the desired
drying temperature. Primary-drying stage is where the majority of frozen water
is removed by sublimation. Afterwards, the secondary-drying removes the rest of
the water at higher temperature by desorption \cite{Luthra2007,Griebenow1995}.
The Klibanov group have demonstrated that lyophilization of bovine pancreatic
trypsin inhibitor (BPTI) in water (pH is adjusted to 3.5 using \ch{NaOH})
increased $\beta$-sheet with a decrease in $\alpha$-helix content via Fourier
transform infrared (FTIR) spectroscopy. Notably, the content of unordered
structure decreased upon the lyophilization, providing stability to protein
\cite{Griebenow1995} (Figure \ref{fig:lyo}). 
% --------------------------
\begin{figure}[htbp] \centering \includegraphics[width=0.7\textwidth]{fig3_35}
    \caption[FTIR spectra of Bovine pancreatic trypsin inhibitor (BPTI) in (A)
    aqueous solution at pH 3.5 and (B) the powder lyophilized from that
solution after the Gaussian curve-fitting process. The results of the added
Gaussian bands and the original spectra (solid lines) are superimposed and are
nearly identical. The area of the individual Gaussian bands (broken lines) has
been used to calculate the secondary structure content. The individual bands
were assigned as follows: (A) a, b, and c, $\alpha$-helix; d and e, unordered;
f and g, $beta$-sheet; (B) a, $alpha$-helix; b and c, unordered; d and e,
$beta$-sheet. The bands at \SI{1205}{\per\cm}  are not an amide III vibration
and are presented solely for the fit. The band at around \SI{1340}{\per\cm} in
the spectrum of the powder (B), which is of an unknown origin and not found in
the BPTI spectrum in aqueous solution, was not assigned to any secondary
structural element.]{FTIR spectra of Bovine pancreatic trypsin inhibitor (BPTI)
    in (A) aqueous solution at pH 3.5 and (B) the powder lyophilized from that
    solution after the Gaussian curve-fitting process. The results of the added
    Gaussian bands and the original spectra (solid lines) are superimposed and
    are nearly identical. The area of the individual Gaussian bands (broken
    lines) has been used to calculate the secondary structure content. The
    individual bands were assigned as follows: (A) a, b, and c, $\alpha$-helix;
    d and e, unordered; f and g, $beta$-sheet; (B) a, $alpha$-helix; b and c,
    unordered; d and e, $beta$-sheet. The bands at \SI{1205}{\per\cm}  are not
    an amide III vibration and are presented solely for the fit. The band at
    around \SI{1340}{\per\cm} in the spectrum of the powder (B), which is of an
unknown origin and not found in the BPTI spectrum in aqueous solution, was not
assigned to any secondary structural element \cite{Griebenow1995}.}
    \label{fig:lyo}
\end{figure}
% --------------------------

However, solid-sate stabilization methods generate stresses to proteins, such
as lowering of temperature and formation of substantial ice-aqueous interface
\cite{Luthra2007}, which degrades proteins during processes \cite{Taylor2010}.
For example, Carpenter \latin{et al.} have demonstrated that both freeze-drying
and freeze-thawing processes greatly impact activity recovery of
phosphofructokinase (PFK) \cite{Carpenter1993}.  Without any additive, PFK
recovers less than 10 \% of activity \cite{Carpenter1993} (Figure
\ref{fig:pfk}). To resolve the stabilization dilemma from solid state, proteins
are formulated with additives to improve their stability.  Polyethylene
glycol (PEG) results in significant recovery of activity (Figure
\ref{fig:pfk}).  The concentration of 1\% PEG completely protects the
catalytic activity of PFK during the freeze-thawing process.  Moreover,
Carpenter \latin{et al.} have demonstrated that the combination of the sugar,
glucose, and PEG further improves the stability of freeze-drying PFK (Figure
\ref{fig:pfk2}). The activity recovery from the freeze-drying process is raised
to 70 \% via the presence of \SI{25}{\milli\Molar} glucose and 1\% PEG as
additives. 
% --------------------------
\begin{figure}[htbp] \centering \includegraphics[width=0.5\textwidth]{fig3_23}
    \caption[Effect of polyethylene glycol (PEG) on phosphofructokinase (PFK) stability
    during freeze-thawing (circle) and freeze-drying (triangle).] {Effect of
        polyethylene glycerol (PEG) on phosphofructokinase (PFK) stability during
        freeze-thawing (circle) and freeze-drying (triangle)
        \cite{Carpenter1993}.} 
    \label{fig:pfk}
\end{figure}
% --------------------------
% --------------------------
\begin{figure}[htbp] \centering \includegraphics[width=0.5\textwidth]{fig3_24}
    \caption[Comparison of influence of glucose alone (circle) and glucose with
    1\% PEG (square) on stability of freeze-dried PFK.]{Comparison of influence
        of glucose alone (circle) and glucose with 1\% PEG (square) on
        stability of freeze-dried PFK \cite{Carpenter1993}.} \label{fig:pfk2}
    \end{figure}
% --------------------------

\subsection{Stabilization of Proteins via Entrapment}

Other methodologies have been developed to stabilize enzymes inside the
microorganisms, such as entrapment
\cite{Trelles2013,Bhosale1996,Etzel1996,Suekane1982}. Entrapment is used in
industrial scale enzyme operations, such as high fructose corn syrup synthesis.
As enzymes of interest are intracellular, it is easier to use whole-cell
entrapment where the enzyme is synthesized in bacteria \cite{Suekane1982}. One
example of this is glucose isomerase (GI, EC 5.3.1.5), which has been used in
the commercial production of high fructose corn syrup (HFCS)
\cite{Bhosale1996}. \emph{Actinoplanes missouriensis} cells with GI are
occluded in gelatin followed by treatment with glutaraldehyde \cite{Cfibwjps}
(Figure \ref{fig:enzyme-entrapment}).  More than 6 millions tons of HFCS per
year is produced by GI.  Compared with chemical isomerization, using the enzyme
not only increases the purity of the fructose product \cite{Barker1975}, but
also enhances sweetness \cite{Bhosale1996}. To save the cost of isolating GI,
whole-cell entrapment has been employed to adapt to the increased demand of
this enzyme \cite{Suekane1982}.  
% --------------------------
\begin{figure}[htbp] \centering \includegraphics[width=0.9\textwidth]{fig3_39}
    \caption[Schiff base reaction of glutaraldehyde with proteins.]{Schiff base
        reaction of glutaraldehyde with proteins. \cite{Cfibwjps}}.
        \label{fig:enzyme-entrapment} 
\end{figure}
% --------------------------

\subsection{PTE and Its Applications}

Conventional methods, including incineration or neutralization
\cite{Ghanem2005a,Porzio2007}, are currently used to decontaminate
organophosphates (OPs).  For example, the OP liquid tabun (GA, G series, first
synthesized) is burned in a furnace of temperatures over \SI{1093}{\celsius}
\cite{Afriat-Jurnou2012}.  In addition, VX nerve agent is neutralized via
reaction with sodium hydroxide \cite{Afriat-Jurnou2012}. The hydrolysis is
initiated by the nucleophilic attack of the hydroxide ion on the phosphorus
groups. As high temperatures and treatment with sodium hydroxide may damage
objects applied, an enzyme-based decontamination would provide an alternative
solution during the decontamination process \cite{Defrank}.  Phosphotriesterase
(PTE) is a dimeric protein that hydrolyzes a wide range of OPs \cite{
Lewis1988,Chen2007a,Mulbry1989,Benning2001a,Omburo1992a,Benning1995,Naqvi2014}
(Chapter \ref{chap:dimer}).  PTE has been demonstrated to neutralize OPs, such
as pesticides and chemical weapons
\cite{Chen-Goodspeed2001a,Hanusa2011,Aubert2004b,Chen-Goodspeed2001a,Bigley2013}
(Section \ref{sec:pte-intro}), rendering it an excellent agent for
detoxification. While PTE can catalyze OPs, it is unstable
\cite{Rochu2002b,LeJeune1997a,Tsai2012b,Defrank}.  Thus, approaches to
stabilize PTE for prolonged use outside the biological confines of the cell
have been developed
\cite{Rochu2002b,LeJeune1997a,Tsai2012b,Defrank,Yang2014a,Baker2011b}. 

\subsubsection{PTE Heat Inactivation}

In general, PTE becomes heat inactivated in solution. Rochu \latin{et al.}
have demonstrated the heat-induced unfolding of PTE in
\SI{200}{\milli\Molar} borate (pH 9.4) \cite{Rochu2002b} (Figure
\ref{fig:pte-thermo-inactive}). Upon elevated temperatures, PTE undergoes
irreversible denaturation, leading to aggregates at high
temperatures with an apparent T\textsubscript{m} of roughly \SI{75}{\celsius}
using DSC \cite{Rochu2002b}.  The residual activity assays also demonstrate
that above \SI{46}{\celsius}, a rapid loss of activity occurs with complete
inactivation at \SI{60}{\celsius} (Figure \ref{fig:pte-thermo-inactive}).
% --------------------------pte-heat-inactivation
\begin{figure}[h!] \centering \includegraphics[width=0.7\textwidth]{fig3_17}
    \caption[Effect of heat treatment on the catalytic activity of PTE in
    \SI{200}{\milli\Molar} borate (pH 9.4). Samples were heated in a water bath
for 15 min at the desired temperature, then cooled to \SI{25}{\celsius} and the
PTE activity immediately measured. This validates the reality of the zig-zag
plot with an activation phase near \SI{46}{\celsius}. The inset depicts the
thermal inactivation profile of the enzyme as a function of time at
\SI{60}{\celsius}. The percent residual activity is shown for treated
(triangle) or without (square) previous incubation for 15 min at
\SI{60}{\celsius}.]{Effect of heat treatment on the catalytic activity of PTE
    in \SI{200}{\milli\Molar} borate (pH 9.4). Samples were heated in a water
    bath for 15 min at the desired temperature, then cooled to
    \SI{25}{\celsius} and the PTE activity immediately measured. This validates
    the reality of the zig-zag plot with an activation phase near
    \SI{46}{\celsius}. The inset depicts the thermal inactivation profile of
    the enzyme as a function of time at \SI{60}{\celsius}. The percent
    residual activity is shown for treated (triangle) or without (square)
    previous incubation for 15 min at \SI{60}{\celsius} \cite{Rochu2002b}.}
    \label{fig:pte-thermo-inactive} 
\end{figure}
% --------------------------

\subsubsection{PTE Organic Solvent Inactivation}

Organic solvent negatively affects PTE stability and
activity \cite{LeJeune1997a}. For example, LeJeune \latin{et al.} have
investigated the inactivation of PTE in the presence of 0 to 50\% dimethyl
sulfoxide (DMSO) \cite{LeJeune1997a} (Figure \ref{fig:pte-organic-inactive}).
After even 10\% DMSO is added, PTE activity decreases dramatically (Figure
\ref{fig:pte-organic-inactive}). They attribute this phenomenon to
environmental dielectric changes \cite{LeJeune1997a}. The results also
demonstrate that PTE with DMSO concentrations above 40\% exhibits substantial
irreversible effects \cite{LeJeune1997a} (Figure
\ref{fig:pte-organic-inactive}).
% --------------------------pte-organic
\begin{figure}[h!] \centering \includegraphics[width=0.5\textwidth]{fig3_18}
    \caption[Dependence of soluble PTE activity in the
    presence of DMSO during enzyme assay. Assay conditions were
\SI{0.5}{\milli\Molar} paraoxon at \SI{25}{\celsius}, PTE concentration was 0.9
pM.]{Dependence of soluble PTE activity in the presence of DMSO during enzyme
    assay. Assay conditions were \SI{0.5}{\milli\Molar} paraoxon at
    \SI{25}{\celsius}, PTE concentration was 0.9 pM \cite{LeJeune1997a}.}
    \label{fig:pte-organic-inactive}
\end{figure}
% --------------------------

\subsubsection{Approaches to Stabilize PTE}

To decontaminate OPs using PTE, strategies including mutagenesis and
immobilization have been developed to address issues described above. In the
previous two chapters, discussion focused on unnatural amino acid
incorporation and computational modeling for PTE stabilization. Here, two
alternative methods are provided as follows. 

\paragraph{Altering the Surface Residues with the Active Site}
Chen and coworkers have demonstrated surface residues improve catalytic
efficiency for paraoxon, parathion, and chlorpyrifos, through mutations
K185R/I274N in PTE \cite{Mee-HieCho2006a}. The combination of two mutations
introduce the hydrogen bond The E219 or E181, leading to an improved activity
\cite{Mee-HieCho2006a}. Raushel group have employed site-directed mutagenesis
on the PTE surface in combination with active site mutations for stabilization
\cite{Tsai2012b}.  With the introduction of four mutations (Figure
\ref{fig:qfrn}), H254Q, H257F, K185R, and I274N, mutant QFRN exhibits an 8 to
50 fold increase in k\textsubscript{cat}/K\textsubscript{M} for VX and sarin
analogs, relative to wild-type PTE \cite{Tsai2012b}. The X-ray crystal
structure of QFRN shows additional hydrogen bonding interactions from the K185R
residue to S218, leading to PTE stabilization \cite{Tsai2012b}. 
% --------------------------
\begin{figure}[htbp] \centering \includegraphics[width=0.8\textwidth]{fig3_26}
    \caption[(A) Location of the mutations in QFRN (H254Q, H257F, K185R,
and I274N). The mutated residues are colored green. (B) The additional hydrogen
bond at K185R. The parent protein is colored yellow, and the mutant is colored
in cyan.] {(A) Location of the mutations in QFRN (H254Q, H257F, K185R, and
    I274N). The mutated residues are colored green. (B) The additional hydrogen
    bond at K185R. The parent protein is colored yellow, and the mutant is
    colored in cyan \cite{Tsai2012b}.} \label{fig:qfrn}
\end{figure}
% --------------------------

\paragraph{Immobilization onto Solid Support}
To stabilize PTE, Caldwell \latin{et al.} have immobilized it on a trityl
agarose resin \cite{Caldwell1991} (Figure \ref{fig:pump}). This approach
involves noncovalent binding between hydrophobic residues of enzymes to
tritylated agarose fibers (Figure \ref{fig:tri}). Cashine \latin{et al.}
previously employed tritylated agarose because of the high coupling properties
and high retention of activity upon immobilization of enzymes
\cite{Cashion1982}. 
% --------------------------
\begin{figure}[htbp] \centering \includegraphics[width=0.5\textwidth]{fig3_36}
    \caption[Trityl disaccharide repeat unit of tritylagarose: 3\rq linked,
    6\rq-trityl-D-galactose joined to 3\rq-6\rq-anhydro-$\alpha$-L-galactose by
a $\beta$-1-4 bond.]{Trityl disaccharide repeat unit of tritylagarose: 3\rq
linked, 6\rq-trityl-D-galactose joined to
3\rq-6\rq-anhydro-$\alpha$-L-galactose by a $\beta$-1-4 bond.} \label{fig:tri}
\end{figure}
% --------------------------

PTE is immobilized onto the tritylated agarose fibers, and the mixture is
poured into a reactor\cite{Caldwell1991} (Figure \ref{fig:tri}, \ref{fig:pump} C).
The enzyme activity is determined by monitoring the absorbance of
\emph{p}-nitrophenol with a Gilford 260 spectrophotometer (Figure
\ref{fig:pump} D). The effect of organic solvent on the affinity of the
immobilized enzyme for the trityl agarose matrix is determined by washing
reactors with organic solvent (10\% methanol).
% --------------------------
\begin{figure}[htbp] \centering \includegraphics[width=0.5\textwidth]{fig3_40}
    \caption[A general schematic for the experimental system. (A) Substrate
        reservoir. (B) Pump. (C) Fixed bed reactor (tritylated agarose fibers
        with PTE). (D) Gilson Model HM Holochrome UV-visible detector. (E)
        Effluent
collector]{A general schematic for the experimental system. (A) Substrate
    reservoir. (B) Pump. (C) Fixed bed reactor (tritylated agarose fibers with
    PTE). (D) Gilson Model HM Holochrome UV-visible detector. (E) Effluent
collector} \label{fig:pump}
\end{figure}
% --------------------------

A correlation between the flow rate, paraoxon concentration
and the percentage of hydrolysis is observed (Figure \ref{fig:pte-app-imm}).
Approximately 50\% hydrolysis of paraoxon is achieved with flow rate of
\SI{11.5}{\mL\per\hour} (Figure \ref{fig:pte-app-imm}). Experiments have been
also conducted at the 1.0 and 10-unit reactors. While an improvement of
effective hydrolysis is observed at 1.0 and 10-unit reactors, the presence of
organic solvent, dissociates the enzyme from the reactor, limiting its
application \cite{Cashion1982}.
% --------------------------pte-app
\begin{figure}[htbp] \centering \includegraphics[width=0.7\textwidth]{fig3_19}
    \caption[The hydrolysis of paraoxon in \SI{125}{\milli\Molar}
        2-(N-cyclohexylamino)ethanesulfonic acid (CHES), 10\% methanol at
    varied flow rates with an immobilized phosphotriesterase fixed reactor (0.1
unit) (triangle, \SI{0.92}{\milli\Molar}, closed circle,
\SI{0.1}{\milli\Molar}, open circle, \SI{0.046}{\milli\Molar}); (1.0 unit)
(closed square, \SI{0.17}{\milli\Molar}, open square,
\SI{0.087}{\milli\Molar}); (10.0 unit) (Diamond, \SI{0.97}{\milli\Molar}).]{The
    hydrolysis of paraoxon in \SI{125}{\milli\Molar} 2-(N-cyclohexylamino)
    ethanesulfonic acid (CHES), 10\% methanol at varied flow rates with an
    immobilized phosphotriesterase fixed reactor (0.1 unit) (triangle,
    \SI{0.92}{\milli\Molar}, closed circle, \SI{0.1}{\milli\Molar}, open
    circle, \SI{0.046}{\milli\Molar}); (1.0 unit) (closed square,
    \SI{0.17}{\milli\Molar}, open square, \SI{0.087}{\milli\Molar}); (10.0
    unit) (Diamond, \SI{0.97}{\milli\Molar}) \cite{Caldwell1991}.}
    \label{fig:pte-app-imm}
\end{figure}
% --------------------------

\paragraph{Foam and Microemulsion}
Being catalytic, PTE is highly efficient and can detoxify large amounts of
organophosphates. The The North Atlantic Treaty Organization (NATO) have
demonstrated approximately 99.5 \% destruction of soman (GD) in 15 to 30
minutes \cite{Defrank} using organophosphorus acid anhydrolas (OPAA) in foam.
NATO project 31 previously conducted research on PTE formulations
\cite{Defrank}, such as foam \cite{Lejeune1996,LeJeune1997a,Cheng1996} (Figure
\ref{fig:pte-foam}) and microemulsion \cite{Komives1994} (Figure
\ref{fig:micelles}).

LeJeune \latin{et al.} have immobilized PTE with Hypol polyurethane prepolymers
\cite{LeJeune1997a,Lejeune1996} (Figure \ref{fig:foam}, \ref{fig:pte-foam}).
The reaction to synthesize foam is initiated when water is in contact with
isocyanante groups \cite{Lejeune1996} (R\textsubscript{1}, Figure
\ref{fig:foam}). Afterwards, isocyanate groups react with water to form an
unstable carbamic acid, leading to degradation of an amine and carbon dioxide.
Amines react with isocyanate groups to give a urea-type linkage. As PTE
contains multiple functional groups, such as amines and hydroxyls, the enzyme
could be immobilized onto the foam \cite{Lejeune1996} (Figure
\ref{fig:pte-foam}). Polymers are synthesized with a toluene diisocyanate
based polyurethane (prepolymer) and 1 wt \% Hypol 3000 surfactant in the
aqueous phase (1:1 of prepolymer: aqueous phase). PTE is then added
(\SI{10}{\mL}) to a buffered solution containing 1\% Pluronic P-65 surfactant
(\SI{5}{\mL}) \cite{Lejeune1996,LeJeune1997a} (Figure \ref{fig:pte-foam}). The
immobilized PTE has demonstrated a 20\% increase activity toward paraoxon
relative to PTE in buffer (\SI{0.12}{\Molar} Hepes buffer, pH 7.4,
\SI{50}{\micro\Molar} \ch{CoCl2}) after 25 days of storage at
\SI{25}{\celsius}.  They further demonstrate that the foam also stabilizes PTE
at \SI{50}{\celsius} for more than 158 hours in comparison with soluble enzyme
at \SI{50}{\celsius} for merely 1.5 hours \cite{LeJeune1997a}. 
% --------------------------
\begin{figure}[htbp] \centering \includegraphics[width=0.8\textwidth]{fig3_27}
    \caption[Schematic of occurring reactions during protein-foam synthesis. E:
    eznyme.]{Schematic of occurring reactions during protein-foam synthesis. E:
        eznyme \cite{Lejeune1996}.}
    \label{fig:foam}
\end{figure}
% --------------------------
% -------------------------- pte-foam 
\begin{figure}[htbp] \centering \includegraphics[width=0.7\textwidth]{fig3_06}
    \caption[Photographs of synthesized polymers with PTE. Polymers were
        synthesized with Hypol 3000 surfactant. A 1:1 wt ratio of prepolymer to
        aqueous phase was used with 1 wt \% surfactant in the aqueous phase.
        PTE (\SI{0.6}{\mg\per\mL}) was added (\SI{10}{\mL}) to a buffered
        aqueous solution (\SI{0.12}{\Molar} Hepes buffer, pH 7.4,
        \SI{50}{\milli\Molar}\ch{CoCl2}) containing 1 \% Pluronic P-65
        surfactant (\SI{5}{\mL}). Approximately \SI{4}{\mL} of Hypol 3000, a
        toluene diisocyanate based polyurethane prepolymer, were added to the
        PTE solution(\SI{0.6}{\mg\per\mL}).]{Photographs of synthesized
            polymers with PTE. Polymers were synthesized with Hypol 3000 surfactant. A
            1:1 wt ratio of prepolymer to aqueous phase was used with 1 wt \%
            surfactant in the aqueous phase. PTE (\SI{0.6}{\mg\per\mL}) was
            added (\SI{10}{\mL}) to a buffered aqueous solution
            (\SI{0.12}{\Molar} Hepes buffer, pH 7.4,
            \SI{50}{\milli\Molar}\ch{CoCl2}) containing 1\% Pluronic P-65
            surfactant (\SI{5}{\mL}). The surfactant is added to make foam.
            Approximately \SI{4}{\mL} of Hypol 3000, a toluene diisocyanate
            based polyurethane prepolymer, were added to the PTE
            solution(\SI{0.6}{\mg\per\mL}) \cite{LeJeune1997a}.}
            \label{fig:pte-foam}
\end{figure}
% --------------------------*

Alternatively, Komives \latin{et al.} have developed a reversed micelle system
with polyoxyethylene sorbitan trioleate (Tween 85) to stabilize PTE
\cite{Komives1994} (Figure \ref{fig:micelles}). The system is composed of
four components, including buffer (\SI{0.05}{\Molar} Tris-HC1, pH 7),
surfactant (Tween 85), oil (hexane), and alcohol (isopropanol). The catalytic
efficiency evaluated using this system have demonstrated that the half-life at
1.5 \% isopropanol, compared with that at 8 \% isopropanol, is 2-fold higher
\cite{Komives1994} (Table \ref{tab:micelle}).  The low water content appears to
have a deleterious effect on enzyme stability (See Section \ref{sec:osi}). 
% --------------------------
\begin{figure}[htbp] \centering \includegraphics[width=0.5\textwidth]{fig3_28}
    \caption[Schematic of enzyme partitioning in surfactant layer and aqueous
    pool. At low water content, the enzyme partitions in the surfactant layer
due to the small volume of the water core (left). At high water contents, the
enzyme partitions into the water pool, where it functions as in buffer
(right).] {Schematic of enzyme partitioning in surfactant layer and aqueous
    pool. At low water content, the enzyme partitions in the surfactant layer
    due to the small volume of the water core (left). At high water contents,
    the enzyme partitions into the water pool, where it functions as in buffer
    (right) \cite{Komives1994}.}
    \label{fig:micelles}
\end{figure}
% --------------------------
% --------------------------
\begin{table}[htbp]
    \centering
    \caption[Comparison of organophosphorus hydrolase stability. Enzyme
    concentration was fixed at 0.96 pg/mL, and all solutions were incubated at
\SI{30}{\celsius} in shaker.] {Comparison of organophosphorus hydrolase
    stability. Enzyme concentration was fixed at 0.96 pg/mL, and all solutions
    were incubated at \SI{30}{\celsius} in shaker.} 
    \begin{tabular}{lllll}
    \hline

    Solvent & Surfactant & Cosurfactant & Tris-HCI & Half-life (hour) \\
    \hline

    Hexane & 0.071 M Tween 85 & 1.5 vol \% isopropanol & 3 vol \%, pH 7 & 60 \\
    Hexane & 0.071 M Tween 85 & 8 vol \% isopropanol & 3 vol \%, pH 7 & 31 \\

    \hline  
    \end{tabular} 
    \label{tab:micelle}
\end{table}
% --------------------------

\subsection{Lactose Monohydrate}

Lactose (\iupac{4-\O-\chembeta-\D-galactopyranosyl-\D-glucopyranose},
\ce{C12H22O11}) is a disaccharide consisting of a \iupac{\D-glucose} and a
\iupac{\D-galactose} joined by a \iupac{\chembeta-1,4-glycosidic} bond (Figure
\ref{fig:lactose-structure}). In the dairy industry, crystallization is a
separation process that refines lactose from whey solutions
\cite{Hourigan2013}. During this operation, the crystallization is considered a
two-step process; the first is called the nucleation
\cite{Schmitt1999,Wong2014} and second is growth of the nucleus
\cite{Hourigan2013}.  This nuclei from the first step grows depending on the
condition of supersaturation of lactose solution.  The lactose crystals are
affected by process parameters and solutions, including: temperature,
viscosity, pH, and the presence of impurities \cite{Bhargava1996}. The pH
changes the rate of mutarotation, which is an important factor in lactose
crystallization \cite{Ganzle2008,Hourigan2013}. A mutarotation occurs during
the change in the optical rotation between two anomers \cite{Jelen1973a}. As
the crystallization process removes only one anomer from the equilibrium,
depletion of the anomer may occur in the crystallizing form if mutarotation is
slow \cite{Ganzle2008}. As the slow rate of mutarotation limits the growth of
crystal, the crystal shape can be controlled \cite{Ganzle2008}. 
% --------------------------
\begin{figure}[htbp] \centering \includegraphics[width=0.5\textwidth]{fig3_01}
    \caption[Molecular structures of $\alpha$- and $\beta$- lactose.]{Molecular
    structures of $\alpha$- and $\beta$- lactose.}
    \label{fig:lactose-structure}
\end{figure}
% --------------------------

A diagram of a typical $\alpha$-lactose monohydrate (LM) crystal is shown in
Figure \ref{fig:lm-crystal}. The crystals are shaped as tomahawk with one
axis of symmetry. The solvent compositions and temperatures during
crystallization result in different yields, purities, shapes, and sizes of
lactose crystals \cite{Hourigan2013}.
% --------------------------ALM-crystal
\begin{figure}[htbp] \centering \includegraphics[width=0.5\textwidth]{fig3_11}
    \caption[Tomahawk crystal of $\alpha$-lactose monohydrate showing faceted
        structure. Crystals display a morphology having a broad base (010)
        further bounded by (100), (110), (110), and (011), (010) and (150)
    faces.] {Tomahawk crystal of $\alpha$-lactose monohydrate showing faceted
        structure. Crystals display a morphology having a broad base (010)
        further bounded by (100), (110), (110), and (011), (010) and (150)
        faces \cite{Kurimoto1999,Wong2014}.} 
    \label{fig:lm-crystal} 
\end{figure}
% --------------------------

\subsubsection{A Co-crystallization of Macromolecules with LM}

$\alpha$-lactose monohydrate (LM) is capable of incorporating various
macromolecules and biopolymers into its crystal structure
\cite{Wang2001a,Kurimoto1999}. It has been
hypothesized that this is due to peripheral hydrogen atoms of LM, which are
capable of forming hydrogen bonds to neighboring macromolecules
\cite{Kurimoto1999,Aizenberg1994}.

Kurimoto \latin{et al.} have demonstrated that LM is capable of stabilizing
green fluorescent protein (GFP) in its native condition in single crystals,
leading to dissolution into its native state \cite{Kurimoto1999} (Figure
\ref{fig:lm-intro}).  Upon denaturation, the GFP interior is exposed, and the
fluorescence is rapidly quenched. After LM crystallization with GFP (LM/GFP)
for three to four days, the LM/GFP crystals emits a green fluorescence while
excited with UV lamp (Figure \ref{fig:lm-intro}), localizing within a pyramid
shape corresponding to the (010) growth sector \cite{Kurimoto1999} (Figure
\ref{fig:lm-intro}). Upon elevated temperature at
\SI{59.85}{\celsius}, the LM/GFP crystals exhibits steady-state fluorescence.
In contrast, both saturated lactose solution and lyophilized LM decays
over 3600 seconds \cite{Kurimoto1999}.
% --------------------------
\begin{figure}[htbp] \centering \includegraphics[width=0.7\textwidth]{fig3_02}
    \caption[Crystals of lactose monohydrate (LM) as hosts for the guest green
    fluorescent protein (GFP)]{Crystals of lactose monohydrate (LM) as hosts
        for the guest green fluorescent protein (GFP) \cite{Wang2001a}.}
    \label{fig:lm-intro}
\end{figure}
% --------------------------

Wang \latin{et al.} have further demonstrated that multiple guests molecules could be
crystallized with LM, including GFP, Cytochrome c (Cyt C), lysozyme,
Ribonuclease B (RNAse B), lectin, or Avidin \cite{Wang2001a} (Figure
\ref{fig:lm-other}). These guest molecules exhibit different conformations; GFP
is 27 kDa protein with an 11-stranded $\beta$-barrel \cite{Kurimoto1999};  Cyt C
is 12.5 kDa with four $\beta$-stands \cite{Hirota2010}; lysozyme is 14.7 kDa
with five helical region as well as five $\beta$-sheets \cite{Schwalbe2001};
RNAse B is 15.9 kDa with a long four-stranded $\beta$-sheets and three short
$\alpha$-helices \cite{Lamontagne2001}; lectin is 30 kDa with five 4-stranded
$\beta$-sheets \cite{Rini1995}, and avidin is a 16.5 kDa dimeric protein with
one eight-stranded $\beta$-sheets in one monomer \cite{Rosano1999} (Figure
\ref{fig:lm-other}). The LM crystallization has been performed at
\SI{30}{\celsius} for 15 to 36 hours in the presence of \ce{Zn}
porphyrin-cytochrome c (Zn-Cyt c), fluorescein labeled-lysozyme (Fl-lysozyme),
texas red-labeled lectin (TR-lectin), fluorescein labeled-ribonuclease B (RNAse
B) or fluorescein labeled-avidin.  Notably, the luminescence is observed at the
(010) growth sector across these examples (Figure \ref{fig:lm-other}). After
protein uptake has been quantified by amino acid analysis, a molar ratio of
1:\num{6e5} to 1:6$\times$10 (Guest:Host) is determined \cite{Wang2001a}. 
% --------------------------
\begin{figure}[htbp] \centering \includegraphics[width=1.0\textwidth]{fig3_29}
    \caption[Crystals of $\alpha$-lactose monohydrate containing corresponding
    proteins. In each case, the mixed crystal shows a characteristic
    fluorescence from the (010) growth sector. GFP is a 27 kDa protein with
    11-stranded $\beta$-barrel;  Cyt C is a 12.5 kDa with 4 $\beta$-stands;
    lysozyme is a 14.7 kDa enzyme with a half-chair conformation; RNAse B is a
    15.9 kDa enzyme with a long 4-stranded $\beta$-sheets and 3 short
$\alpha$-helices. ]{Crystals of $\alpha$-lactose monohydrate containing
    corresponding proteins. In each case, the mixed crystal shows a
    characteristic fluorescence from the (010) growth sector \cite{Wang2001a}.
    Cyt C is 12.5 kDa with four $\beta$-stands \cite{Hirota2010}; lysozyme is
    14.7 kDa with five helical region as well as five $\beta$-sheets
    \cite{Schwalbe2001}; RNAse B is 15.9 kDa with a long four-stranded
    $\beta$-sheets and three short $\alpha$-helices \cite{Lamontagne2001};
    lectin is 30 kDa with five 4-stranded $\beta$-sheets \cite{Rini1995}, and
    avidin is a 16.5 kDa dimeric protein with one eight-stranded $\beta$-sheets
    in one monomer \cite{Rosano1999}. RNAse B contains a single glycosylation
    site at Asn34 that consists of an N-linked Man\textsubscript{5-9}
    GlcNAc\textsubscript{2} carbohydrate, and a avidin contains a heterogeneous
    carbohydrate structure also composed of Man and GlcNAc}
    \label{fig:lm-other}
\end{figure}
% --------------------------

To understand the complex binding process of a guest molecules to a growing
crystal surface, Wang \latin{et al.} performed differential interference
contrast (DIC) microscopy and atomic force microscopy (AFM) to investigate the
luminescence pattern and the topography of the growth active surface
\cite{Wang2001a}. The growth of LM crystal was observed only on the (010) face
of the crystal. The images indicate that GFP was only deposited at the lateral
slopes.  Unfortunately, the molecular basis of this selectivity is unclear
\cite{Wang2001a}.
% --------------------------
\begin{figure}[htbp] \centering \includegraphics[width=0.6\textwidth]{fig3_37}
    \caption[Differential interference
contrast (DIC) micrograph of a pure LM crystal (010) face (\SI{1}{\cm} across,
bottom right) and atomic force microscopy (AFM) images of vicinal sectors where
GFP would (top right) and would not (bottom left) be recognized. An AFM image
of the hillock core is at top left.]{Differential interference contrast (DIC)
    micrograph of a pure LM crystal (010) face (\SI{1}{\cm} across, bottom
    right) and atomic force microscopy (AFM) images of vicinal sectors where
    GFP would (top right) and would not (bottom left) be recognized. An AFM
    image of the hillock core is at top left \cite{Wang2001a}.}
    \label{fig:010}
\end{figure}
% --------------------------

In this chapter, we co-crystallize PTE with LM, leading to the enhanced
stability and extended shelf life at room temperature within LM crystals. This
is the first report that uses enzyme kinetic parameter to evaluate the PTE
stability in LM crystal. The formulation provides effective storage of PTE, and
ultimately can be used for multiple applications in agriculture and military. 

\section{Methods}

\subsection{General}

Fluorescein isothiocyanate (FITC) FluoroTag kit and paraoxon were purchased
from Sigma (St. Louis, MO).  $\alpha$-lactose monohydrate was also purchased
from Sigma (St. Louis, MO). All other chemicals, including \ch{NaCl},
\ch{CoCl2}, Tris-HCl, tryptone, yeast extract, paraoxon, ampicillin,
chloramphenicol, sodium phosphates monobasic, sodium phosphate dibasic,
imidazole, or Whatman\textregistered filter papers were purchased from Sigma
(St. Louis, MO) or VWR (Radnor, PA).  96-well plates were purchased from Thermo
Fisher Scientific (Waltham, MA). FPLC column was purchased from G.E Healthcare
(Piscataway, NJ). A 3.5K MWCO dialysis SnakeSkin was purchased from Life
Technologies (Carlsbad, CA).

\subsection{Protein Expression}
\label{sec:pte-chap3}

PTE DNA, pQE30-PTE, was transformed into AFIQ cells as described in our previous
work \cite{Yang2014a} and Chapter \ref{chap:uaa}. Cells were plated on agar
plates containing \SI{200}{\ug\per\mL} ampicillin, \SI{34}{\ug\per\mL}
chloramphenicol. A single colony was picked and grown in Lysogeny broth (LB)
with \SI{200}{\ug\per\mL} ampicillin, and \SI{34}{\ug\per\mL} chloramphenicol)
at \SI{37}{\celsius}, 300 r.p.m for 16 hours \SI{37}{\celsius} incubation.
Afterwards, \SI{250}{\mL} of LB medium for large-scale expression was
innoculated 1:50 with the overnight culture.  After optical density reached 1.0
at 600 nm, the expression media was supplemented with \SI{1}{\milli\Molar}
isopropyl-$\beta$-D-thiogalactopyranoside (IPTG) to induce protein expression.
In addition, \SI{1}{\milli\Molar} of \ce{CoCl2} was added in each
post-induction medium.  After three hours incubation at \SI{37}{\celsius}, 300
r.p.m., the cells were harvested by using 4000 r.p.m centrifugation (Beckman
Coulter, Jersey City, NJ.  F10 rotor) at \SI{4}{\celsius} for 15 minutes and
then resuspended with \SI{20}{\milli\Molar} Tris-HCl, \SI{500}{\milli\Molar}
\ce{NaCl}, \SI{5}{\milli\Molar} imidazole, 10\% glycerol (pH 8.0) and
\SI{100}{\micro\Molar} \ce{CoCl2}. Cell lysate was immediately sonicated at 400 kJ
for 1.5 minutes at \SI{4}{\celsius} (Q500 sonicator, Qsonica, Newtown, CT) and then a
clarification spin was performed (20,000 g, \SI{4}{\celsius}, 30 minutes).
Clarified supernatants were loaded into a \SI{5}{\mL} His Trap column (G.E
Healthcare, Piscataway, NJ) using AKTA FPLC purifier (G.E.  Healthcare,
Piscataway, NJ).  Protein elution was generated using 30\% elution buffer B
(\SI{20}{\milli\Molar} Tris-HCl, \SI{500}{\milli\Molar} sodium chloride,
\SI{500}{\milli\Molar} imidazole (pH 8.0)). The purified samples were then
transferred into 3.5K MWCO dialysis SnakeSkin (Life Technologies, Carlsbad, CA)
for buffer exchange using \SI{12}{\L} \SI{20}{\milli\Molar} phosphate buffer
(pH 8.0) at \SI{4}{\celsius} overnight. The purify of protein was
determined by sodium dodecyl sulfate polyacrylamide gel electrophoresis 
(SDS-PAGE) analysis. The protein concentration was measured by Nano-Drop Thermo
Scientific (Waltham, MA) by using the extinction coefficient
\SI{29575}{\per\Molar\per\cm} for PTE \cite{Gasteiger2005, Pace1995}. Dialyzed
protein was subjected to crystallization immediately.

\subsection{Protein Conjugation to Fluorescein Isothiocyanate and Its
Crystallization}

Fluorescein isothiocyanate (FITC) kit was used for fluorescein conjugation with
PTE. FITC dissolved in \SI{0.1}{\Molar} sodium carbonate-bicarbonate buffer (pH
9.0).  \SI{1}{\mg\per\mL} was added to the \SI{0.14}{\mg\per\mL} enzyme in
\SI{20}{\milli\Molar} phosphate buffer (pH 8.0, \SI{100}{\micro\Molar}
\ce{CoCl2}) to a final ratio of 1:500 (PTE:FITC). The reaction mixture was
incubated for \SI{1.5}{hour} at room temperature in tube rack, then dialyzed
into 3.5K MWCO dialysis SnakeSkin against \SI{2}{\liter} of phosphate buffer
(\SI{20}{\milli\Molar}, pH 8.0, \SI{100}{\micro\Molar} \ce{CoCl2}) at
\SI{4}{\celsius} overnight.  The resulting FITC-PTE sample was then transferred
to a lactose solution for crystallization (FITC-PTE.LM). After the lactose
solution was cooled down to estimated \SI{40}{\celsius}, a total of \SI{3}{\mL} elution
(\SI{20}{\milli\Molar} phosphate buffer (pH 8.0, \SI{100}{\micro\Molar}
\ce{CoCl2}) was added to \SI{2}{\mL} \SI{0.40}{\gram\per\mL} LM in water at
\SI{6}{\celsius} for approximately 1 month incubation.

\subsection{Lactose and Crystallization of PTE}
\label{sec:mm-xal}

To prepare PTE.LM crystals, a total of \SI{0.20}{\milli\gram} purified PTE was
added to \SI{2}{\mL} \SI{0.28}{\gram\per\mL} lactose solution
(\SI{20}{\milli\Molar} PBS buffer, pH 7.4). The super saturated lactose was
prepared by dissolving \SI{0.6}{\gram} $\alpha$-lactose monohydrate into boiled
\SI{20}{\milli\Molar} PBS buffer (pH 7.4). PTE protein (in
\SI{20}{\milli\Molar} phosphate buffer, \SI{100}{\micro\Molar} \ce{CoCl2}, pH
8.0) was prepared according to the procedure in Section \ref{sec:pte-chap3}.
After the lactose solution was cooled down to estimated \SI{40}{\celsius},
\SI{0.20}{\milli\gram} PTE was then added. The mixture was incubated at
\SI{6}{\celsius} for roughly 2.5 weeks until PTE.LM crystals of a suitable size
were obtained.  Powder lactose monohydrate (less than \SI{1}{\mg}) was used as
the seed for crystallization. The crystals were then harvested by filtration on
a \SI{11}{\micro\meter} Whatman filter paper and washed with distilled water
and dried under room temperature. PTE.LM crystals were then stored in vials at
\SI{4}{\celsius} or room temperature for subsequent microscope studies or
paraoxon hydrolysis experiments.

\subsection{Fluorescence Phase Contrast Microscopy}

The crystal (FITC-PTE.LM) was then first observed under the Zeiss Axioskop 40
Fluorescence Microscope (Peabody, MA).  Wavelength \SI{495}{\nm} was used for
excitation, and \SI{525}{\nm} was used for emission. The 4X object lens was used to
collect images of FITC-PTE.LM.

\subsection{Estimate of PTE in LM}

Before crystallization, PTE concentration was measured using Nano-Drop (Thermo
Scientific, Waltham, MA). Approximately \SI{0.20}{\mg\per\mL} of PTE was
collected after dialysis in \SI{20}{\milli\Molar} phosphate buffer
(\SI{100}{\micro\Molar} \ce{CoCl2}, pH 8.0). The co-crystallization process was
described in Section \ref{sec:mm-xal}. After approximately 2.5 weeks, the
supernatant of co-crystallization mixture was collected for the estimate of PTE
concentration in solution. Using Nano-Drop, the protein concentration was
estimated by the extinction coefficient \SI{29575}{\per\Molar\per\cm} for PTE
\cite{Gasteiger2005, Pace1995} 

\subsection{Enzyme Kinetics}

The crystal (both PTE.LM and FITC-PTE.LM) was reconstituted in
\SI{20}{\micro\liter} sodium phosphate (pH 8.0, \SI{100}{\micro\Molar}
\ce{CoCl2}). Reactions were monitored spectrophotometrically (Synergy H1,
BioTek, Winooski VT) at \SI{405}{\nm} for paraoxon (coefficient =
\SI{17000}{\per\Molar\per\cm}).  Reactions for paraoxon
(\SIrange{13}{104}{\micro\Molar}) was done in 0.2\% methanol.
K\textsubscript{M} and V\textsubscript{max} values were determined by a
Lineweaver-Burk plot \cite{Baker2011b}. The equation used is shown below
(Eq.~\ref{eqn:MM-chap3}): 
\begin{equation} 
    \frac{1}{v} =
    \frac{K\textsubscript{M}}{V\textsubscript{max}}\times\frac{1}{S} +
    \frac{1}{V\textsubscript{max}} 
    \label{eqn:MM-chap3}
\end{equation}
where S represents substrate concentration in \si{\Molar}; K\textsubscript{M}
represents the substrate concentration at which the reaction rate is half of
V\textsubscript{max}. The data reported is the average of three trials and the
error represents the standard deviation of those trials.

\section{Results and Discussion}

\subsection{Biosynthesis of PTE And Crystallization of PTE.LM}

Wild-type PTE was biosynthesized from the pQE30-PTE plasmid \cite{Yang2014a}.
After transformation into AFIQ \latin{E. coli} \cite{Baker2011b}, expression
was performed in LB media. After three hours of induction with
\SI{1}{\milli\Molar} IPTG, pure PTE protein was isolated via FPLC (Chapter
\ref{chap:uaa}), and the purity of protein was analyzed via SDS-PAGE. After
dialysis in \SI{20}{\milli\Molar} phosphate buffer (pH 8.0,
\SI{100}{\micro\Molar} \ce{CoCl2}), the protein concentration was determined to
be \SI{0.139}{\mg\per\mL} by using Nano-Drop, consistent with previous studies
\cite{Yang2014a}.

To prepare the lactose solution for crystallization, supersaturated solutions
of \SI{0.28}{\gram\per\mL} LM was incubated with \SI{0.20}{\milli\gram} PTE
protein at \SI{40}{\celsius}. The mixture was then cooled down to estimated
\SI{36}{\celsius} temperature at a rate of \SI{0.5}{\celsius\per\minute},
consistent with the range that has been reported \cite{Valle-Vega1977}. The
resulting PTE.LM crystals were harvested after roughly 2.5 weeks incubation at
\SI{4}{\celsius} (Figure \ref{fig:ptelm-bottle}). \SI{10}{\milli\gram} of
PTE.LM was collected, yielding 1.5 \% from the lactose monohydrate and PTE
used, consistent with three batches of crystallization. 
% --------------------------ptelm-bottle
\begin{figure}[htbp] \centering \includegraphics[width=0.8\textwidth]{fig3_07}
    \caption[The crystal of PTE.LM. A bottle contains
        \SI{79}{\mg} of PTE.LM after 2.5 weeks incubation at \SI{4}{\celsius}.
        \SI{0.2}mg PTE were crystallized in \SI{0.28}{\gram\per\mL} lactose
        solution (\SI{20}{\milli\Molar} PBS buffer, pH 7.4).]{The
            crystal of PTE.LM. A bottle contains \SI{79}{\mg} of PTE.LM
            after 2.5 weeks incubation at \SI{4}{\celsius}.  \SI{0.2}mg PTE
            were crystallized in \SI{0.28}{\gram\per\mL} lactose solution
            (\SI{20}{\milli\Molar} PBS buffer, pH 7.4).}
        \label{fig:ptelm-bottle} 
\end{figure}
% --------------------------*

Using supersaturated lactose solution, purified PTE was successfully deposited
onto the lactose crystals yielding PTE.LM crystals measuring
\SIrange{0.5}{3}{\mm} in length. Using phase contrast microscope, the dimension
of a PTE.LM co-crystal was calculated at \SI{3.0}{\mm} (h) $\times$ \SI{1.0}{\mm}
(w) $\times$ \SI{0.5}{\mm} (d) (Figure \ref{fig:ptelm-image}). Previously, protein
co-crystals were expected to display a hatchet morphology having a broad base
(010) further bound by (100), (110) and (011) \cite{Kurimoto1999}. While Kurimoto
\latin{et al.} \cite{Kurimoto1999} and Wang \latin{et al.} \cite{Wang2001a}
reported the similar sizes of crystal as we measured, the shape of PTE.LM
crystals did not exhibit tomahawk form \cite{Kurimoto1999,Wong2014}. However,
Jelen and Ganzle \latin{et al.} demonstrated that different crystal shapes, such as
pyramid or prism, could be formed from different incubation temperatures, pH,
or salt of monosodium phosphate \cite{Jelen1973a,Jelen1973,Ganzle2008} (Figure
\ref{fig:pyramid}). As Kurimoto \latin{et al.} \cite{Kurimoto1999} and Wang
\latin{et al.} \cite{Wang2001a} co-crystallized guest molecules in the
absence of sodium phosphate, we expected a pyramid shape different from
tomahawk.
% --------------------------
\begin{figure}[htbp] \centering \includegraphics[width=0.7\textwidth]{fig3_03} 
    \caption[Microscope image of PTE.LM. The estimated dimension of pyramidal
    PTE.LM crystal are estimated as  \SI{3.0}{\mm} (h) $\times$ \SI{1.0}{\mm}
(w) $\times$ \SI{0.5}{\mm} (d).]{Microscope image of PTE.LM. The estimated
    dimension of pyramidal PTE.LM crystal are estimated as  \SI{3.0}{\mm} (h)
    $\times$ \SI{1.0}{\mm} (w) $\times$ \SI{0.5}{\mm} (d).}
    \label{fig:ptelm-image} 
\end{figure}
% --------------------------

% --------------------------
\begin{figure}[htbp] \centering \includegraphics[width=0.8\textwidth]{fig3_30}
    \caption[Crystals of lactose in the single crystal growth studies (The
    scale is in cm).] {Crystals of lactose in the single crystal growth studies
        (The scale is in cm) \cite{Jelen1973}.}
    \label{fig:pyramid}
\end{figure}
% --------------------------

\subsection{Reconstituted PTE.LM Hydrolysis Reaction}

While wild-type PTE in solution hydrolyzed OP substrates \cite{Yang2014a,Baker2011b},
it exhibited short half life, losing activity within one week (Chapter
\ref{chap:uaa}, Figure \ref{fig:kinetics-fig}) \cite{Yang2014a}. Further
evidence demonstrated that PTE required stabilization through embodiment, such
as immobilization and formulation discussed in introduction
\cite{Chen1998,Gill2000,Havens1993,Masson2009a}. To stabilize PTE, it was
crystallized in the presence of LM, and its ability to hydrolyze paraoxon after
storage was assessed. After the crystallization, PTE.LM was stored at room
temperature or \SI{4}{\celsius}. Approximately \SI{0.05}{\mg\per\mL} of PTE.LM
was prepared in \SI{20}{\milli\Molar} phosphate buffer (pH 8,
\SI{100}{\micro\Molar} \ce{CoCl2}) for activity measurement. One month old
PTE.LM stored at \SI{4}{\celsius} revealed activity for paraoxon hydrolysis
after reconstituted in \SI{200}{\micro\liter} \SI{20}{\milli\Molar} (pH 8,
\SI{100}{\micro\Molar} \ce{CoCl2}) phosphate buffer (Figure
\ref{fig:ptelm-one-month}) at room temperature. 
% --------------------------
\begin{figure}[htbp] \centering \includegraphics[width=0.7\textwidth]{fig3_08}
    \caption[Reconstituted one month old PTE.LM hydrolysis of paraoxon.
        \SI{10}{\mg} of PTE.LM was dissolved in \SI{200}{\micro\liter} of
        \SI{20}{\milli\Molar} phosphate buffer. Paraoxon was prepared at
        \SI{200}{\micro\liter} in \SI{20}{\milli\Molar} phosphate buffer (0.4
    \% MeOH). The hydrolysis reaction was measured at \SI{405}{\nm}.]
    {Reconstituted one month old PTE.LM hydrolysis of paraoxon.  \SI{10}{\mg}
    of PTE.LM was dissolved in \SI{200}{\micro\liter} of \SI{20}{\milli\Molar}
    phosphate buffer (pH8, \SI{100}{\micro\Molar} \ce{CoCl2}).  Paraoxon was
    prepared at \SI{200}{\micro\liter} in \SI{20}{\milli\Molar} phosphate
    buffer  (0.4 \% MeOH).  The hydrolysis reaction was measured at
    \SI{405}{\nm}.} \label{fig:ptelm-one-month} 
\end{figure}
% --------------------------

After harvesting the PTE.LM crystals, the supernatant after the co-crystallization
was collected. Using the extinction coefficient \SI{29575}{\per\Molar\per\cm}
for PTE \cite{Gasteiger2005, Pace1995}, the protein concentration was measured
at 280 nm via Nano-Drop, leading to an estimate of \SI{0.13}{\mg\per\mL} of PTE
in the supernatant. Using the original PTE concentration of
\SI{0.20}{\mg\per\mL} after dialysis, approximately \SI{0.07}{\mg\per\mL}
of PTE was deposited into PTE.LM. 

The absorbance at \SI{405}{\nm} clearly demonstrated the
increased hydrolysis of \emph{p}-nitrophenol by the one-month old PTE.LM
(Figure \ref{fig:ptelm-one-month}). PTE.LM crystals after 2 months
storage at \SI{4}{\celsius} also demonstrated hydrolysis of paraoxon (Figure
\ref{fig:ptelm-two-month}). By contrast, paraoxon in the absence of
PTE revealed no hydrolysis, indicating that the PTE.LM crystals indeed were
active. Assuming the same amount of PTE within the crystals as these were of
the same batch as the 1-month sample, the 2-month sample exhibited less
activity (Figure \ref{fig:ptelm-two-month}). When compared to a dried PTE
sample dissolved in buffer, the 2-month PTE.LM sample exhibited activity while
no detectable hydrolysis was observed with the dried stored PTE (Figure
\ref{fig:ptelm-two-month}). This suggested that the crystallization of PTE in
LM provided stability.
% --------------------------
\begin{figure}[htbp] \centering \includegraphics[width=1.0\textwidth]{fig3_09}
    \caption[(A) Reconstituted two month old PTE.LM hydrolysis of paraoxon.
        \SI{10}{\mg} of PTE.LM was dissolved in \SI{200}{\micro\liter} of
        \SI{20}{\milli\Molar} phosphate buffer. Paraoxon was prepared at
        \SI{200}{\micro\liter} in \SI{20}{\milli\Molar} phosphate buffer. The
        hydrolysis reaction was measured at \SI{405}{\nm}. Dried PTE represents
        the air-dried PTE that was stored at room temperature and dissolved in
        \SI{20}{\milli\Molar} phosphate buffer (pH 8, \SI{100}{\micro\Molar}
        \ce{CoCl2}). (B) Control experiment of the same paraoxon with
        \SI{0.08}{\gram\per\mL} lactose solution (\SI{20}{\milli\Molar}
        phosphate buffer, pH 8, \SI{100}{\micro\Molar} \ce{CoCl2}).]
        {(A) Reconstituted two month old PTE.LM hydrolysis of paraoxon.
            \SI{10}{\mg} of PTE.LM was dissolved in \SI{200}{\micro\liter} of
            \SI{20}{\milli\Molar} phosphate buffer (pH 8,
            \SI{100}{\micro\Molar} \ce{CoCl2}). Paraoxon was prepared at
            \SI{200}{\micro\liter} in \SI{20}{\milli\Molar} phosphate buffer.
            The hydrolysis reaction was measured at \SI{405}{\nm}.  Dried PTE
            represents the air-dried PTE from the soluble sample stored at room
            temperature. The dried PTE was then dissolved in
            \SI{20}{\milli\Molar} phosphate buffer (pH 8,
            \SI{100}{\micro\Molar} \ce{CoCl2}). (B) Control experiment of the
            same paraoxon with \SI{0.08}{\gram\per\mL} lactose solution
            (\SI{20}{\milli\Molar} phosphate buffer, pH 8,
            \SI{100}{\micro\Molar} \ce{CoCl2})} \label{fig:ptelm-two-month} 
\end{figure}
% --------------------------

To increase the amounts of PTE.LM for multiple measurements of activity,
volumes of LM and PTE were both increased to 2-fold with the same concentration
described in the previous section, followed by the same preparation at
\SI{6}{\celsius}. The resulting PTE.LM crystals were harvested at \SI{17}{\mg},
consistent with the 1.8\% yield reported in the previous batch. 

Using this batch of PTE.LM, crystals was stored at room temperature to
investigate half-life and stability of PTE.LM. After storing at room
temperature for ~3.5 months, \SI{0.08}{\gram\per\mL} PTE.LM crystals was then
reconstituted in \SI{1}{\mL} \SI{20}{\milli\Molar} phosphate buffer (pH 8,
\SI{100}{\micro\Molar} \ce{CoCl2}). To determine kinetics,
\SIrange{13}{104}{\micro\Molar} paraoxon was then used for assays.  Within
three minutes, reactions reached the maximum absorbance at \SI{405}{\nm}
(Figure \ref{fig:ptelm-hydrolysis}). Within the linear range of the reaction,
enzyme kinetics was calculated with V\textsubscript{max} and K\textsubscript{M}
of 0.0043 $\pm$ \SI{0.0012}{\micro\Molar\per\second} and 1186 $\pm$
\SI{336}{\micro\Molar}, respectively (Table \ref{tab:ptelm-table}).  Compared
to the freshly prepared wild-type PTE in solution, reconstituted PTE.LM
demonstrated decreased activity in K\textsubscript{M} by 3.5-fold and Vmax by
118-fold, leading to an overall 416-fold loss in Vmax/K\textsubscript{M}.  

The loss in activity may be contributed a number of factors. Due to the
elevated temperature at estimated \SI{40}{\celsius} during the crystallization
process, the PTE activity may be partially impaired from heat inactivation as
observed from Masson and Tawfik group \cite{Rochu2002b,Roodveldt2005}. Rochu
\latin{et al.} have reported a 5\% loss and 25\% of PTE activity at 40 and
\SI{45}{\celsius}, respectively \cite{Rochu2002b}. However, this subtle change
in temperature was not the only factor. The dilution of \ce{Co^{2+}} in the
active site could impact the PTE catalytic efficiency. During the
crystallization, LM solution did not contain \ce{Co^{2+}}, leading to a 3-fold
dilution of \ce{Co^{2+}}. While Omburo \latin{et al.} have demonstrated the
total inactivation of \ce{Co^{2+}}-PTE in the presence of chelating agents
\cite{Omburo1992a}, we also contribute the loss of activity to \ce{Co^{2+}}
deprivation \cite{Benning1995,Samples2005}. 
% --------------------------
\begin{figure}[htbp] \centering \includegraphics[width=0.8\textwidth]{fig3_05} 
    \caption[The hydrolysis of reconstituted PTE.LM after storage at room
    temperature for 3.5 months. (A) paraon hydrolysis and (B) Lineweaver Burk
plot.]{The hydrolysis of reconstituted PTE.LM after storage at room temperature
for 3.5 months. (A) paraon hydrolysis and (B) Lineweaver Burk plot.}
\label{fig:ptelm-hydrolysis} \end{figure}
% --------------------------
% --------------------------ptelm-table
\begin{table}[htbp]
    \centering
    \caption[Kinetic comparisons of PTE and PTE.LM.]{Kinetics comparisons of
    PTE and PTE.LM.} 
    \begin{tabular}{ llll }
        \hline
        Protein & K\textsubscript{M} & Vmax & Vmax/K\textsubscript{M} \\
        \hline
        PTE \cite{Yang2014a} & 342 $\pm$ 102 & 0.51 $\pm$ 0.13 & \num{1.5e-3}\\
        PTE.LM & 1186 $\pm$ 336 & 0.0043 $\pm$ 0.0012 & \num{3.6e-6}\\
        \hline
        \multicolumn{4}{l}{VMax/K\textsubscript{M}: \SI{}{\per\second};
        K\textsubscript{M}: \SI{}{\micro\Molar}.}
    \end{tabular}
    \label{tab:ptelm-table} 
\end{table}
% --------------------------

\subsection{FITC Conjugation And Crystallization}

Kurimoto \latin{et al.} have estimated the amount of GFP from the absorbance of
a LM/ GFP crystal using the known absorbance at 395 nm
(\SI{30000}{\liter\per\Molar\per\cm}) \cite{Kurimoto1999}. To visualized the
location of non-GFP protein in the crystal, Wang \latin{et al.} conjugated
fluorescein onto Cyt c, lysozyme, lectin, RNAse B, or avidin, leading to
characterization of proteins at (010) growth sector \cite{Wang2001a}. To both
visualize and quantify the amounts of PTE in crystals, fluorescein
isothiocyanate (FTIC) conjugation was performed to label with the lysine and
N-terminus of PTE. The FITC probe has an absorption at \SI{495}{\nm} and
emission at \SI{525}{\nm}, and forms a stable conjugate with free amino groups
of proteins \cite{Rogers1999}. After the conjugation, the resulting FITC-PTE
was incubated with LM after the LM solution was cooled down to an estimated
\SI{40}{\celsius}, leading to the crystallization at \SI{6}{\celsius} for one
month (Figure \ref{fig:ptelm-fitc}). \SI{20}{\mg} of FITC-PTE.LM was harvested,
consistent with the yield reported in the previous batch.
% --------------------------
\begin{figure}[htbp] \centering \includegraphics[width=0.8\textwidth]{fig3_10} 
    \caption[The image of PTE.LM-FITC crystal. An absorption at \SI{495}{\nm}
    and an emission at \SI{525}{\nm} were used for FITC probe. The crystal was
collected after two weeks crystallization.]{The image of PTE.LM-FITC crystal.
    An absorption at \SI{495}{\nm} and an emission at \SI{525}{\nm} were used
for FITC probe. The crystal was collected after two weeks crystallization.}
\label{fig:ptelm-fitc} 
\end{figure}
% --------------------------

The FITC-PTE.LM crystals illustrate identical pyramidal shape, consistent with
PTE.LM crystals (Figure \ref{fig:ptelm-fitc}). However, the fluorescence from
FITC did not demonstrate the (010) sector illumination in the
crystal as observed in Wang \latin{et al.} \cite{Wang2001a} (Figure
\ref{fig:ptelm-fitc}). This could contribute to low concentration of PTE. In
comparison with \SI{1}{\mg\per\mL} of GFP used in the previous study
\cite{Kurimoto1999}, only \SI{0.14}{\mg\per\mL} of PTE is used for conjugation
prior to crystallization. In addition, the hydrolysis experiment demonstrates
that the reconstituted FITC-PTE.LM (\SI{0.05}{\mg\per\mL} in
\SI{200}{\micro\liter} \SI{20}{\milli\Molar} phosphate buffer, pH 8,
\SI{100}{\micro\Molar} \ce{CoCl2}) is inactive toward substrate paraoxon
(Figure \ref{fig:last}). While Rogers \latin{et al.} have demonstrated that the
conjugation of FITC with PTE in buffer resulted in 50\% loss of PTE activity
\cite{Rogers1999}, the combination of crystallization process and FITC
conjugation leads to a greater activity loss of FITC-PTE.LM. To visualize the
functional PTE inside the crystal, especially at extremely low concentration,
alternative method are required for this study.
% --------------------------
\begin{figure}[h!] \centering \includegraphics[width=0.7\textwidth]{fig3_38}
    \caption[Reconstituted FITC-PTE.LM hydrolysis of paraoxon. \SI{10}{\mg} of
    FITC-PTE.LM was dissolved in \SI{200}{\micro\liter} of
\SI{20}{\milli\Molar} phosphate buffer. Paraoxon was prepared at
\SI{200}{\micro\liter} in \SI{20}{\milli\Molar} phosphate buffer (0.4\% MeOH).
The hydrolysis reaction was measured at \SI{405}{\nm}.]{Reconstituted
    FITC-PTE.LM hydrolysis of paraoxon. \SI{10}{\mg} of FITC-PTE.LM was
    dissolved in \SI{200}{\micro\liter} of \SI{20}{\milli\Molar} phosphate
    buffer. Paraoxon was prepared at \SI{200}{\micro\liter} in
    \SI{20}{\milli\Molar} phosphate buffer (0.4\% MeOH). The hydrolysis
    reaction was measured at \SI{405}{\nm}.}
    \label{fig:last}
\end{figure}
% --------------------------

\section{Future work}

Using supersaturated $\alpha$lactose monohydrate solution, PTE was deposited in
pyramidal LM crystals (Figure \ref{fig:ptelm-image}). Kinetic assays further
demonstrated that PTE was active to paraoxon after storage up 3.5 months
(Figure \ref{fig:ptelm-two-month}, Figure \ref{fig:ptelm-one-month}, Figure
\ref{fig:ptelm-hydrolysis}). This is the first report to identify catalytic
efficiency of guest molecule inside the LM crystal. Notably, LM crystallization
also stabilized PTE under room temperature, which is consistent with the GFP study
(Figure \ref{fig:ptelm-hydrolysis}). While the high temperatures during the
co-crystallization of PTE.LM or \ce{Co^{2+}} depletion may contribute to
approximately 416-fold loss of activity to paraoxon (Table
\ref{tab:ptelm-table}), further optimization during crystallization process may
be done. With the removal of monosodium phosphate in buffer, we may expect the
tomahawk shape of PTE.LM. In addition, to maintain a more functional PTE in the
PTE.LM, \ce{Co^{2+}} may be added into LM solution. 

To quantify the catalytic efficiency of
PTE in the crystals, FITC provided limited information of protein
concentration. The FITC-PTE.LM results showed the absence of illumination at
the (010) sector. In addition, Rogers \latin{et al.} demonstrated that the
conjugation of FITC with PTE resulted in 50\% loss of PTE activity
\cite{Rogers1999}, indicating that an alternative quantification method is required
for PTE.LM.

Using inductively coupled plasma-mass spectrometry
(ICP-MS), the \ch{Co^{2+}} at the active site could be detected, providing the
amounts of PTE in the crystals. ICP-MS ionizes the PTE.LM with inductively
coupled plasma using an electromagnetic coil \cite{Garbarino1996}. The ionized
molecules are then separated based on their mass-to-charge ratio proportional
to the concentration \cite{Garbarino1996}. The detection limit is generally in
the range between 50 to \SI{100}{\pico\gram\per\mL} \cite{Garbarino1996}. This
information in addition to the k\textsubscript{cat}/K\textsubscript{M}
calculated from the protein concentration, will further demonstrate the impact
during the co-crystallization in a more accurate manner.

% removed content
%The other application was developed for the decontamination of
%organophosphates
%\cite{Cheng1996,LeJeune1997a,Little1989,Chen1998,Gill2000,Havens1993,Masson2009a}.

\printbibliography[heading=subbibliography]

\end{refsection}

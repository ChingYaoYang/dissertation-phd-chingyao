\chapter{Formulation of Phosphotriesterase Using $\alpha$-Lactose Monohydrate} 
\label{chap:lactose}

\begin{refsection}

\section{Introduction}

\subsection{Applications of Phosphotriesterase}

The unique catalytic properties of enzymes led to their rapid exploitation in
the food industry,219 analytical chemistry,220 preparative organic chemistry,221
medicine,222-223 etc. Moreover, of course, the whole field of genetic
engineering (recombinant DNA technology) would not have been possible without
the use of enzymes. However, it quickly became apparent that there were
limitations to this technology owing to the denaturation or inactivation of
enzymes brought about by heat, proteolysis, action of organic solvents, etc.
There was a major incentive to find solutions to these problems in order to
take advantage of enzymes active, specificity and other attractive features.
Even in the early 1970's there were already several good reviews on the
development of the new biochemistry based on immobilization procedures.224-226

Entrapment is the method that has had the greatest use in large-scale enzyme
operations. Since most enzymes used commercially are intracellular, the
simplest method, which is not shown in the figure, is to use whole cell
preparations where the enzyme is never released from the bacteria in which it
is produced. One of the best examples of this is glucose isomerase, which has
been used in the commercial production of high fructose corn syrup (HFCS) since
1967.230 Most of the glucose isomerase used in the production of over 6,000,000
tons of HFCS per year is in the form of immobilized whole cells. 2 3 1 This is
often done by spray drying the harvested cells to give a granulated product
that is then treated with polymeric materials (such as polyethylenimine) to
stabilize them. To further improve reactivity, the cells may be permeabilized.
This removes the barrier for the free diffusion of the substrate/product across
the cell membrane, and also empties the cell of most of the small molecular
weight cofactors, etc., thus minimizing unwanted side reactions. Obviously,
this and all other entrapment techniques are most applicable for low molecular
weight substrates and simple bioconversions like hydrolysis, isomerization and
oxidation reactions that do not require a cofactor-regeneration system. 23 2

It is possible to use entrapment as a mean of immobilizing a cofactor requiring
system. This can be accomplished in several ways. Initial efforts involved
covalently attaching the cofactor (usually NAD or NADH) directly onto a solid
support or in a gel matrix along with the enzymes.233 However, while the
enzymes and cofactor would be stable, the level of activity was generally quite
low. Other researchers attached the cofactor directly to the enzyme by means of
a bifunctional linker such as modified polyethylene glycol (PEG). The linker
would ideally be short enough to keep the cofactor close to the active site.
However, it would also need to be long and flexible enough to permit its easy
entrance in and out of the active site.234 The most commonly used method has
been to attach the cofactor to PEG or other large polymer. This is then placed
with the enzyme(s) in ultrafiltration systems with a semipermeable membrane,
microencapsulated, or immobilized in membranes. In most cases, the cofactor
would be regenerated with a second enzyme such as formate or alcohol
dehydrogenase, for which a second substrate is required. When the process is
being used for the synthesis of a valuable product, this can be cost effective.
However, for use in biodegradation such an approach would be too expensive. Two
enzymes described recently could simplify the system and reduce costs. The
first involves a hydrogenase that in the presence of hydrogen can regenerate
NADH.235 This is still somewhat involved since hydrogen would need to be
provided. An even more interesting enzyme is an NADH Oxidase that can use
dissolved oxygen to regenerate NAD with the production of water.2 3 6 Thus,
only aeration of the system would be required and no unwanted products would be
generated.

While entrapment in a gel matrix has been extensively used for the
immobilization of cells, it has not been as common for free enzymes. The major
limitation of this technique for enzymes is the possible slow leakage during
continuous use.237 The primary natural polymers used for entrapment have been
agar, agarose and gelatin through thermoreversal polymerization, and alginate
and carrageenan by ionotropic gelation. In addition to possible enzyme leakage,
these are relatively soft materials that will deform in large packed columns.
They are also subject to deterioration if used in fluidized bed reactors. In
the case of alginate and carrageenan, the ionic species used in the
polymerization (usually Ca2+) needs to be present continuously to maintain the
integrity of the gels. With some enzymes and processes, this can be a problem.

In the past, members of this laboratory have tested lactose's ability to
incorporate all kinds of molecules: large organic dyes, green fluorescent
protein, bovine serum albumin, horse-radish peroxidase, and now
phosphotriesterase (PTE). PTE is a dimeric protein composed of identical
subunits which come together in the protein's active state. PTE is capable of
hydrolyzing a wide range of organophosphates, and has been shown to work in the
breakdown and neutralization of pesticides and herbicides, as well as nerve
agents like sarin gas. This makes PTE an excellent target for kinetic
stabilization and storage by lactose - consider field medics in a warzone,
stocked with tablets of PTE stored in lactose crystals, or the application of
PTE/LM crystals to the processing of crops or other chemically treated
foodstuff.

\subsection{Lactose Monohydrate}

Lactose (4-O-β-D-galactopyranosyl-D-glucopyranose, \ce{C12H22O11} ) is a
disaccharide consisting of a D - glucose and a D - galactose molecule joined by
a β - 1,4 - glycosidic linkage. (Figure \ref{fig:lactose-structure})  It has
been previously shown by members of this laboratory that $\alpha$-lactose
monohydrate (LM) is capable of incorporating various macromolecules and
biopolymers into its crystal structure; it has been reasoned that this is made
possible by LM's abundance of peripheral hydrogen atoms, which are capable of
hydrogen bonding to said macromolecules, trapping them, orienting them, and
ultimately allowing the forming crystal to overgrow and absorb them1.

% --------------------------
\begin{figure}[h!] 
    \centering 
    \includegraphics[width=0.5\textwidth]{fig3_01}
    \caption[Molecular structures of $\alpha$- and $\beta$- lactose.]{Molecular
    structures of $\alpha$- and $\beta$- lactose.}
    \label{fig:lactose-structure}
\end{figure}
% --------------------------

In the dairy industry, crystallization is an important separation
process used in the refining of lactose from whey solutions. In the refining
operation, lactose crystals are separated from the whey solution through
nucleation, growth, and/or aggregation. The rate of crystallization is
determined by the combined effect of crystallizer design, processing
parameters, and impurities on the kinetics of the process. This review
summarizes studies on lactose crystallization, including the mechanism, theory
of crystallization, and the impact of various factors affecting the
crystallization kinetics. In addition, an overview of the industrial
crystallization operation highlights the problems faced by the lactose
manufacturer. The approaches that are beneficial to the lactose manufacturer
for process optimization or improvement are summarized in this review. Over the
years, much knowledge has been acquired through extensive research. However,
the industrial crystallization process is still far from optimized. Therefore,
future effort should focus on transferring the new knowledge and technology to
the dairy industry.

\subsubsection{Protein conjugation}

% This is such a depressing section to write about. I hate doing this shitty
% research in the lab. Why on earth that I am still not getting any job. There
% must be something there I did not solve. 

\section{Methods}

\subsection{Biosynthesis And Protein Purification}

After confirming the sequences of PTE variants through DNA sequencing, we
transformed the DNA into AFIQ cells as described in our previous
work\cite{Yang2014a}. Mutant and wild type plasmids were transformed into
electro-competent \latin{E.  coli} phenylalanine auxotrophic strains (AF-IQ
cells). Electroporation was done
at \SI{25}{\micro\farad}, \SI{100}{\ohm}, 2.5 kV (Biorad Gene Pulser II).
Cells were plated on agar plates containing \SI{200}{\ug\per\mL} ampicillin,
\SI{34}{\ug\per\mL} chloramphenicol. A single colony was picked and grown in LB
with \SI{200}{\ug\per\mL} ampicillin, and \SI{34}{\ug\per\mL} chloramphenicol)
at \SI{37}{\celsius}, 300 r.p.m for 16 hours \SI{37}{\celsius} incubation.
Afterwards, \SI{250}{\mL} of LB medium for large-scale expression was
innoculated 1:50 with the overnight culture.  After optical density reached 1.0
at 600 nm, the expression media were supplemented with \SI{1}{\milli\Molar}
isopropyl-$\beta$-D-thiogalactopyranoside (IPTG) to induce protein expression.
\SI{1}{\milli\Molar} of \ce{CoCl2} was added in each post-induction medium.
After three hours incubation at \SI{37}{\celsius}, 300 r.p.m., the cells were
harvested by using 4000 r.p.m centrifugation at \SI{4}{\celsius} for 15 minutes
and then resuspended with \SI{20}{\milli\Molar} Tris-HCl,
\SI{500}{\milli\Molar} \ce{NaCl}, \SI{5}{\milli\Molar} imidazole, 10\% glycerol
(pH 8.0) and \SI{1}{\micro\Molar} \ce{CoCl2}. Cell lysate was immediately
sonicated for 1.5 minutes at \SI{4}{\celsius} and then a clarification spin was
performed (20, 000 g, \SI{4}{\celsius}, 30 minutes).  Clarified supernatants
were loaded into a \SI{5}{\mL} His Trap column (G.E Healthcare, Piscataway, NJ)
using AKTA FPLC purifier (G.E.  Healthcare, Piscataway, NJ).  Protein elution
was generated using elution buffer B (\SI{20}{\milli\Molar} Tris-HCl,
\SI{500}{\milli\Molar} sodium chloride, \SI{500}{\milli\Molar} imidazole (pH
8.0)).  The purified samples were then transferred for buffer exchange using
\SI{12}{\L} \SI{20}{\milli\Molar} phosphate buffer (pH 8.0).  Dialyzed protein
was subjected to kinetic assays immediately.

\subsection{Lactose and Crystallization}

In a typical procedure, 500 mL of a deionized lactose21 solution (41.5 g/100mL
H2O) was added to 100 mL of an approximately 1–5mg/mL solution of the desired
biopolymer. Zn-cyt c was prepared according to the procedure of Fisher.22
Fluorescein derivatives were prepared as previously described.3 TR-lectin was
purchased from Molecular Probes. Solutions were incubated at 30 ?C for 15–36 h
until crystals of a suitable size were obtained. The crystals were harvested
and washed with hexanes. Typical crystals, 2 to 5 MM in length, were observed
and photographed with a fluorescence microscope.

\subsection{Protein Conjugation}


\subsection{Enzyme Kinetics}

The protein was diluted to a final concentration of \SI{30}{\nano\Molar} in
\SI{20}{\milli\Molar} sodium phosphate (pH 8.0) by using the extinction
coefficient \SI{29280}{\per\Molar\per\cm}. Reactions were monitored
spectrophotometrically (Synergy H1, BioTek, Winooski VT) at \SI{405}{\nm} for
paraoxon (coefficient = \SI{17000}{\per\Molar\per\cm}).  Reactions for paraoxon
(\SIrange{13}{104}{\micro\Molar}) was done in 0.4\% methanol.
K\textsubscript{M} and k\textsubscript{cat} values were determined by a
Lineweaver-Burk plot.\cite{Baker2011b} The equation used is shown below
(Eq.~\ref{eqn:MM-chap2}): 
\begin{equation} 
    \frac{1}{v} =
    \frac{K\textsubscript{M}}{V\textsubscript{max}}\times\frac{1}{S} +
    \frac{1}{V\textsubscript{max}} 
    \label{eqn:MM-chap3}
\end{equation}
where S represents substrate concentration; K\textsubscript{M} represents the
substrate concentration at which the reaction rate is half of
V\textsubscript{max}. The data reported is the average of three trials and the
error represents the standard deviation of those trials.

\section{Results}

The lazy brown fox.

\section{Discussion}

% Damn lactose is working, but I am still sitting here writing bullshit. 

\printbibliography[heading=subbibliography]

\end{refsection}

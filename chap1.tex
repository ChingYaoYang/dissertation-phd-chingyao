\chapter{Improved Stability and Half-Life of Fluorinated Phosphotriesterase Using Rosetta}
\label{chap:rosetta}
\begin{refsection}

\section{Introduction}

\subsection{Rosetta and Protein Engineering}

% Introduce Rosetta and protein engineering. Also focus on protein engineering.
\label{sec:rosetta}

Computational tools are widely used for protein engineering [ref]. Rosetta
suite was first developed in University of Washington [ref]. Baker et al.
adapted this suite for prediction of three dimensional structure of proteins.
This suite provides a handful of protocols for analyzing and mutating protein
structures. The simulation replies heavily on knowledge-based potentials. It is
a suite of libraries and tools for macromolecular ligand docking, to
thermo-stabilize proteins, to design a hydrogen-bond network, to design novel
protein folds, to create novel protein interfaces, and to design enzymes,
including some containing unnatural amino acids in the active sites. 

\subsection{Phosphotriesterase}
\label{sec:pte}

% This section focuses on PTE.
PTE is a homodimeric protein composed of two monomers, each of which contains a
metallo-active site.Phosphotriesterase (PTE) are enzymes, which hydrolyze
organophosphates (OPs) as well as synthetic esters.\cite{Ghanem2005a} OPs are a
synthetic class of small molecule that irreversibly inactivate
acetylcholinesterase (AChE), disrupting neural transmission. AChE is an enzyme
that degrades the neurotransmitter, acetylcholine, at the neuromuscular
junction in the cholinergic nervous system. After the acetylcholine is
hydrolyzed, the synaptic transmission would be terminated. Inhibition of AChE
lead to hyper-stimulation from toxic accumulation of
acetylcholine.\cite{Soreq2001} Army also adapted this protein for chemical
weapons neutralization. \cite{Yang2014a}

\subsection{Incorporation of non-natural amino acids}
\label{sec:rsi}

Several methods have been developed for the incorporation of unnatural amino
acids into proteins: solid-phase synthesis (SPPS)\cite{Mahto2011}, in vivo and in vitro
site-specific incorporation, 16 and residue-specific incorporation (Fig.
1)1d, 17. In SPPS, activated amino acids are immobilized on a solid support and
synthesized step-by-step in the reactant solution. This method is convenient
for the introduction of functional groups into peptides, but it is still
restricted to the yield and the expense of peptides. For example, if each
coupling step has 99\% yield, a 26-amino acid peptide would be synthesized in
77\% final yield. To synthesis longer chain peptides and proteins bearing UAAs,
biosynthetic methods have been developed.  There exists two contemporary
methods to biosynthetically incorporate non-natural amino acids into proteins:
site-specific incorporation and residue-specific incorporation. Schultz and
their coworkers18 have developed a general approach for the in vitro synthesis
of proteins. The approach relies on the suppression of an amber termination
codon (UAG) in the mRNA by an amber suppressor tRNA charged with the amino acid
analog. This method has been well studied and developed in research of protein
structures and functions19. 


Methods to incorporate amino acid analogues site-specifically into proteins in
vivo greatly expand research of unnatural amino acids. We are not only able to
synthesize large amounts of protein, but capable of overcoming potential
problems including post translational modifications. An in vivo site-specific
method to incorporation UAAs was developed by Schultz and coworkers21. A stop
codon at the position of interest is encoded in the mRNA. For in vivo
site-specific UAA incorporation, an orthogonal aminoacyl-tRNA synthetase
charges an orthogonal tRNA with particular UAA, and the suppressor tRNA would
help the incorporation of UAA with recognition of a stop codon. As cells
contain 20 aminoacyl-tRNA synthetase/suppressor tRNA pairs, a new one is
required for the incorporation. An orthogonal aminoacyl-tRNA
synthetase/suppressor tRNA pair based on a TyrRS/tRNATyr pair in the
Methanococcus jannaschii has been engineered for use in E. coli for the
incorporation of tyrosine analogs21a.

As an alternative to site-specific incorporation, residue-specific
incorporation has been developed in which a natural amino acid is replaced with
an UAA. Auxotrophic strains or organisms that cannot biosynthesize a particular
natural amino acid, has been used to introduce multiple UAAs throughout the
protein sequence. UAAs that are isosteric to natural amino acids are capable of
being recognized by the natural aminoacyl-tRNA synthetase (aaRS), charging the
appropriate tRNA enabling the introduction of UAA into the protein sequence
without alteration of the biosynthetic machinery. However, to introduce UAAs
with gross differences from the natural amino acids, further engineering of the
aaRS is required. To incorporate refractory methionine analogs, Tirrell and
coworkers engineered additional copied of the methionyl-tRNA synthetase (MetRS)
by adding the MetRS gene under constitutive promotor22. Alternatively, Schimmel
and coworkers mutated editing pocket of valyl-tRNA synthetase (ValRS) to
facilitate the incorporation of analogs that normally would not be accepted by
endogenous aaRS23. Finally, Kast and coworkers generated a mutated
phenylalanyl-tRNA synthetase (PheRS), ePheRS* under a constitutive promoter,
with a large binding pocket (T251G) and showed relaxed specificity.24

\subsection{Fluorinated amino acids in proteins}
\label{sec:faa}

% Describe the properties of faa in the proteins.

Fluorinated amino acids (FAAs), represent a unique class of UAAs. They have
different bond energies, electron distributions, and hydrophobzgicity 26 as
compared to their hydrogenated counterparts. As we compare the structure of
fluorocarbon groups, t(Bh)e C-F bond is highly dipolar while the hydrocarbon is
less. The C-F bond is roughly 0.24Å longer than C-H bond26 (Table 1). While in
some cases the global replacement of hydrophobic amino acids with fluorinated
analogs has led to the stabilization of protein structure27, it has all been
shown that in some cases they can reduce the thermodynamic stability28.

\subsection{Scope of work}

The primary goals of this work were to done adapt Rosetta for
phosphotriesterase. Overall, with incorporation of \emph{p}FF into protein, we
will be able evaluate the performance of scoring function. In advance, we would
evaluate the shelf life and thermo-stability of phosphotriesterase.

\section{Methods}

\subsection{General}

All reagents were obtained from Alfa Aesar or Fisher
Scientific unless otherwise stated. This is where everything starts. Writing is
a extremely painful precess of it. Now, I would like to test this thing out
while I am writing my article. Now, this line is added from MacVim in my
Home\_iMac.There are so many issue to deal with at this point. One is the
format of it. and second is the problem of this. The line break seems weird at
this point. Need to work really hard to get this right.

\subsection{Recombinant gene construction}

Plasmid vectors containing the genes for the block protein polymer fusions were
synthesized.

\subsection{Protein Expression}

This section is about the protein expression.

\subsection{Thermo-stability and Stability of Phosphotriesterase}
\label{sec:thermo}

Here, we introduce the thermo-stability of phosphotriesterase.
\emph{p}FF-ECE, respectively. 

\subsection{Enzyme Kinetics}

wild-type counterparts. Moreover, fluorination yielded robust elastic network
formation for all three protein polymers at elevated temperatures.Kinetics will be done in 200

\section{Discussion}

While the secondary structure appears conserved with respect to diblock
variants, the supramolecular behavior appears to have been altered, suggesting

\subsection{Protein design}

In the realm of synthetic chemistry, there has been a long standing interest in
the physicochemical properties of fluorinated polymers. Self-assembly into
higher-order structures has gained particular focus, in the cases of
semifluorinated dendritic Janus particles and fluorinated amphiphiles, which
affect assemblies on the supramolecular scale in different ways, and
despite the early successes in the incorporation of fluorinated amino acids into
protein polymers, little has been accomplished in the field with respect to
material characterization. Our studies demonstrate that fluorinating biopolymers
cannot only impact the secondary structure and ${T_t}$, but, more importantly,
influence the supramolecular assemblies and mechanical properties. While these
fluorinated protein polymers exist as soft gels, the observed modifications to
the self-assembly and rheological properties from the incorporation of
non-natural amino acids provides a precedence and an opportunity for tuning
protein-based materials. This provides a novel and alternative route for tuning
smart materials that rely on gel mechanics, in the case of applications in
tissue engineering, and thermoresponsive transition, in the case of applications
in drug delivery. 

\subsection{Future work}

Further optimization devoted to the incorporation efficiencies of fluorinated
amino acids into these block copolymers is necessary for further practical
development of the compositions. More uniform incorporation will further promote
consistency in the observable physicochemical properties of the block
copolymers, notwithstanding any inherent stochastic behavior that embodies the
self-assembly processes. This may be carried out by either carrying out the
biosynthetic expression under stricter selective control, or by engineering the
expression system to one more based on orthogonal transcription technologies.

Beyond the improvement in incorporation methods, additional attention should be
devoted to the functional characteristics of COMP, as it exists as part of the
block copolymer. These observations on COMP - specifically its ability to 1)
oligomerize and 2) bind to small molecules - should be carried out at conditions
preceding and following thermo-actuated transitions, as they are measured in the
body of this work. This work may seek, for example, to correlate the thermal
transition of the elastin domain to the incrementation of COMP oligomerization
states, providing insight into the engineered application of elastin-like
peptides as oligomerization chaperones. Assessment of binding properties of the
COMP domain, and/or the other elements of the block copolymers, will further
promote these inventions toward \latin{in vitro} and \latin{in vivo}
applications; the challenge lies with adopting viable small molecule as payload
candidates.

\printbibliography[heading=subbibliography]

\end{refsection}

\chapter{Improved Stability and Half-Life of Fluorinated Phosphotriesterase
Using Rosetta} 
\label{chap:rosetta}

\begin{refsection}

\section{Introduction}

\subsection{Rosetta and Protein Engineering}
\label{sec:rosetta}

% Introduce Rosetta and protein engineering. Also focus on protein engineering.
Computational tools are widely used for protein engineering [ref]. Rosetta
suite was first developed in University of Washington [ref]. Baker et al.
adapted this suite for prediction of three: dimensional structure of proteins.
This suite provides a handful of protocols for analyzing and mutating protein
structures. The simulation replies heavily on knowledge-based potentials. It is
a suite of libraries and tools for macromolecular ligand docking, to
thermo-stabilize proteins, to design a hydrogen-bond network, to design novel
protein folds, to create novel protein interfaces, and to design enzymes,
including some containing unnatural amino acids in the active sites.

\subsection{Phosphotriesterase} 
\label{sec:pte}

% This section focuses on PTE.
PTE is a homodimeric protein composed of two monomers, each of which contains a
metallo-active site.Phosphotriesterase (PTE) are enzymes, which hydrolyze
organophosphates (OPs) as well as synthetic esters.\cite{Ghanem2005a} OPs are a
synthetic class of small molecule that irreversibly inactivate
acetylcholinesterase (AChE), disrupting neural transmission. AChE is an enzyme
that degrades the neurotransmitter, acetylcholine, at the neuromuscular
junction in the cholinergic nervous system. After the acetylcholine is
hydrolyzed, the synaptic transmission would be terminated. Inhibition of AChE
lead to hyper-stimulation from toxic accumulation of
acetylcholine.\cite{Soreq2001} Army also adapted this protein for chemical
weapons neutralization. \cite{Yang2014a}

\subsection{Incorporation of Non-natural Amino Acids}
\label{sec:rsi}

Several methods have been developed for the incorporation of unnatural amino
acids into proteins: solid-phase synthesis (SPPS)\cite{Mahto2011}, in vivo and
in vitro site-specific incorporation, 16 and residue-specific incorporation
(Fig. 1)1d, 17. In SPPS, activated amino acids are immobilized on a solid
support and synthesized step-by-step in the reactant solution. This method is
convenient for the introduction of functional groups into peptides, but it is
still restricted to the yield and the expense of peptides. For example, if each
coupling step has 99\% yield, a 26-amino acid peptide would be synthesized in
77\% final yield. To synthesis longer chain peptides and proteins bearing UAAs,
biosynthetic methods have been developed.  There exists two contemporary
methods to biosynthetically incorporate non-natural amino acids into proteins:
site-specific incorporation and residue-specific incorporation. Schultz and
their coworkers18 have developed a general approach for the in vitro synthesis
of proteins. The approach relies on the suppression of an amber termination
codon (UAG) in the mRNA by an amber suppressor tRNA charged with the amino acid
analog. This method has been well studied and developed in research of protein
structures and functions19. 

Methods to incorporate amino acid analogues site-specifically into proteins in
vivo greatly expand research of unnatural amino acids. We are not only able to
synthesize large amounts of protein, but capable of overcoming potential
problems including post translational modifications. An in vivo site-specific
method to incorporation UAAs was developed by Schultz and coworkers21. A stop
codon at the position of interest is encoded in the mRNA. For in vivo
site-specific UAA incorporation, an orthogonal aminoacyl-tRNA synthetase
charges an orthogonal tRNA with particular UAA, and the suppressor tRNA would
help the incorporation of UAA with recognition of a stop codon. As cells
contain 20 aminoacyl-tRNA synthetase/suppressor tRNA pairs, a new one is
required for the incorporation. An orthogonal aminoacyl-tRNA
synthetase/suppressor tRNA pair based on a TyrRS/tRNATyr pair in the
\emph{Methanococcus jannaschii} has been engineered for use in \emph{E. coli}
for the incorporation of tyrosine analogs21a.

As an alternative to site-specific incorporation, residue-specific
incorporation has been developed in which a natural amino acid is replaced with
an UAA. Auxotrophic strains or organisms that cannot biosynthesize a particular
natural amino acid, has been used to introduce multiple UAAs throughout the
protein sequence. UAAs that are isosteric to natural amino acids are capable of
being recognized by the natural aminoacyl-tRNA synthetase (aaRS), charging the
appropriate tRNA enabling the introduction of UAA into the protein sequence
without alteration of the biosynthetic machinery. However, to introduce UAAs
with gross differences from the natural amino acids, further engineering of the
aaRS is required. To incorporate refractory methionine analogs, Tirrell and
coworkers engineered additional copied of the methionyl-tRNA synthetase (MetRS)
by adding the MetRS gene under constitutive promotor22. Alternatively, Schimmel
and coworkers mutated editing pocket of valyl-tRNA synthetase (ValRS) to
facilitate the incorporation of analogs that normally would not be accepted by
endogenous aaRS23. Finally, Kast and coworkers generated a mutated
phenylalanyl-tRNA synthetase (PheRS), ePheRS* under a constitutive promoter,
with a large binding pocket (T251G) and showed relaxed specificity.24

\subsection{Fluorinated Amino Acids In Proteins} 
\label{sec:faa}

% Describe the properties of faa in the proteins.
Fluorinated amino acids (FAAs), represent a unique class of UAAs. They have
different bond energies, electron distributions, and hydrophobzgicity 26 as
compared to their hydrogenated counterparts. As we compare the structure of
fluorocarbon groups, t(Bh)e C-F bond is highly dipolar while the hydrocarbon is
less. The C-F bond is roughly 0.24 {\AA} longer than C-H bond26 (Table 1). While in
some cases the global replacement of hydrophobic amino acids with fluorinated
analogs has led to the stabilization of protein structure27, it has all been
shown that in some cases they can reduce the thermodynamic stability28.

\subsection{Scope of Work}

The primary goals of this work were to done adapt Rosetta for
phosphotriesterase. Overall, with incorporation of \emph{p}FF into protein, we
will be able evaluate the performance of scoring function. In advance, we would
evaluate the shelf life and thermo-stability of phosphotriesterase.

\section{Methods}

\subsection{General}

All chemicals, reagents, and substrate were purchased from Sigma. T4 DNA ligase
was purchased from Roche. DNA sequence was confirmed by Eurofins MWG Operon.
96-well plates were purchased from Thermo Fisher Scientific (Waltham, MA).

\subsection{Recombinant Gene Construction}

pQE30-S5 was used as described before.\cite{Baker2011} The pQE30-104A plasmid
was prepared with forward primers (5’-GATGTGTCGACTGCCGATATCGGTCG-3’, Fisher
Scientific), reverse primers (5’-CGACCGATATCGGCAGTCGACACA-3’, Fisher
Scientific). The PCR parameters were set as follow for 18 cycles: initial
denaturation in \SI{95}{\celsius} for 30 seconds, sequential denaturation in
\SI{95}{\celsius} for 30 seconds, annealing in \SI{55}{\celsius} for 1 minute,
and extension in \SI{68}{\celsius} for 4 minutes. The mixture was then
incubated \SI{37}{\celsius} overnight with DpnI to digest methylated parent DNA
strands, which lack the desired mutation. DNA sequence was further confirmed by
Eurofins MWG Operon.

\subsection{Protein Expression}
Mutant and wild type plasmids were transformed into \latin{E. coli} phenylalanine
auxotrophic strains (AF-IQ cells).[5] Electroporation was done at
\SI{25}{\micro\farad}, \SI{100}{\ohm}, 2.5 kV (Biorad Gene Pulser II). Cells were
plated on agar plates containing 200 μg/mL ampicillin, 34 μg/mL
chloramphenicol. A Single colony was picked and grown in medium (M9 medium
supplemented with 0.2 wt \% glucose, 35 mg/L thiamine, 1mM \ch{MgSO4}, 0.1 mM
\ch{CaCl2}, 200 μg/mL ampicillin, and 34 μg/mL chloramphenicol) with 20 mg/L of
20 amino acids at \SI{37}{\celsius}, 300 r.p.m.  Afterwards, 250 mL of M9
medium for large-scale expression was innoculated 1:50 with an overnight
culture. After optical density reached 1.0 at 600 nm, media shift was carried
out by washing the cells three times with 0.9\% \SI{4}{\celsius} \ch{NaCl}.
Cells were then transferred to M9 minimal medium containing either 20 amino
acids or 19 amino acids (-Phe). \emph{p}FF-PTE and \emph{p}FF-104A expression
media were supplemented with and 3 mM of \emph{p}FF and 1 mM
isopropyl-$\beta$-D-thiogalactopyranoside (IPTG) to induce protein expression.
1mM of \ch{CoCl2} was added in each post-induction medium. After three hours
incubation at 37 °C, 300 r.p.m., the cells were harvested and then resuspended
with 20 mM Tris-HCl, 500 mM \ch{NaCl}, 5 mM imidazole, 10\% glycerol (pH 8.0)
and \SI{1}{\micro\moLar} \ch{CoCl2}. Cell lysate was sonicated on ice for 1.5
minutes and then a clarification spin was performed (20, 000 g,
\SI{4}{\celsius}, 30 min).  Clarified supernatants were loaded into a His
Trap column (G.E Healthcare, Piscataway, NJ) using ÄKTA FPLC purifier (G.E.
Healthcare, Piscataway, NJ).  Protein elution was generated using elution
buffer B (20 mM Tris- HCl, 500 mM sodium chloride, 500 mM imidazole (pH 8.0)).
The purified samples were then transferred for buffer exchange using
\SI{12}{\liter} 20 mM phosphate buffer (pH 8.0).  Dialyzed protein was
subjected to kinetic assays immediately.

\subsection{Thermo-stability and Secondary Structure of Phosphotriesterase}
\label{sec:thermo}

\subsubsection{Nano-DSC}

DSC (Nano-DSC, TA instrument, USA) was preformed by using 600 μL (0.1 mg/mL) of
protein right after dialysis. Measurements were conducted at a scan rate of
\SI{1}{\celsius}/min. Signals was blanked with buffer under the same condition.
The observed diagram was then analyzed by using NanoAnalyze software.

\subsubsection{Circular Dichroism}

CD spectra were recorded on a JASCO J-815 Spectropolarimeter (Easton, MD) using
Spectra Manager software. Temperature was controlled using a Fisher Isotemp
Model 3016S water bath. Proteins concentrations were 10 μM in 20 mM phosphate
buffer (pH 8.0). 20 mM phosphate buffer was used for blanking signals. To
calculate ellipticities, the following formula was used:
% use equation here
θmrw = MRW(θobs) / (10 * c * l)
where \emph{MRW} is the mean residue weight of the specific phosphotriesterase, θobs
is the observed ellipticities (mdeg), \emph{l} is the path length (cm), \emph{c} is the
concentration in μM. Spectra was recorded from 190 nm to 250 nm with a scan
speed of 1 nm/min.

\subsection{Enzyme Kinetics}

The protein was diluted to a final concentration of 30 nM in 20 mM sodium
phosphate (pH 8.0) by using the extinction coefficient 29,280
M\textsuperscript{-1} cm\textsuperscript{-1}. Reactions were monitored
spectrophotometrically (Synergy H1, Bio-Tek, Winooski VT) at 405 nm for
paraoxon (coefficient = 17,000 M\textsuperscript{-1}  cm\textsuperscript{-1} ).
Reactions for paraoxon (13 – 104 μM) was done in 0.4\% methanol. KM and kcat
values were determined by a Lineweaver-Burk plot (1/v vs 1/[S]).[5] The
equation used is shown below: 
% use equation here 
1/v = (KM/ VMax) ? (1/[S]) + 1/VMax S2
where [S] represents substrate concentration; KM represents the substrate
concentration at which the reaction rate is half of Vmax. The data reported is
the average of three trials and the error represents the standard deviation of
those trials.

\subsection{MALDI-TOF Mass Spectrometry}

To determine level of \emph{p}FF incorporation, \SI{20}{\micro\liter} of
purified PTE pFF-PTE, F104A, or \emph{p}FF-104A was incubated with 12.5 ng/μL
of trypsin solution (in 50 mM of ammonium bicarbonate) at \SI{37}{\celsius}
overnight. 2 μL of 10\% trifluoroacetic acid (TFA) was used to quench each
reaction. Reaction was then purified with a C\textsubscript{18} packed zip-tip
(Millipore, Billerica, MA).  Tips were wetted in 50\% acetonitirile (ACN),
equilibrated in 0.1\% TFA, and eluted with 0.1\% TFA in 75\% ACN. Matrix was
dissolved in 10 mg/mL $\alpha$-cyano-4-hydrocinnamic acid (CCA) in 50\% ACN,
0.05\% TFA. Theoretical trypsin digest were calculated from Peptide Mass
(www.expasy.org/tools/peptide-mass.html). Samples were added to the matrix at a
1:1 ratio and spotted on MALDI plate. Five standards were spotted separately
for calibration: angiotensin I (MW = 1295.69 g/mol), neurotensin (MW =
1671.92g/mol), ACTH (1-17) (MW = 2092.09 g/mol), ACTH (18-39) (MW = 2464.20
g/mol), and ACTH (7-38) (MW =3656.93 g/mol). Compass 1.4 for flex software was
then used to analyze the MALDI spectra (www.bruker.com/).

\section{Results and Discussion}

\subsection{Biosynthesis of phosphotriesterase}

The \emph{p}FF-F104A variant and the \emph{p}FF-PTE parent were biosynthesized
by residue-specific incorporation with the phenylalanine auxotrophic
\emph{Escherichia coli} strain AFIQ.[6] As controls, the non-fluorinated
counterparts, PTE and F104A, were expressed under conventional conditions. As
expected, all four proteins exhibited good expression in the presence of
phenylalanine or \emph{p}FF. \emph{p}FF-F104A and \emph{p}FF-PTE exhibited 80
and 92\% incorporation, respectively, as determined by MALDI-TOF mass
spectrometry. Notably, purified yields of \emph{p}FF-F104A were twofold higher
than for \emph{p}FF-PTE, thus indicating more soluble protein yield.

\subsection{Thermo-stability And Secondary Structure}

Circular dichroism (CD) was performed to determine whether the mutation had an
impact on the overall secondary structure and stability. Far-UV wavelength
scans of \emph{p}FF-F104A and \emph{p}FF-PTE showed a double minimum at 208 and
222 nm (\SI{25}{\celsius}), as expected for a
($\beta$/$\alpha$)\textsubscript{8}-barrel protein, thus suggesting that the
mutation did not affect the overall structure (Figure S3).  Surprisingly,
comparison of the non-fluorinated counterparts revealed that F104A was less
structured than PTE (Figure S3). To assess the stability, differential scanning
calorimetry (DSC) was performed (Figure 2). Upon heating the sample from 0 to
\SI{70}{\celsius}, \emph{p}FF-PTE exhibited two transitions
(T\textsubscript{m}1: 42.0 $\pm$ 0.1; T\textsubscript{m}2 : 48.6 ± 0.2 ; Table
S1); this is consistent with our previous studies.[6] This biphasic unfolding
was also observed by Grimsley et al. in a study of organophosphorus hydrolase
(EC 3.1.8.1), and was attributed to the presence of a dimeric unfolded
intermediate.[17] In contrast, \emph{p}FF-F104A exhibited a single transition
at (49.7 ± 0.2) 8C, which was higher than both \emph{p}FF-PTE values (by 7.7
and 1.18C; Figure 2B, Table S1).  Remarkably, after heating, pFF-F104A retained
the single Tm of (49.2 ± 0.1) \SI{8}{\celsius}, thus demonstrating regaining of
structure after undergoing thermal unfolding.  In the absence of \emph{p}FF,
F104A demonstrated two transitions similar to \emph{p}FF-PTE (Figure 2A), thus
suggesting that fluorination was critical for stability.  These data
demonstrate the overall thermodynamic stability of \emph{p}FF-F104A.

\subsection{Enzymatic Kinetics of PTE}

\subsection{Protein Design}

Although methods enabling the biosynthesis of artificial proteins bearing NCAAs
are abundant,[1] tools to help further improve the overall activity and
stability are needed. Mutagenesis and evolutionary approaches have been
employed successfully to identify variants with enhanced function; however,
these rely heavily on testing or screening several to millions of
constructs.[3c, 5, 20] We demonstrate the use of computational methods to
identify a fluorinated protein variant that exhibits superior heat stability
and half-life. Notably, the \emph{p}FF-F104A variant is only functional in the
fluorinated form, thus validating Rossetta-based design with \emph{p}FF. This
provides another useful tool for protein design and could be employed in
conjunction with the abov ementioned approaches.

\subsection{Future Work}

\printbibliography[heading=subbibliography]

\end{refsection}

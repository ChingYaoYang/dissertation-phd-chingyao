\chapter{Improved Stability and Half-Life of Fluorinated Phosphotriesterase Using Rosetta}
\label{chap:rosetta}
\begin{refsection}

\section{Introduction}

\subsection{Rosetta and Protein Engineering}

% Introduce Rosetta and protein engineering. Also focus on protein engineering.
\label{sec:rosetta}

Computational tools are widely used for protein engineering [ref]. Rosetta
suite was first developed in University of Washington [ref]. Baker et al.
adapted this suite for prediction of three dimensional structure of proteins.
This suite provides a handful of protocols for analyzing and mutating protein
structures. The simulation replies heavily on knowledge-based potentials.\cite{Singh2015, Yang2014a}

\subsection{Phosphotriesterase}
\label{sec:pte}

% This section focuses on PTE.
Phosphotriesterase is a homo-dimeric protein. Previously, this protein has been
intensively used for detosification. Raushel group had been used directed
evolution method to modify the protein. [ref] Army also adapted this protein
for chemical weapons neutralization. \cite{Yang2014a}

\subsection{Incorporation of non-natural amino acids}
\label{sec:rsi}

While the physicochemical properties of protein-based materials have been
modulated with extraordinary capacity and range - in large part due to
the non-covalent interplay of amino acid residues - they are still limited by
the constrained 22 amino acids that define functional group diversity.
This limitation has been addressed, and in many ways deprecated, by various
efforts in genetic engineering.

There exists two contemporary methods to biosynthetically incorporate
non-natural amino acids into proteins: site-specific incorporation and
residue-specific incorporation.

\subsection{Fluorinated amino acids}
\label{sec:faa}

% What's the earliest precedence found for fluorinated amino acids and the
Fluorine exists naturally in all type of materials. People were using fluorine
for amino acids a lot.

It has been shown that fluorinated amino acids stabilize helix peptide.

\subsection{Scope of work}

The primary goals of this work were to evaluate the incorporation of the
the extent of macroscopic assembly occurring as a result of temperature-induced
transitions, the mechanical nature of which was probed by microrheology. 

\section{Methods}

\subsection{General}

All reagents were obtained from Alfa Aesar or Fisher
Scientific unless otherwise stated. This is where everything starts. Writing is
a extremely painful precess of it. Now, I would like to test this thing out
while I am writing my article. Now, this line is added from MacVim in my
Home\_iMac.There are so many issue to deal with at this point. One is the
format of it. and second is the problem of this. The line break seems weird at
this point. Need to work really hard to get this right.

\subsection{Recombinant gene construction}

Plasmid vectors containing the genes for the block protein polymer fusions were
synthesized. 

\subsection{Protein Expression}

This section is about the protein expression.

\subsection{Thermo-stability and Stability of Phosphotriesterase}
\label{sec:thermo}

Here, we introduce the thermo-stability of phosphotriesterase.
\emph{p}FF-ECE, respectively.

\subsection{Enzyme Kinetics}

wild-type counterparts. Moreover, fluorination yielded robust elastic network
formation for all three protein polymers at elevated temperatures. 

\section{Discussion}

While the secondary structure appears conserved with respect to diblock
variants, the supramolecular behavior appears to have been altered, suggesting

\subsection{Protein design}

In the realm of synthetic chemistry, there has been a long standing interest in
the physicochemical properties of fluorinated polymers. Self-assembly into
higher-order structures has gained particular focus, in the cases of
semifluorinated dendritic Janus particles and fluorinated amphiphiles, which
affect assemblies on the supramolecular scale in different ways, and
despite the early successes in the incorporation of fluorinated amino acids into
protein polymers, little has been accomplished in the field with respect to
material characterization. Our studies demonstrate that fluorinating biopolymers
cannot only impact the secondary structure and ${T_t}$, but, more importantly,
influence the supramolecular assemblies and mechanical properties. While these
fluorinated protein polymers exist as soft gels, the observed modifications to
the self-assembly and rheological properties from the incorporation of
non-natural amino acids provides a precedence and an opportunity for tuning
protein-based materials. This provides a novel and alternative route for tuning
smart materials that rely on gel mechanics, in the case of applications in
tissue engineering, and thermoresponsive transition, in the case of applications
in drug delivery. 

\subsection{Future work}

Further optimization devoted to the incorporation efficiencies of fluorinated
amino acids into these block copolymers is necessary for further practical
development of the compositions. More uniform incorporation will further promote
consistency in the observable physicochemical properties of the block
copolymers, notwithstanding any inherent stochastic behavior that embodies the
self-assembly processes. This may be carried out by either carrying out the
biosynthetic expression under stricter selective control, or by engineering the
expression system to one more based on orthogonal transcription technologies.

Beyond the improvement in incorporation methods, additional attention should be
devoted to the functional characteristics of COMP, as it exists as part of the
block copolymer. These observations on COMP - specifically its ability to 1)
oligomerize and 2) bind to small molecules - should be carried out at conditions
preceding and following thermo-actuated transitions, as they are measured in the
body of this work. This work may seek, for example, to correlate the thermal
transition of the elastin domain to the incrementation of COMP oligomerization
states, providing insight into the engineered application of elastin-like
peptides as oligomerization chaperones. Assessment of binding properties of the
COMP domain, and/or the other elements of the block copolymers, will further
promote these inventions toward \latin{in vitro} and \latin{in vivo}
applications; the challenge lies with adopting viable small molecule as payload
candidates.

kinetically-driven assembly.

\printbibliography[heading=subbibliography]

\end{refsection}

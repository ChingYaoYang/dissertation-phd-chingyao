% $Log: abstract.tex,v $
% Revision 1.1  93/05/14  14:56:25  starflt
% Initial revision
% 
% Revision 1.1  90/05/04  10:41:01  lwvanels
% Initial revision
% 
%
%% The text of your abstract and nothing else (other than comments) goes here.
%% It will be single-spaced and the rest of the text that is supposed to go on
%% the abstract page will be generated by the abstractpage environment.  This
%% file should be \input (not \include 'd) from cover.tex.

Flavobacterium phosphotriesterase (PTE) is capable of robustly neutralizing
various organophosphate (OPs) compounds. Herein multiple protein engineering
strategies are employed to further optimize the catalytic performance of this
enzyme to hydrolyze OPs. Specifically, \emph{in vitro} correlations between structure,
temperature, and activity gauge the fidelity of \emph{in silico} designs using
Rosetta.  Methods of tuning the structure-function relationship, and thereby
performance, of the enzyme include the incorporation of the unnatural amino
acid and site-directed mutagenesis of rationally selected residues. In Chapter
I, an evaluation on the effects of a unnatural amino acid,
\emph{para}-fluorophenylalanine (\emph{p}FF), is presented.  Using the 
computational modeling suite, Rosetta, a designed variant has demonstrated
extended half-life and thermoactivity toward its substrate, paraoxon.  Chapter
II, an alternative strategy using a sequence alignment of PTE is performed.
Eight phenylalanines outside the dimer interface are evaluated experimentally
and computationally. The structural comparisons of these eight variants
demonstrate the promise for prediction of stability and function. Chapter III,
to formulate such enzyme for an application, lactose monohydrate is
co-crystallized with PTE, leading to a further extension of shelf life.  The
strategies described here demonstrate a combination of protein engineering
methods to develop a progressive product for combating OP-poisoning. 


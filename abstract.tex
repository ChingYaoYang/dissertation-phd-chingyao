% $Log: abstract.tex,v $
% Revision 1.1  93/05/14  14:56:25  starflt
% Initial revision
% 
% Revision 1.1  90/05/04  10:41:01  lwvanels
% Initial revision
% 
%
%% The text of your abstract and nothing else (other than comments) goes here.
%% It will be single-spaced and the rest of the text that is supposed to go on
%% the abstract page will be generated by the abstractpage environment.  This
%% file should be \input (not \include 'd) from cover.tex.
With the advancement of technologies to probe and manipulate biophysical matter,
the scientific community continues to ever better engineer biological systems with the
complexity and elegance in design that is necessary to address 
biomedical challenges. The growing maturity of the field of
protein engineering is a testament to this proclamation.
%Within this field,
%modification of naturally occurring proteins as a manner of rests the 
Herein, two fundamental ideas are explored. In Chapter I, an evaluation is
presented on the effects of the incorporation of a non-canonical, fluorinated amino
acid into a protein-based block copolymer. The ramifications of these results,
and similar others in the field, on the promise for predictable tuning of the
physicochemical behavior and properties of protein-based materials are
emphasized.
In Chapter II, an alternative application of an endogenous protein is
examined, harnessing its inherent form and function. Hypotheses postulate
the ability of a coiled-coil protein, of particularly high oligomeric order, to
facilitate the delivery of small molecule therapeutics for the treatment of
osteoarthritis, whilst addressing dominant hurdles pertaining to drug
localization.
This complete body of work rests on the themes of control and repurposed
application of biophysical matter, contributing to the formalization of
engineered systems within protein science.

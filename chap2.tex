\chapter{Effects of Phenylalanines Outside Dimer Interface of Phosphotriesterase}
\label{chap:dimer}
\begin{refsection}

\section{Introduction}

\subsection{Phosphotriesterase}

\subsection{Dimer Interface of Phosphotriesterase}

\subsection{Side-chain Effects}

\section{Methods}

\subsection{General}

All chemicals, reagents, and substrate were purchased from Sigma. T4 DNA ligase
was purchased from Roche. DNA sequence was confirmed by Eurofins MWG Operon.
96-well plates were purchased from Thermo Fisher Scientific (Waltham, MA)\cite{Yang2014a}.

\subsection{Variants of Phosphotriesterase}
Purified protein product was assayed for concentration by way of a Thermo

\subsection{Biosynthesis}

In anticipation for the need of large quantities of protein mass for the \ldots
delete comtent of the supernatant and storage at \SI{-20}{\celsius}.

\subsection{Protein Purification}
The purification was described previously in the section
\ref{sec:protein-expression-method}. All solutions used in the extraction and
purification of recombinant proteins \ldots delete content 5 CVs of buffer
prior to each injection.

\subsection{Enzyme Kinetics}

The protein was diluted to a final concentration of \SI{30}{\nano\Molar} in
\SI{20}{\milli\Molar} sodium phosphate (pH 8.0) by using the extinction
coefficient \SI{29280}{\per\Molar\per\cm}. Reactions were monitored
spectrophotometrically (Synergy H1, BioTek, Winooski VT) at \SI{405}{\nm} for
paraoxon (coefficient = \SI{17000}{\per\Molar\per\cm}).  Reactions for paraoxon
(\SIrange{13}{104}{\micro\Molar}) was done in 0.4\% methanol.
K\textsubscript{M} and k\textsubscript{cat} values were determined by a
Lineweaver-Burk plot.\cite{Baker2011b} The equation used is shown below
(Eq.~\ref{eqn:MM-chap2}): 
\begin{equation} 
    \frac{1}{v} =
    \frac{K\textsubscript{M}}{V\textsubscript{max}}\times\frac{1}{S} +
    \frac{1}{V\textsubscript{max}} 
    \label{eqn:MM-chap2}
\end{equation}
where S represents substrate concentration; K\textsubscript{M} represents the
substrate concentration at which the reaction rate is half of
V\textsubscript{max}. The data reported is the average of three trials and the
error represents the standard deviation of those trials.

\subsection{Thermo-stability and Secondary Structure of Phosphotriesterase}

\subsubsection{Nano-DSC}

The details are described in the section \ref{sec:dsc-method}. DSC (Nano-DSC,
TA instrument, USA) was preformed by using \SI{600}{\micro\L}
(\SI{0.1}{\mg\per\mL}) of protein right after dialysis. Measurements were
conducted at a scan rate of \SI{1}{\celsius\per\minute}. Signals was blanked with
buffer under the same condition.  The observed diagram was then analyzed by
using NanoAnalyze software.

\subsubsection{Circular Dichroism}

The details are described in the section \ref{sec:cd-method}. CD spectra were
recorded on a JASCO J-815 Spectropolarimeter (Easton, MD) using Spectra Manager
software. Temperature was controlled using a Fisher Isotemp Model 3016S water
bath. Proteins concentrations were \SI{10}{\micro\Molar} in
\SI{20}{\milli\Molar} phosphate buffer (pH 8.0). \SI{20}{\milli\Molar}
phosphate buffer was used for blanking signals. To calculate ellipticities, the
following formula was used(Eq.~\ref{eqn:CD-chap2}): 
\begin{equation}
    θmrw = MRW(θobs) / (10 * c * l)
    \label{eqn:CD-chap2}
\end{equation}
where \emph{MRW} is the mean residue weight of the specific phosphotriesterase,
θobs is the observed ellipticities (mdeg), \emph{l} is the path length (cm),
\emph{c} is the concentration in \SI{}{\micro\Molar}. Spectra was recorded from
\SIrange{190}{250}{\nm} with a scan speed of \SI{1}{\nano\meter\per\minute}.

\section{Results}

\subsection{DNA Alignments And PTE Variants}

Expression of COMPcc was carried out in auto-inducing media, encompassing a
\ldots delete content typical in inducible expression systems.

Auto-induction growth can be sustained in baffled shaker flasks, according to
\ldots delete content to purification.

\subsection{Effects of COMPcc on BMS493 therapy}
\label{sec:bms493_results}

\ldots lots content delete here\ldots potential for COMPcc to behave as an
inhibitor of hypertrophic differentiation.

\subsection{Endotoxin levels in COMPcc protein}
\label{sec:endotoxins}

Endotoxin levels of the protein were measured using a limulus amebocyte lysate
catabolic events reported in Section \ref{sec:bms493_results}.

\section{Discussion}
% --------------------------
\subsection{Contamination of recombinant COMPcc with endotoxins}
The importance of endotoxin removal from recombinant protein preparations has
\ldots lots content deleted \ldots COMPcc may provide additional benefit to the
delivery of RAR/RXR antagonists and inverse agonists as demonstrated by early
evidence of down-regulating hypertrophic gene expression and terminal
differentiation.

\subsection{Meso-scale features of COMPcc promote intra-articular delivery
applications}
\label{sec:invivo_fate}
small molecules and macromolecules, including Paclitaxel,\ Mefoxin,

\subsection{Future work}
% --------------------------
\label{sec:future_work_endotoxin}
These preliminary results suggest that the COMPcc does not significantly improve
the therapeutic effects of BMS493. The signal from these experiments suffers,
however, from potential noise originating from the presence of endotoxins in the
sample, possibly eliciting similar catabolic gene expression to what is
observed. While dosage experiments were performed in this study of COMPcc and
COMPcc that had undergone endotoxin removal procedures, additional control
samples can improve the confidence of these results. While LPS have been shown
to induce catabolic gene expression, known standards of LPS may be administered
to chondrocytes in future studies to gauge actual response from a positive
control. In addition, purified lysate from non-transformed BL21 (DE3)
\latin{E.coli} that undergoes endotoxin removal procedures may be used and
considered as a negative control sample group toward the testing of COMPcc
alone.

these conditions. A balance must be stricken between the solvation of BMS493 (or
any other hydrophobic/non-polar small molecule compound) and the (molecular and
meso-scale) structural integrity of the protein-based carrier.

\printbibliography[heading=subbibliography]

\end{refsection}

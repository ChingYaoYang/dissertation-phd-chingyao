\chapter{Effects of Phenylalanines Outside Dimer Interface of Phosphotriesterase}
\label{chap:dimer}
\begin{refsection}

\section{Introduction}

\subsection{Phosphotriesterase}

PTE is a homodimeric protein composed of two monomers, each of which contains a
metallo-active site. Phosphotriesterase (PTE) are enzymes, which hydrolyze
organophosphates (OPs) as well as synthetic esters (Figure
\ref{fig:pte-structure})\cite{Ghanem2005a}. The proenzyme form of PTE contains
29 amino acids signal peptide at the N-terminus. It is originally found as a
39kDa monomeric form in the solution\cite{Mulbry1989}. Later, the proenzyme of
PTE is engineered and expressed in the form of mature protein from \latin{E.
coli}. A ($\beta$/$\alpha$)\textsubscript{8} TIM-barrel structure forms the
monomeric PTE\cite{Roodveldt2005,Seibert2005}. The globular monomer is roughly
51\AA $\times$ 55\AA $\times$ 51\AA.  OPs are a synthetic class of small molecule
that irreversibly inactivate acetylcholinesterase (AChE), disrupting
neural transmission. AChE is an enzyme that degrades the neurotransmitter,
acetylcholine, at the neuromuscular junction in the cholinergic nervous system.
After the acetylcholine is hydrolyzed, the synaptic transmission would be
terminated. Inhibition of AChE lead to hyper-stimulation from toxic
accumulation of acetylcholine\cite{Soreq2001}. Army also adapted this protein
for chemical weapons neutralization \cite{Yang2014a}.

\subsection{Protein Stability}

Protein stabilities have been studied in decades. Shoichet \latin{et al.}
determined the relationship between stability and functionality of T4 lysozyme
via mutagenesis\cite{Shoichet1995}. Previously, Baker \latin{et al.}
incorporated \emph{p}-fluoro-phenylalanine into PTE and demonstrated the
enhanced residual activity among its substrates, paraoxon, chlorpyrifos, and
2-naphthyl-acetate (2NA)\cite{Baker2011b}.  

\subsection{Active Site of Phosphotriesterase}

Solving the structure of a PTE containing both \ch{Cd} and
\ch{Zn} in the PTE structure was studied with NMR\cite{Benning2001a}. The
$\alpha$ metal is coordinated to the enzyme by His55 and His57 from the
end of $\beta$-strand 1 as well as by Asp301 from $\beta$-strand 8. The
$\beta$-site metal is coordinated through His201 and His230. In addition, the
metals are coordinated through the bridging ligands identified as a
carboxylated Lys169.

% --------------------------
\begin{figure}[h!] 
    \centering 
    \includegraphics[width=0.5\textwidth]{fig2_03}
    \caption[Metal
        coordination of resting state of PTE (PDB 1HZY). Bond distances from
        metal to ligand are D301--$\alpha$ = 2.2\AA, H57--$\alpha$ = 2.1\AA,
        H55--$\alpha$ = 1.8\AA, K169-–$\alpha$ = 2.1\AA, K169-–$\beta$ =
        2.0\AA, \ce{H2O}-–$\beta$ = 2.1\AA, H230-–$\beta$ = 2.1\AA,
    H201-–$\beta$ = 2.2\AA.]{Metal coordination of resting state of PTE (PDB
        1HZY). Bond distances from metal to ligand are D301--$\alpha$ = 2.2\AA,
        H57--$\alpha$ = 2.1\AA, H55--$\alpha$ = 1.8\AA, K169-–$\alpha$ =
        2.1\AA, K169-–$\beta$ = 2.0\AA,
    \ce{H2O}-–$\beta$ = 2.1\AA, H230-–$\beta$ = 2.1\AA, H201-–$\beta$ = 2.2\AA.} 
    \label{fig:pte-active-site-chap2}
\end{figure}
% --------------------------

The reaction catalyzed by PTE was described as an SN2-like mechanism. The metals in the
active site are coordinated to the majority of the residues from the C-terminal
loops. The active site pocket is designated to \emph{small}, \emph{large}, and
\emph{leaving group} pockets. The leaving group pocket consists of residues
including Trp131, Phe132, Phe306, and Tyr309; small group pocket are Gly60, Ile106,
Leu303, Ser308; the large pocket are Typ131, Phe132, Phe306, Tyr309.

\subsection{Side-chain Effects}
\label{sec:side-chain}

Several residues across the PTE sequence have been studied for enhancement of
stability \cite{Baker2011b,Yang2014a} or changes of
selectivities\cite{Bigley2013b,Chen-Goodspeed2001a,Pavelka2009}. Rauchel group
using the alanine scanning method to locate key residues for PTE hydrolysis
efficiency\cite{Chen-Goodspeed2001a}. As Ile106, Ser306 were swapped with
alanine, they found the mutations increased substrates hydrolysis
efficiency, including R\textsubscript{P}-enantiomers containing a phenyl
substituent. Due to the sterically hindering effect, they suggested that small
pocket of PTE dictated the chiral preference for the
S\textsubscript{P}-enantiomers. Interestingly, enlarging the large pocket site
did not significantly increase the hydrolysis efficiency. However, they did
note that H254A in the large pocket sites might interacted differently via
Asp301 as they were binding with $\alpha$ metal. 

In this article, we would like to investigate the effects of phenylalanines
outside the dimer interface of phosphotriesterase. While Baker \latin{et al.}
demonstrated how unnatural amino acid (\emph{p}-fluoro-phenylalanine)
stabilized PTE, we mutate phenylalanines positions to study the individual
effect on PTE. With the comparison of paraoxon hydrolysis of F51L, F150M,
F216L, F304L, F306L, F327L, F335M, and F357L, we also report and evaluate the
performance of computational modeling tool, Rosetta, for the future screening
and analysis of PTE.

\section{Methods}

\subsection{General}

\emph{DpnI} and dNTP were purchased from Roche. All other chemicals, including
\ch{NaCl}, paraoxon, sodium phosphates monobasic, sodium phosphate dibasic, were
purchased from Sigma or VWR. DNA primers for mutagenesis were purchased from
Fisher Scientific. DNA sequence was confirmed by Eurofins MWG Operon.  96-well
plates were purchased from Thermo Fisher Scientific (Waltham,
MA)\cite{Yang2014a}.

\subsection{Rosetta Design of Phosphotriesterase}

A symmetric starting model of wild type PTE from the B chain of PDB structure
1HZY\cite{Benning2001a} was built using the Rosetta suite of macromolecular
modeling tools\cite{Leaver-Fay2011}. Three positions in the
wild-type PTE sequence were mutated (K185R, D208G, and R319S) based on
S5PTE\cite{Roodveldt2005}. Both active site \ce{Zn^{2+}} ions were
replaced with \ce{Co^{2+}} to reflect the metal used in the experimentally
produced mutants.  Distance constraints between the cobalt cations and the
coordinating residues were taken from PDB structure 3A4J\cite{Jackson2009b}.
Torsional and partial charge parameterSs for the non-standard carboxylated
lysine residue (Lys 169) were calculated quantum mechanically using the
HF/6-31G(d) level of theory in Gaussian09\cite{Frisch2009a} with an overall
charge of -1.  Rotamer libraries for the carboxylated lysine were generated
with the Rosetta MakeRotLib\cite{Renfrew2012b} protocol.  Models were
constructed for each of the point mutations: F51L, F150M, F216L, F304L, F306L,
F327L, F335M, and F357L using the Rosetta fixbb (fixed backbone design)
protocol with symmetry\cite{DiMaio2011a}.  \ce{Co^{2+}} coordinating residues
were held fixed to their native rotamers. To propagate point mutation effects
throughout a mutant model, the Rosetta relax protocol was used to repack and
minimize the entire PTE structure with backbone flexibility. For each point
mutant, an ensemble of 500 relaxed decoys were generated. Interatomic distances
between \ce{Co^{2+}} and coordinating residues were enforced with harmonic
constraints.  The change in stability for a mutation was calculated as the
difference between the mutant and wild type ensemble averages of the total
Rosetta score. All protocols used here included the native rotamers and extra
rotamers sampling as additional parameters. All decoys were scored using the
talaris2013 score function\cite{Leaver-Fay2013a}.
`
\subsection{Biosynthesis}

The original DNA sequence of PTE was obtained from Tawfik group. S5-PTE was
used for protein sequence BLAST. Three mutations were introduced into the
original PTE sequence. BLAST was used to identify different strains of PTE. The
original PTE can be found in the previous literatures\cite{Yang0214a}.
pQE30-PTE was used as described before\cite{Yang2014a}. The PTE variants
plasmid were prepared as the followings: F51L, forward primer
(5$'$-TCT GAA GCG GGT \emph{CTG} ACA CTG ACT CAC G-3$'$), reverse primer
(5$'$-G AGA CTT CGC CCA \emph{GAC} TGT GAC TGA GTG-3$'$). F150M, forward primer
(5$'$-TC ACA CAG TTC \emph{ATG} CTG CGT GAG ATT CAA TAT GGC-3$'$), reverse primer
(5$'$-CAT CTC CTT GAG TGT GTC AAG \emph{TAC} GAC GCA CTC TA-3$'$). F216L, forward primer
(5$'$-AG GCC GCC ATT \emph{TTA} GAG TCC GAA GG-3$'$), reverse primer
(5$'$-CGG CGG TAA \emph{AAT} CTC AGG CTT CCG A-3$'$). F304L, forward primer
(5$'$-AT GAC TGG CTG \emph{CTG} GGG TTT TCG AGC TAT GTC-3$'$), reverse primer
(5$'$-CAA AGC TTA CTG ACC GAC \emph{GAC} CCC AAA AGC TC-3$'$). F306L, forward primer
(5$'$-TGG CTG TTC GGG \emph{CTG} TCG AGC TAT GTC ACC-3$'$), reverse primer
(5$'$-CTG ACC GAC AAG CCC \emph{GAC} AGC TCG ATA CAG-3$'$). F327L, forward primer
(5$'$-AC GGG ATG GCC \emph{TTA} ATT CCA CTG AG-3$'$), reverse primer
(5$'$-CCC TAC CGG \emph{AAT} TAA GGT GAC TCT C-3$'$). F335M, forward primer
(5$'$-G AGA GTG ATC CCA \emph{CTG} CTA CGA GAG AAG G-3$'$), reverse primer
(5$'$-C TCT CAC TAG GGT \emph{GAC} GAT GCT CTC TTC C-3$'$). F357L, forward primer
(5$'$-T AAC CCG GCG CGG \emph{TTA} TTG TC ACC GAC CTT GC-3$'$), reverse primer
(5$'$-GA TTG GGC CGC GCC \emph{AAT} AAC AGT GGC TGG AAC-3$'$). The polymerase
chain reaction (PCR) parameters were set as follow for 18 cycles: initial
denaturation in \SI{95}{\celsius} for 30 seconds, sequential denaturation in
\SI{95}{\celsius} for 30 seconds, annealing in \SI{55}{\celsius} for 1 minute,
and extension in \SI{68}{\celsius} for 4 minutes. The mixture was then
incubated \SI{37}{\celsius} overnight with DpnI to digest methylated parent DNA
strands, which lack the desired mutation. DNA sequence was further confirmed by
Eurofins MWG Operon.

\subsection{Protein Purification}

Mutant and wild type plasmids were transformed into electro-competent \latin{E.
coli} phenylalanine auxotrophic strains (AF-IQ cells). Electroporation was done
at \SI{25}{\micro\farad}, \SI{100}{\ohm}, 2.5 kV (Biorad Gene Pulser II).
Cells were plated on agar plates containing \SI{200}{\ug\per\mL} ampicillin,
\SI{34}{\ug\per\mL} chloramphenicol. A single colony was picked and grown in LB
with \SI{200}{\ug\per\mL} ampicillin, and \SI{34}{\ug\per\mL} chloramphenicol)
at \SI{37}{\celsius}, 300 r.p.m for 16 hours \SI{37}{\celsius} incubation.
Afterwards, \SI{250}{\mL} of LB medium for large-scale expression was
innoculated 1:50 with the overnight culture.  After optical density reached 1.0
at 600 nm, the expression media were supplemented with \SI{1}{\milli\Molar}
isopropyl-$\beta$-D-thiogalactopyranoside (IPTG) to induce protein expression.
\SI{1}{\milli\Molar} of \ch{CoCl2} was added in each post-induction medium.
After three hours incubation at \SI{37}{\celsius}, 300 r.p.m., the cells were
harvested by using 4000 r.p.m centrifugation at \SI{4}{\celsius} for 15 minutes
and then resuspended with \SI{20}{\milli\Molar} Tris-HCl,
\SI{500}{\milli\Molar} \ch{NaCl}, \SI{5}{\milli\Molar} imidazole, 10\% glycerol
(pH 8.0) and \SI{1}{\micro\Molar} \ch{CoCl2}. Cell lysate was immediately
sonicated for 1.5 minutes at \SI{4}{\celsius} and then a clarification spin was
performed (20, 000 g, \SI{4}{\celsius}, 30 minutes).  Clarified supernatants
were loaded into a \SI{5}{\mL} His Trap column (G.E Healthcare, Piscataway, NJ)
using AKTA FPLC purifier (G.E.  Healthcare, Piscataway, NJ).  Protein elution
was generated using elution buffer B (\SI{20}{\milli\Molar} Tris-HCl,
\SI{500}{\milli\Molar} sodium chloride, \SI{500}{\milli\Molar} imidazole (pH
8.0)).  The purified samples were then transferred for buffer exchange using
\SI{12}{\L} \SI{20}{\milli\Molar} phosphate buffer (pH 8.0).  Dialyzed protein
was subjected to kinetic assays immediately.

\subsection{Enzyme Kinetics}

The protein was diluted to a final concentration of \SI{30}{\nano\Molar} in
\SI{20}{\milli\Molar} sodium phosphate (pH 8.0) by using the extinction
coefficient \SI{29280}{\per\Molar\per\cm}. Reactions were monitored
spectrophotometrically (Synergy H1, BioTek, Winooski VT) at \SI{405}{\nm} for
paraoxon (coefficient = \SI{17000}{\per\Molar\per\cm}).  Reactions for paraoxon
(\SIrange{13}{104}{\micro\Molar}) was done in 0.4\% methanol.
K\textsubscript{M} and k\textsubscript{cat} values were determined by a
Lineweaver-Burk plot.\cite{Baker2011b} The equation used is shown below
(Eq.~\ref{eqn:MM-chap2}): 
\begin{equation} 
    \frac{1}{v} =
    \frac{K\textsubscript{M}}{V\textsubscript{max}}\times\frac{1}{S} +
    \frac{1}{V\textsubscript{max}} 
    \label{eqn:MM-chap2}
\end{equation}
where S represents substrate concentration; K\textsubscript{M} represents the
substrate concentration at which the reaction rate is half of
V\textsubscript{max}. The data reported is the average of three trials and the
error represents the standard deviation of those trials. One hour incubation at
\SI{35}{\celsius}, \SI{45}{\celsius}, and \SI{55}{\celsius} samples were cooled
back to room temperature for residual kinetics assays. 

\subsection{Thermo-stability and Secondary Structure of Phosphotriesterase}

\subsubsection{Nano-DSC}

Differential scanning calorimetry (Nano-DSC, TA instrument, USA) was preformed
by using \SI{600}{\micro\L} (\SI{0.1}{\mg\per\mL}) of protein right after
dialysis. Measurements were conducted at a scan rate of
\SI{1}{\celsius\per\minute}. Signals was blanked with buffer under
the same condition.  The observed diagram was then analyzed by using
two-scaled model in NanoAnalyze software (TA instrument, USA). Cp and
T\textsubscript{m} were both determined by the NanoAnalyze simulation through
two-scaled model.

\subsubsection{Circular Dichroism}

The details are described in the section \ref{sec:cd-method}. CD spectra were
recorded on a JASCO J-815 Spectropolarimeter (Easton, MD) using Spectra Manager
software. Temperature was controlled using a Fisher Isotemp Model 3016S water
bath. Proteins concentrations were \SI{10}{\micro\Molar} in
\SI{20}{\milli\Molar} phosphate buffer (pH 8.0). \SI{20}{\milli\Molar}
phosphate buffer was used for blanking signals. To calculate ellipticities, the
following formula was used(Eq.~\ref{eqn:CD-chap2}): 
\begin{equation}
    θmrw = MRW(θobs) / (10 * c * l)
    \label{eqn:CD-chap2}
\end{equation}
where \emph{MRW} is the mean residue weight of the specific phosphotriesterase,
θobs is the observed ellipticities (mdeg), \emph{l} is the path length (cm),
\emph{c} is the concentration in \SI{}{\micro\Molar}. Spectra was recorded from
\SIrange{190}{250}{\nm} with a scan speed of \SI{1}{\nano\meter\per\minute}.

\section{Results}

\subsection{DNA Alignments And PTE Variants}

Sequence alignments of PTE from different strains is illustrated (Figure ) and
phenylalanines outside the dimer interface of PTE are highlighted. We located
those phenylalanines positions in the phosphotriesterase and identify different
OPH from different origins, including  \latin{Brevundimonas.},
\latin{Flavobaterium.}, \latin{Rhizobiacae.}, and \latin{Photorhabdus.}. Most
of phenylalanines are conserved among these variants. Phenylalanines share the
similar sizes and properties with methionine and leucine\cite{Richards1974}.
Phenylalanine has Van der Waals volume of 135\AA as methionine and leucine
exhibit 124\AA. With the consideration of residue sized and hydrophobicity,
we then chose leucine for the variants\cite{Barnes2007}. While two residues,
F150 and F335, are conservative among strains, we mutate these two into
methionine.  Eight residues are identified outside the dimer interface of
phosphotriesterase,  including F51L, F216L, F304L, F306L, F327L, and F357L.
With respect to the codon usages\cite{Sivashanmugam2009b}, we design primer
sets for each of variants of PTE.  With the site-directed mutagenesis, each
variant is mutated from the parent DNA and the sequences confirmed.  Plasmids
for expression of PTE variants were generated from our pQE30-S5PTE construct
using the polymerase chain reactions.  After 18 rounds of DNA amplification
reactions, the methylated parent DNA was digested by DpnI.  The rest of the DNA
products were then transformed into \latin{E. coli} stain BL21 for
amplification.   

\subsection{Variants Expression And Purification}

Mutant and wild type plasmids were transformed into \latin{E. coli}
phenylalanine auxotrophic strain (AF-IQ cells) and expressed and purified with
the method discussed  previously.\cite{Yang2014a} Over-expression of proteins
were confirmed by SDS-PAGE gel. Variants were dialysed in \SI{20}{\milli\Molar}
phosphate buffer (pH 8.0) after purification and concentrations of proteins
were determined by using Nano-Drop.

\subsection{Rosetta Simulation}

Using the crystal structure of PTE (1HZY), we mutated K185R, D208G, and R319S
so that S5-PTE would be accessible for Rosetta. Originally, the model of PTE
incorporated \ce{Zn^{2+}} at the active sites of protein. To ensure the accuracy
of simulation, we then changes the \ce{Zn^{2+}} to \ce{Co^{2+}}. With the
simulation of Rosetta, we performed 1000 decoys for each variants.  Wild-type
PTE has an average of -1119.93 $\pm$ 5.42 of Rosetta Energy Unit (REU). In
comparison with wild-type, three variants, F304L, F327L, and F335L, have
increased REU as -1107.29 $\pm$ 5.69, -1112.77 $\pm$ 6.27, and -1111.89 $\pm$
6.66, respectively. Among eight variants, only F306L shows lower REU score of
-1121.28. In the Figure \ref{fig:rosetta-pte-chap2}, the Rosetta simulations
generated 1000 assembles for each variants, which followed normal distribution.
The rest of variants (F51L, F150M, F216L, and F357L) showed similar REU scores
when compared with wild-type (|$\Delta$REU| < 2). 

% --------------------------
\begin{figure}[h!] \centering \includegraphics[width=1.0\textwidth]{fig2_02}
    \caption[Rosetta simulation results of PTE.]{Rosetta simulation results of PTE.}
    \label{fig:rosetta-pte-chap2}
\end{figure}
% --------------------------
% --------------------------
\begin{figure}[h!] \centering \includegraphics[width=1.0\textwidth]{fig2_01}
    \caption[RMSD comparison of PTE and its variants.]{RMSD comparison of PTE
    and its variants.} 
    \label{fig:rmsd-pte-chap2}
\end{figure}
% --------------------------
After simulation, we calculated the RMSD of variants among the superimposed
wild type PTE. With the cut-off value of \SI{0.2}{\AA}, six variants kept the
similar structures to wild type. Interestingly, the introduction of three
individual mutations, F304L, and F304L, and F335M dramatically perturbed PTE
structure (residue 200 - 300 region). More than 35 residues interactions were
affected by single mutation of F306L or F335M. 

\subsection{Variants Enzyme Kinetics}

Hydrolysis activity was evaluated by measuring the increase of
\emph{p}-nitrophenol. To assess function, we determined the Michaelis–Menten
kinetics of PTE and its variants with paraoxon. At \SI{25}{\celsius},
wild-type PTE exhibited the hydrolysis activity of
k\textsubscript{cat}/K\textsubscript{M} = \SI{170000}{\per\Molar\per\second};
F216L and F306L ware slightly higher (k\textsubscript{cat}/K\textsubscript{M} =
\SI{250000}{\per\Molar\per\second}, \SI{243000}{\per\Molar\per\second}). 

After the one hour incubation at \SIlist{35;45;55}{\celsius}, the residual
activities of wild-type and PTE variants decreased. While wild-type PTE retains
only 38\% initial activity, F51L exhibited sightly higher
k\textsubscript{cat}/K\textsubscript{M} at 50\%. We are comparing hydrolysis of
paraoxon among variants thermo-stabilities from \SI{25}{\celsius} to
\SI{55}{\celsius} with \SI{10}{\celsius} increment. The wild-type PTE lost 41\%
of its activity after \SI{45}{\celsius} incubation and only exhibited
\SI{65000}{\per\Molar\per\second} after treated at \SI{55}{\celsius}. The
results followed the pattern published before.\cite{Yang2014a} 

Previously, Gopal \latin{et al.} extensively studied the residue F306 of
PTE.\cite{Gopal2000} With the intension of enhanced activity of VX, F306A and
F306Y were introduced in PTE sequence. After we identify one of the Phe
residues outside the dimer interface, F306,  and choose to mutate it into Leu.
We compare the result of F306L with other variants. It seems that the mutation
increase the hydrolysis of paraoxon by 43\% compared with wild-type PTE at
\SI{25}{\celsius}. While the majority of variants retain their hydrolysis
efficiency, three variants, however, lose more than 30\% of
k\textsubscript{cat}/K\textsubscript{M}. F304L exhibited only 32\% of wild-type
PTE activity at \SI{25}{\celsius} (k\textsubscript{cat}/K\textsubscript{M} =
\SI{55000}{\per\Molar\per\second}). F327L and F335M, at the same time, reduce
the paraoxon hydrolysis efficiency to 69\% and 60\%, respectively.

Focusing on the thermo-stability, we then compare the variants activities at
\SIlist{35;45;55}{\celsius}. While F216L loses almost half (~48\%) of its
\SI{25}{\celsius} activity, the rest of seven variants retain more than 70\% of
their own \SI{25}{\celsius} activities. However, upon elevated temperature at
\SI{55}{\celsius}, three variants, F304L, F327L, and F335M, were inactivated in
comparison with other variants.

% --------------------------
\begin{table}[h!]
    \centering
    \begin{tabular}{llllll}
    \hline%%
    protein &  & \SI{25}{\celsius} & \SI{35}{\celsius} &
    \SI{45}{\celsius} & \SI{55}{\celsius} \\
    \hline

    \multirow{2}{*}{PTE}    & k\textsubscript{cat}/K\textsubscript{M} & 1.70 $
    \pm$ 0.20 & 1.20 $\pm$ 0.23 & 1.00 $\pm$ 0.11 & 0.65 $\pm$ 0.10 \\
    & k\textsubscript{cat} & 2.3 $\pm$ 0.5 & 2.0 $\pm$ 0.7 & 1.4 $\pm$ 0.5 & 1.3
    $\pm$ 0.5 \\
    
    \multirow{2}{*}{F51L}  & k\textsubscript{cat}/K\textsubscript{M} & 1.40
    $\pm$ 0.19 & 1.20 $\pm$ 0.20 & 1.10 $\pm$ 0.20 & 0.70 $\pm$ 0.09 \\
    & k\textsubscript{cat} & 3.0 $\pm$ 1.1 & 2.6 $\pm$ 0.5 & 1.2 $\pm$ 0.3 &
    0.8 $\pm$ 0.3 \\
    
    \multirow{2}{*}{F150M} & k\textsubscript{cat}/K\textsubscript{M} &
    1.30 $\pm$ 0.08 & 1.20 $\pm$ 0.10 & 0.90 $\pm$ 0.08 & 0.50 $\pm$ 0.07 \\
    & k\textsubscript{cat} & 2.5 $\pm$ 1.0 & 2.0 $\pm$ 1.0 & 1.4 $\pm$ 0.3 &
    1.0 $\pm$ 0.4 \\
    
    \multirow{2}{*}{F216L} & k\textsubscript{cat}/K\textsubscript{M}
    & 2.50 $\pm$ 0.13 & 1.30 $\pm$ 0.10 & 1.00 $\pm$ 0.08 & 0.09 $\pm$ 0.20 \\
    & k\textsubscript{cat} & 2.5 $\pm$ 0.6 & 2.4 $\pm$ 0.6 & 1.8 $\pm$ 0.5 &
    1.7 $\pm$ 0.6 \\
    
    \multirow{2}{*}{F304L}  & k\textsubscript{cat}/K\textsubscript{M} & 0.55
    $\pm$ 0.13 & 0.05 $\pm$ 0.01 & n.a. & n.a. \\
    & k\textsubscript{cat} & 0.6 $\pm$ 0.3 & 0.2 $\pm$ 0.1& n.a. & n.a. \\
    
    \multirow{2}{*}{F306L}  & k\textsubscript{cat}/K\textsubscript{M} & 2.43
    $\pm$ 0.23 & 1.72 $\pm$ 0.04 & 1.38 $\pm$ 0.36 & 0.90 $\pm$ 0.012 \\
    & k\textsubscript{cat} & 2.1 $\pm$ 0.6 & 1.8 $\pm$ 0.5 & 1.7 $\pm$ 0.5 &
    1.2 $\pm$ 0.6 \\
    
    \multirow{2}{*}{F327L}  & k\textsubscript{cat}/K\textsubscript{M} & 1.17
    $\pm$ 0.13 & 1.10 $\pm$ 0.19 & 1.38 $\pm$ 0.36 & n.a. \\
    & k\textsubscript{cat} & 1.5 $\pm$ 0.6 & 1.3 $\pm$ 0.5 & 0.8 $\pm$ 0.3 &
    n.a. \\
    
    \multirow{2}{*}{F335M}  & k\textsubscript{cat}/K\textsubscript{M} & 1.02
    $\pm$ 0.13 & 0.78 $\pm$ 0.10 & 0.33 $\pm$ 0.15 & n.a. \\
    & k\textsubscript{cat} & 1.3 $\pm$ 0.5 & 0.6 $\pm$ 0.5 & 0.3 $\pm$ 0.2 &
    n.a. \\
    
    \multirow{2}{*}{F357L}  & k\textsubscript{cat}/K\textsubscript{M} & 1.91
    $\pm$ 0.25 & 1.50 $\pm$ 0.13 & 1.10 $\pm$ 0.09 & 0.56 $\pm$ 0.09 \\
    & k\textsubscript{cat} & 2.5 $\pm$ 0.7 & 1.8 $\pm$ 0.3 & 1.5 $\pm$ 0.5 &
    0.4 $\pm$ 0.2 \\

    \hline
    \multicolumn{6}{l}{n.a = not available; 
        k\textsubscript{cat}/K\textsubscript{M}:
        $\times$10\textsuperscript{5}\SI{}{\per\Molar\per\second};
        k\textsubscript{cat}: \SI{}{\per\second}.}
    \end{tabular}
    \caption[Paraoxon hydrolysis efficiency summary of PTE and its variatns.
    Residual activities were preformed after incubation at
\SIlist{35;45;55}{\celsius}.] {Paraoxon hydrolysis efficiency summary of PTE
    and its variants.  Residual activities were preformed after incubation at
    \SIlist{35;45;55}{\celsius}.}
    \label{tab:kinetics-chap2-result}
\end{table}
% --------------------------

\subsection{Thermo-stability and CD of PTE Variants}

Differential Scanning Calorimetry were preformed to confirm the
thermo-stabilities of PTE and its variants. Each protein sample was heated up
from \SI{20}{\celsius} to \SI{70}{\celsius}. Two melting temperatures were
identified, and wild-type temperatures are similar to what have been published
in out lab previously\cite{Baker2011b}. Wild-type PTE exhibits
T\textsubscript{m}1 at 50.94 $\pm$ 0.03 \si{\celsius} and T\textsubscript{m}2
at 63.54 $\pm$ 0.10 \si{\celsius}. Among eight variants, two mutations show
higher T\textsubscript{m}1 temperatures and three show lower
T\textsubscript{m}1. F304L and F306L exhibited 53.72 and 52.03 \si{\celsius},
respectively. On the contrary, F51L, F150M, and F357L lower their
T\textsubscript{m}1 values by 1.48, 1.61, and 2.04 \si{\celsius}, respectively.  

% --------------------------
\begin{table}[h!]
    \centering
    \begin{tabular}{lll}
    \hline
    %%
    protein & T\textsubscript{m}1 (\si{\celsius}) & T\textsubscript{m}2 (\si{\celsius})\\
    \hline
    %%
    PTE    & 50.94 $\pm$ 0.03  & 63.54 $\pm$ 0.10 \\
    F51L & 49.46 $\pm$ 0.50 & 53.98 $\pm$ 0.13 \\
    F150M & 49.33 $\pm$ 1.14 & 60.26 $\pm$ 1.00\\
    F216L & 50.80 $\pm$ 0.40 & 63.90 $\pm$ 0.05 \\
    F304L & 53.72 $\pm$ 0.04 & 62.35 $\pm$ 0.05 \\
    F306L & 52.03 $\pm$ 0.03 & 63.13 $\pm$ 0.03 \\
    F327L & 50.16 $\pm$ 0.03 & 62.79 $\pm$ 0.50 \\
    F335M & 50.07 $\pm$ 0.08 & 62.03 $\pm$ 0.03 \\
    F357L & 48.90 $\pm$ 0.15 & 57.68 $\pm$ 0.35 \\

    \hline      
    \end{tabular}
    \caption[Differential scanning calorimetry results of PTE and its variants.]
    {Differential scanning calorimetry results of PTE and its variants.}
    \label{tab:dsc-chap2-result}

\end{table}
% --------------------------

T\textsubscript{m}2 of eight variants also exhibit mixed results with five
decreased temperatures and one slight-increased temperature. Among eight
variants, F51L and F357L show lowest temperatures of T\textsubscript{m}2 at
52.98 and 57.68 \si{\celsius}. The highest temperature of T\textsubscript{m}2
exists in variant F216L. Overall, while there are mixed results among
T\textsubscript{m}1 and T\textsubscript{m}2, we expect slight differences as
these phenylalanines are replaced with leucines and methionines that share
similar sizes and properties.  

\section{Discussion}

Wild-type PTE and its variants were successfully expressed from \latin{E.
coli}. After three hours incubation, IPTG induced the over-expression of
proteins. Each variant and wild-type protein were then purified by using FPLC.

Among eight variants, F304L and F306L show dramatic differences in terms of
paraoxon hydrolysis efficiency. (Table \ref{tab:kinetics-chap2-result})

While replacements of phenylalanines with methionines (F150M, F335M) or
leucines (F51L, F216L, F304L, F306L, F327L ,and F357L) were expected to
maintain the majority of structure and property of PTE, we found out that
kinetic results behaved differently.  

Among eight variants, F51L and F150M PTE hydrolyzed paraoxon similarly to
wild-type PTE (k\textsubscript{cat}/K\textsubscript{M}
\SI{140000}{\per\Molar\per\second} and \SI{130000}{\per\Molar\per\second},
respectively). With REU scores of -1118.75 and -1118.31, these two variants
slightly changed their structures. In addition to these two neutral variants,
F357L also exhibited similar hydrolysis efficiency compared with wild-type.
F357L was scored at -1117.16 from Rosetta, with
k\textsubscript{cat}/K\textsubscript{M} of \SI{191000}{\per\Molar\per\second}.
While both scores and kinetics numbers are within the error, these mutations
turn out to be neutral in PTE sequence despite the minor changes in REU.
Considering the sizes and properties of methionine, leucine, and phenylalanine,
we conclude that the replacement at F51L, F150M, and F357L can be selectively
combined without sacrificing PTE hydrolysis efficiency on paraoxon.    

Raushel group demonstrated that the loop 7 region were important for hydrolysis
of PTE. Among hundreds of residues in PTE, F306 position was studied
intensively, and it had been shown that the mutations, F306Y and F306H, altered
the efficiency of hydrolysis\cite{Pavelka2009, Chen-Goodspeed2001a}. To invest
the behaviors of F304L and F306L, we render the simulated structures through
Rosetta. After minimization. 

While focusing on key residues that were
discussed\cite{Bigley2013b,Chen-Goodspeed2001a,Pavelka2009}, we focus on the
comparisons of three variant, F304L, F306L, and F335M. With the REU score of
-1121.28, F304L showed enhanced k\textsubscript{cat}/K\textsubscript{M}
compared with wild-type. To understand the interactions between F306L, we look
into the RMSD of F306L. Surprisingly, there was no significant change in RMSD
of F216L as the hydrolysis efficiency of this variant was increased. With only
nine residues that were mildly perturbed, F216L exhibited similar REU score
compared with wild-type PTE (F216: -1118.74 $\pm$ 5.57; wild-type: -1119.94
$\pm$ 5.42).  While two residues, F304 and F306, are
close to each other, the effects from these two mutations are opposite. F304L
shows impaired catalytic efficiency on paraoxon
(k\textsubscript{cat}/K\textsubscript{M} =  ) with REU of . However, on the
contrary, F306L. 

The results confirmed that the replacement of F306 affected the hydrolysis of
PTE substrates. 

Upon heating up to \SIlist{35;45;55}{\celsius}, variants behaved differently
compared with wild-type PTE. As compared with Rosetta REU results, the
simulation results fitted the kinetic parameters at \SI{55}{\celsius}. 

\printbibliography[heading=subbibliography]

\end{refsection}

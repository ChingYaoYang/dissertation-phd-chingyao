\chapter{Effects of Phenylalanines Outside Dimer Interface of Phosphotriesterase}
\label{chap:dimer}
\begin{refsection}

\section{Introduction}

\subsection{Phosphotriesterase}

PTE is a homodimeric protein composed of two monomers, each of which contains a
metallo-active site. Phosphotriesterase (PTE) are enzymes, which hydrolyze
organophosphates (OPs) as well as synthetic esters (Figure
\ref{fig:pte-structure})\cite{Ghanem2005a}. The proenzyme form of PTE contains
29 amino acids signal peptide at the N-terminus. It is originally found as a
39kDa monomeric form in the solution\cite{Mulbry1989}. Later, the proenzyme of
PTE is engineered and expressed in the form of mature protein from \latin{E.
coli}. A ($\beta$/$\alpha$)\textsubscript{8} TIM-barrel structure forms the
monomeric PTE\cite{Roodveldt2005,Seibert2005}. The globular monomer is roughly
51\AA $\times$ 55\AA $\times$ 51\AA.  OPs are a synthetic class of small molecule
that irreversibly inactivate acetylcholinesterase (AChE), disrupting
neural transmission. AChE is an enzyme that degrades the neurotransmitter,
acetylcholine, at the neuromuscular junction in the cholinergic nervous system.
After the acetylcholine is hydrolyzed, the synaptic transmission would be
terminated. Inhibition of AChE lead to hyper-stimulation from toxic
accumulation of acetylcholine\cite{Soreq2001}. Army also adapted this protein
for chemical weapons neutralization \cite{Yang2014a}.

\subsection{Dimer Interface of Phosphotriesterase}



\subsection{Metal Ions Effects At Active Site}

\subsection{Side-chain Effects}

\section{Methods}

\subsection{General}

All chemicals, reagents, and substrate were purchased from Sigma. T4 DNA ligase
was purchased from Roche. DNA sequence was confirmed by Eurofins MWG Operon.
96-well plates were purchased from Thermo Fisher Scientific (Waltham,
MA)\cite{Yang2014a}.

\subsection{Rosetta Design of Phosphotriesterase}

A symmetric starting model of wild type PTE from the B chain of PDB structure
1HZY\cite{Benning2001a} was built using the Rosetta suite of macromolecular
modeling tools\cite{Leaver-Fay2011}. Both active site \ch{Zn^{2+}} ions were
replaced with \ch{Co^{2+}} to reflect the metal used in the experimentally
produced mutants.  Distance constraints between the cobalt cations and the
coordinating residues were taken from PDB structure 3A4J\cite{Jackson2009b}.
Torsional and partial charge parameters for the non-standard carboxylated
lysine residue (Lys 169) were calculated quantum mechanically using the
HF/6-31G(d) level of theory in Gaussian09\cite{Frisch2009} with an overall
charge of -1.  Rotamer libraries for the carboxylated lysine were generated
with the Rosetta MakeRotLib\cite{Renfrew2012b} protocol.  Models were
constructed for each of the point mutations: F51L, F150M, F216L, F304L, F306L,
F327L, F335M, and F357L using the Rosetta fixbb (fixed backbone design)
protocol with symmetry\cite{DiMaio2011a}.  \ch{Co^{2+}} coordinating residues
were held fixed to their native rotamers. To propagate point mutation effects
throughout a mutant model, the Rosetta relax protocol was used to repack and
minimize the entire PTE structure with backbone flexibility. For each point
mutant, an ensemble of 500 relaxed decoys were generated. Interatomic distances
between \ch{Co^{2+}} and coordinating residues were enforced with harmonic
constraints.  The change in stability for a mutation was calculated as the
difference between the mutant and wild type ensemble averages of the total
Rosetta score. All protocols used here included the native rotamers and extra
rotamers sampling as additional parameters. All decoys were scored using the
talaris2013 score function\cite{Leaver-Fay2013a}.

\subsection{Variants of Phosphotriesterase}
Purified protein product was assayed for concentration by way of a Thermo

\subsection{Biosynthesis}

In anticipation for the need of large quantities of protein mass for the \ldots
delete comtent of the supernatant and storage at \SI{-20}{\celsius}.

\subsection{Protein Purification}
The purification was described previously in the section
\ref{sec:protein-expression-method}. All solutions used in the extraction and
purification of recombinant proteins \ldots delete content 5 CVs of buffer
prior to each injection.

\subsection{Enzyme Kinetics}

The protein was diluted to a final concentration of \SI{30}{\nano\Molar} in
\SI{20}{\milli\Molar} sodium phosphate (pH 8.0) by using the extinction
coefficient \SI{29280}{\per\Molar\per\cm}. Reactions were monitored
spectrophotometrically (Synergy H1, BioTek, Winooski VT) at \SI{405}{\nm} for
paraoxon (coefficient = \SI{17000}{\per\Molar\per\cm}).  Reactions for paraoxon
(\SIrange{13}{104}{\micro\Molar}) was done in 0.4\% methanol.
K\textsubscript{M} and k\textsubscript{cat} values were determined by a
Lineweaver-Burk plot.\cite{Baker2011b} The equation used is shown below
(Eq.~\ref{eqn:MM-chap2}): 
\begin{equation} 
    \frac{1}{v} =
    \frac{K\textsubscript{M}}{V\textsubscript{max}}\times\frac{1}{S} +
    \frac{1}{V\textsubscript{max}} 
    \label{eqn:MM-chap2}
\end{equation}
where S represents substrate concentration; K\textsubscript{M} represents the
substrate concentration at which the reaction rate is half of
V\textsubscript{max}. The data reported is the average of three trials and the
error represents the standard deviation of those trials.

\subsection{Thermo-stability and Secondary Structure of Phosphotriesterase}

\subsubsection{Nano-DSC}

The details are described in the section \ref{sec:dsc-method}. DSC (Nano-DSC,
TA instrument, USA) was preformed by using \SI{600}{\micro\L}
(\SI{0.1}{\mg\per\mL}) of protein right after dialysis. Measurements were
conducted at a scan rate of \SI{1}{\celsius\per\minute}. Signals was blanked with
buffer under the same condition.  The observed diagram was then analyzed by
using NanoAnalyze software.

\subsubsection{Circular Dichroism}

The details are described in the section \ref{sec:cd-method}. CD spectra were
recorded on a JASCO J-815 Spectropolarimeter (Easton, MD) using Spectra Manager
software. Temperature was controlled using a Fisher Isotemp Model 3016S water
bath. Proteins concentrations were \SI{10}{\micro\Molar} in
\SI{20}{\milli\Molar} phosphate buffer (pH 8.0). \SI{20}{\milli\Molar}
phosphate buffer was used for blanking signals. To calculate ellipticities, the
following formula was used(Eq.~\ref{eqn:CD-chap2}): 
\begin{equation}
    θmrw = MRW(θobs) / (10 * c * l)
    \label{eqn:CD-chap2}
\end{equation}
where \emph{MRW} is the mean residue weight of the specific phosphotriesterase,
θobs is the observed ellipticities (mdeg), \emph{l} is the path length (cm),
\emph{c} is the concentration in \SI{}{\micro\Molar}. Spectra was recorded from
\SIrange{190}{250}{\nm} with a scan speed of \SI{1}{\nano\meter\per\minute}.

\section{Results}

\subsection{DNA Alignments And PTE Variants}

Auto-induction growth can be sustained in baffled shaker flasks, according to
\ldots delete content to purification.

\subsection{Variants Expression And Purification}

\subsection{Variants Enzyme Kinetics}

Endotoxin levels of the protein were measured using a limulus amebocyte lysate
catabolic events reported in Section.

\subsection{Thermo-stability and CD of PTE Variants}

\section{Discussion}

\printbibliography[heading=subbibliography]

\end{refsection}
